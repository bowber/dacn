% ============================================================
% Chương 5: Kết luận và kế hoạch Khóa luận Tốt nghiệp
% ============================================================

\newpage
\section{KẾT LUẬN VÀ KẾ HOẠCH KHÓA LUẬN TỐT NGHIỆP}

% ------------------------------------------------------------
\subsection{Kết luận}
% ------------------------------------------------------------

Báo cáo đã thực hiện các nội dung sau:

\begin{enumerate}
    \item \textbf{Phân tích hiện trạng vận hành:}
    \begin{itemize}
        \item Xác định các nguồn lãng phí năng lượng khi vận hành với van làm mát mở 100\% và reboiler chạy 100\%.
        \item Phát hiện hệ thống có công suất ngưng tụ dư thừa, gây lãng phí 72\% lưu lượng nước làm mát.
    \end{itemize}
    
    \item \textbf{Xây dựng cơ sở lý thuyết:}
    \begin{itemize}
        \item Trình bày cơ sở lý thuyết quá trình chưng cất và cân bằng năng lượng.
        \item Thiết lập các phương trình cân bằng năng lượng cho hệ thống chưng cất.
    \end{itemize}
    
    \item \textbf{Đề xuất giải pháp tối ưu năng lượng:}
    \begin{itemize}
        \item Thiết kế vòng điều khiển áp suất đỉnh tháp PC-01 với bộ điều khiển PID để tiết kiệm nước làm mát.
        \item Đề xuất phương án điều khiển PWM công suất reboiler để tiết kiệm điện năng (triển khai trong LVTN).
    \end{itemize}
    
    \item \textbf{Tính toán:}
    \begin{itemize}
        \item Xây dựng chương trình tính toán cân bằng năng lượng.
        \item Tính toán công suất reboiler hiệu dụng: $Q_{reboiler} = 5400$ W.
        \item Tính toán độ mở van tối ưu: 28\% (tiết kiệm 72\% nước làm mát).
    \end{itemize}
\end{enumerate}

\textbf{Kết quả tính toán chính:}

\begin{table}[H]
\centering
\begin{tabular}{|l|c|c|}
\hline
\textbf{Thông số} & \textbf{Baseline} & \textbf{Điều khiển van} \\
\hline
Công suất reboiler & 6000 W & 6000 W \\
\hline
Độ mở van & 100\% & 28\% \\
\hline
Lưu lượng nước & 7.2 L/min & 2.0 L/min \\
\hline
\end{tabular}
\caption{Tổng hợp kết quả tính toán}
\label{tab:conclusion_results}
\end{table}

\textbf{Kết luận tổng quát:}
\begin{itemize}
    \item \textbf{Vòng điều khiển áp suất PC-01:} Tiết kiệm 72\% lưu lượng nước làm mát bằng cách điều khiển van theo áp suất đỉnh tháp.
    \item \textbf{Điều khiển PWM reboiler:} Đã đề xuất phương án, việc triển khai và đánh giá hiệu quả tiết kiệm điện năng sẽ được thực hiện trong Khóa luận Tốt nghiệp.
    \item Giải pháp không đòi hỏi thay đổi cấu hình thiết bị, chỉ cần bổ sung hệ thống điều khiển tự động.
\end{itemize}

% ------------------------------------------------------------
\subsection{Kế hoạch Khóa luận Tốt nghiệp}
% ------------------------------------------------------------

Để kiểm chứng kết quả tính toán và triển khai thực tế, kế hoạch Khóa luận Tốt nghiệp bao gồm các nội dung sau:

\begin{enumerate}
    \item \textbf{Thiết kế và lập trình hệ thống điều khiển:}
    \begin{itemize}
        \item Thiết kế bộ điều khiển PID cho vòng điều khiển áp suất PC-01.
        \item Lập trình PLC S7-1200 trên TIA Portal.
        \item Thiết kế giao diện HMI để giám sát và điều khiển.
    \end{itemize}
    
    \item \textbf{Thí nghiệm vòng điều khiển áp suất PC-01:}
    \begin{itemize}
        \item Triển khai thử nghiệm trên hệ thống chưng cất tại phòng thí nghiệm.
        \item Thu thập dữ liệu vận hành: áp suất, lưu lượng, nhiệt độ.
        \item So sánh kết quả thực nghiệm với kết quả tính toán.
        \item Kiểm chứng hiệu quả bằng độ subcooling ($\Delta T_{subcool}$).
    \end{itemize}
    
    \item \textbf{Triển khai điều khiển PWM công suất reboiler:}
    \begin{itemize}
        \item Lập trình điều khiển PWM trên PLC S7-1200.
        \item Xác định duty cycle tối ưu bằng thực nghiệm.
        \item Đánh giá hiệu quả tiết kiệm điện năng.
        \item Nghiên cứu ảnh hưởng đến chất lượng sản phẩm.
    \end{itemize}
    
    \item \textbf{Đánh giá tổng thể:}
    \begin{itemize}
        \item Tổng hợp kết quả tiết kiệm năng lượng (nước làm mát + điện năng reboiler).
        \item Phân tích độ nhạy và giới hạn của phương pháp.
    \end{itemize}
\end{enumerate}

% ------------------------------------------------------------
% Bảng kế hoạch thực hiện
% ------------------------------------------------------------

\begin{table}[H]
\centering
\begin{tabular}{|c|p{10cm}|}
\hline
\textbf{Tuần} & \textbf{Nội dung công việc} \\
\hline
1--2 & Sửa chữa/thay thế cảm biến hỏng, hiệu chuẩn thiết bị đo \\
\hline
3--4 & Thiết kế bộ điều khiển PID, lập trình PLC S7-1200 (TIA Portal) \\
\hline
5--6 & Thiết kế giao diện HMI, tích hợp hệ thống \\
\hline
7--8 & Thí nghiệm vòng điều khiển áp suất PC-01, thu thập dữ liệu, tinh chỉnh PID \\
\hline
9--10 & Triển khai điều khiển PWM reboiler, đánh giá hiệu quả tiết kiệm năng lượng \\
\hline
11--12 & Hoàn thiện báo cáo, chuẩn bị bảo vệ đồ án \\
\hline
\end{tabular}
\caption{Kế hoạch thực hiện Khóa luận Tốt nghiệp}
\label{tab:schedule}
\end{table}
