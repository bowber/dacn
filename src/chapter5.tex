% ============================================================
% Chương 5: Mô phỏng và kết quả
% ============================================================

\newpage
\section{MÔ PHỎNG VÀ KẾT QUẢ}

% ------------------------------------------------------------
\subsection{Thiết lập mô phỏng}
% ------------------------------------------------------------

\subsubsection{Công cụ mô phỏng}

Mô phỏng được thực hiện bằng ngôn ngữ Python với các thư viện:
\begin{itemize}
    \item \texttt{numpy}: Tính toán số học
    \item \texttt{scipy}: Giải phương trình vi phân, tối ưu hóa
    \item \texttt{matplotlib}: Vẽ đồ thị
    \item \texttt{control}: Phân tích hệ thống điều khiển
\end{itemize}

\subsubsection{Thông số mô phỏng}

Các thông số hệ thống được sử dụng trong mô phỏng:

\begin{table}[H]
\centering
\begin{tabular}{|l|c|c|}
\hline
\textbf{Thông số} & \textbf{Giá trị} & \textbf{Đơn vị} \\
\hline
Công suất reboiler tối đa & 6000 & W \\
\hline
Lưu lượng nước làm mát tối đa & 7.2 & L/min \\
\hline
Diện tích trao đổi nhiệt condenser & 0.28 & m$^2$ \\
\hline
Áp suất vận hành & 1.0 & bar \\
\hline
Nồng độ nhập liệu & 10 & \% vol \\
\hline
Lưu lượng nhập liệu & 4.8 & L/h \\
\hline
\end{tabular}
\caption{Thông số hệ thống cho mô phỏng}
\label{tab:simulation_params}
\end{table}

% ------------------------------------------------------------
\subsection{Kết quả mô phỏng}
% ------------------------------------------------------------

Mô phỏng được chạy trong 60 phút (3600 giây) với bước thời gian $\Delta t = 1$ giây.

\subsubsection{Mô phỏng vận hành baseline (van mở 100\%)}

Trong chế độ vận hành baseline, van nước làm mát được mở hoàn toàn (100\%) -- đây là cách vận hành truyền thống không có điều khiển tự động.

\begin{figure}[H]
\centering
\includegraphics[width=0.9\textwidth]{../simulation/outputs/fig_baseline_combined.png}
\caption{Mô phỏng vận hành baseline: Áp suất và lưu lượng sản phẩm khi van làm mát mở 100\%}
\label{fig:sim_baseline}
\end{figure}

Kết quả mô phỏng baseline cho thấy:
\begin{itemize}
    \item Áp suất đỉnh tháp ổn định tại ~1.0 bar (tháp hở khí quyển).
    \item Van nước làm mát mở 100\% -- gây lãng phí nước làm mát.
    \item Công suất ngưng tụ: 6180 W (vượt quá nhu cầu thực tế 4500 W từ reboiler).
    \item Lưu lượng sản phẩm đỉnh: 1.53 L/h.
    \item Nhiệt độ đỉnh ổn định tại 78.2°C.
\end{itemize}

\subsubsection{Mô phỏng với điều khiển áp suất PID}

Bộ điều khiển PID được cài đặt với các thông số:
\begin{itemize}
    \item $K_c = 400$ -- Độ lợi bộ điều khiển
    \item $T_i = 30$ s -- Hằng số thời gian tích phân
    \item $T_d = 0$ s -- Hằng số thời gian vi phân (điều khiển PI)
    \item Setpoint: 1.0 bar
\end{itemize}

Mô phỏng bao gồm nhiễu loạn: tăng 10\% công suất reboiler tại $t = 600$ s (phút thứ 10).

\begin{figure}[H]
\centering
\includegraphics[width=0.9\textwidth]{../simulation/outputs/fig_pid_combined.png}
\caption{Mô phỏng điều khiển PID: Áp suất, độ mở van và lưu lượng sản phẩm}
\label{fig:sim_pressure_response}
\end{figure}

\begin{figure}[H]
\centering
\includegraphics[width=0.9\textwidth]{../simulation/outputs/fig_pid_valve.png}
\caption{Đáp ứng độ mở van làm mát theo thời gian với điều khiển PID}
\label{fig:sim_valve_position}
\end{figure}

Kết quả mô phỏng PID cho thấy:
\begin{itemize}
    \item Áp suất được duy trì ổn định tại 1.0 bar (giống baseline).
    \item Van làm mát mở trung bình 72\% (so với 100\% baseline) -- tiết kiệm 28\% nước.
    \item Công suất ngưng tụ: 4812 W (phù hợp với reboiler 4500 W).
    \item Khi có nhiễu (+10\% công suất reboiler tại $t = 600$ s), van tự động tăng từ 68\% lên 73\%.
    \item Độ lệch áp suất tối đa khi có nhiễu: < 1 mbar.
\end{itemize}

% ------------------------------------------------------------
\subsection{So sánh hiệu quả vận hành}
% ------------------------------------------------------------

Bảng \ref{tab:sim_productivity_comparison} so sánh kết quả mô phỏng giữa hai phương án vận hành.

\begin{table}[H]
\centering
\begin{tabular}{|l|c|c|c|}
\hline
\textbf{Thông số} & \textbf{Baseline} & \textbf{PID Control} & \textbf{Đơn vị} \\
\hline
Áp suất xác lập & 1.00 & 1.00 & bar \\
\hline
Độ mở van xác lập & 100 & 73 & \% \\
\hline
Công suất ngưng tụ & 6180 & 4812 & W \\
\hline
Lưu lượng nước làm mát & 7.2 & 5.3 & L/min \\
\hline
Lưu lượng sản phẩm D & 1.53 & 1.19 & L/h \\
\hline
Nhiệt độ đỉnh tháp & 78.2 & 78.3 & °C \\
\hline
\end{tabular}
\caption{So sánh kết quả mô phỏng giữa baseline và PID control}
\label{tab:sim_productivity_comparison}
\end{table}

\begin{figure}[H]
\centering
\includegraphics[width=0.95\textwidth]{../simulation/outputs/fig_comparison_all.png}
\caption{So sánh toàn diện giữa baseline (van 100\%) và điều khiển PID}
\label{fig:sim_productivity_comparison}
\end{figure}

\textbf{Các chỉ số cải thiện chính:}
\begin{itemize}
    \item Giảm độ mở van: từ 100\% xuống 73\% (giảm 27\%)
    \item Giảm lưu lượng nước làm mát: từ 7.2 L/min xuống 5.3 L/min (tiết kiệm 27\%)
    \item Công suất ngưng tụ phù hợp với reboiler: 4812 W $\approx$ 4500 W
    \item Tự động điều chỉnh khi có nhiễu loạn (+10\% công suất reboiler)
\end{itemize}

% ------------------------------------------------------------
\subsection{Thảo luận}
% ------------------------------------------------------------

\subsubsection{Phân tích kết quả mô phỏng}

Kết quả mô phỏng cho thấy:
\begin{enumerate}
    \item \textbf{Cân bằng năng lượng}: Điều khiển PID tự động điều chỉnh độ mở van để công suất ngưng tụ (4812 W) phù hợp với công suất reboiler (4500 W). Điều này đảm bảo quá trình ngưng tụ hiệu quả mà không lãng phí nước làm mát.
    
    \item \textbf{Tiết kiệm nước làm mát}: Van làm mát giảm từ 100\% xuống 73\%, tiết kiệm 27\% lưu lượng nước làm mát. Đây là kết quả của việc sử dụng đúng lượng nước cần thiết thay vì mở van tối đa.
    
    \item \textbf{Áp suất ổn định}: Do tháp hở với khí quyển, áp suất đỉnh luôn xấp xỉ 1.0 bar. Điều này là đặc điểm cố hữu của tháp chưng cất quy mô phòng thí nghiệm.
    
    \item \textbf{Đáp ứng nhiễu tốt}: Khi công suất reboiler tăng 10\%, bộ điều khiển tự động tăng độ mở van từ 68\% lên 73\% để duy trì cân bằng năng lượng.
\end{enumerate}

\subsubsection{Giải thích cơ chế hoạt động}

\textbf{Nguyên lý vật lý quan trọng:} Hệ số truyền nhiệt $U$ phụ thuộc vào lưu lượng nước làm mát:
\begin{equation}
    Q = U \cdot A \cdot \Delta T_{LMTD}
\end{equation}
trong đó $U$ phụ thuộc vào số Reynolds ($Re$) và số Nusselt ($Nu$):
\begin{itemize}
    \item Lưu lượng cao $\rightarrow$ $Re$ cao $\rightarrow$ $Nu$ cao $\rightarrow$ $U$ cao
    \item Với dòng chảy rối: $U \propto (\text{lưu lượng})^{0.8}$
\end{itemize}

Khi van mở 100\%:
\begin{itemize}
    \item Lưu lượng nước: 7.2 L/min $\rightarrow$ $U = 800$ W/(m²·K)
    \item Công suất ngưng tụ: $Q_c = 6180$ W $>$ $Q_{reboiler} = 4500$ W
    \item Dư thừa ~1700 W công suất làm mát -- lãng phí nước.
\end{itemize}

Với điều khiển PID (van 73\%):
\begin{itemize}
    \item Lưu lượng nước: 5.3 L/min $\rightarrow$ $U = 570$ W/(m²·K)
    \item Công suất ngưng tụ: $Q_c = 4812$ W $\approx$ $Q_{reboiler} = 4500$ W
    \item Cân bằng năng lượng tối ưu -- tiết kiệm 27\% nước làm mát.
\end{itemize}

\subsubsection{Tính toán năng suất từ lưu lượng hồi lưu}

Lưu lượng sản phẩm đỉnh được tính gián tiếp qua lưu lượng hồi lưu (đo được qua lưu lượng kế):
\begin{equation}
    D = \frac{L}{R}
\end{equation}

Trong đó:
\begin{itemize}
    \item $D$: Lưu lượng sản phẩm đỉnh (L/h)
    \item $L$: Lưu lượng hồi lưu (L/h) -- đo trực tiếp
    \item $R$: Tỷ số hồi lưu -- xác định từ thiết kế
\end{itemize}

\subsubsection{Hạn chế của mô phỏng}

\begin{itemize}
    \item Mô hình đơn giản hóa, chưa xét đầy đủ các yếu tố phi tuyến.
    \item Chưa xem xét ảnh hưởng của nhiễu loạn (thay đổi nồng độ nhập liệu, nhiệt độ nước làm mát).
    \item Cần kiểm chứng thực nghiệm để xác nhận kết quả mô phỏng.
\end{itemize}
