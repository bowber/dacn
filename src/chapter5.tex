% ============================================================
% Chương 5: Mô phỏng và kết quả
% ============================================================

\newpage
\section{MÔ PHỎNG VÀ KẾT QUẢ}

% ------------------------------------------------------------
\subsection{Thiết lập mô phỏng}
% ------------------------------------------------------------

\subsubsection{Công cụ mô phỏng}

Mô phỏng được thực hiện bằng ngôn ngữ Python với các thư viện:
\begin{itemize}
    \item \texttt{numpy}: Tính toán số học
    \item \texttt{scipy}: Giải phương trình vi phân, tối ưu hóa
    \item \texttt{matplotlib}: Vẽ đồ thị
    \item \texttt{control}: Phân tích hệ thống điều khiển
\end{itemize}

\subsubsection{Thông số mô phỏng}

Các thông số hệ thống được sử dụng trong mô phỏng:

\begin{table}[H]
\centering
\begin{tabular}{|l|c|c|}
\hline
\textbf{Thông số} & \textbf{Giá trị} & \textbf{Đơn vị} \\
\hline
Công suất reboiler tối đa & 6000 & W \\
\hline
Lưu lượng nước làm mát tối đa & 7.2 & L/min \\
\hline
Diện tích trao đổi nhiệt condenser & 0.28 & m$^2$ \\
\hline
Áp suất vận hành & 1.1 & bar \\
\hline
Nồng độ nhập liệu & 15 & \% vol \\
\hline
Lưu lượng nhập liệu & 4.8 & L/h \\
\hline
\end{tabular}
\caption{Thông số hệ thống cho mô phỏng}
\label{tab:simulation_params}
\end{table}

% ------------------------------------------------------------
\subsection{Kết quả mô phỏng}
% ------------------------------------------------------------

\textit{(Phần này sẽ được cập nhật sau khi hoàn thành mô phỏng Python)}

\subsubsection{Mô phỏng vận hành baseline (van mở 100\%)}

% Placeholder cho kết quả mô phỏng baseline
\begin{figure}[H]
\centering
\fbox{\parbox{0.8\textwidth}{\centering\vspace{4cm}Đồ thị mô phỏng năng suất sản phẩm -- baseline\\(Sẽ cập nhật từ kết quả mô phỏng Python)\vspace{4cm}}}
\caption{Mô phỏng năng suất sản phẩm khi van làm mát mở 100\%}
\label{fig:sim_baseline}
\end{figure}

\subsubsection{Mô phỏng với điều khiển áp suất tự động}

% Placeholder cho kết quả mô phỏng tối ưu
\begin{figure}[H]
\centering
\fbox{\parbox{0.8\textwidth}{\centering\vspace{4cm}Đồ thị mô phỏng đáp ứng áp suất với PID\\(Sẽ cập nhật từ kết quả mô phỏng Python)\vspace{4cm}}}
\caption{Mô phỏng đáp ứng áp suất đỉnh tháp với bộ điều khiển PID}
\label{fig:sim_pressure_response}
\end{figure}

\begin{figure}[H]
\centering
\fbox{\parbox{0.8\textwidth}{\centering\vspace{4cm}Đồ thị mô phỏng độ mở van làm mát\\(Sẽ cập nhật từ kết quả mô phỏng Python)\vspace{4cm}}}
\caption{Mô phỏng độ mở van làm mát theo thời gian}
\label{fig:sim_valve_position}
\end{figure}

% ------------------------------------------------------------
\subsection{So sánh hiệu quả năng suất}
% ------------------------------------------------------------

\textit{(Phần này sẽ được cập nhật sau khi hoàn thành mô phỏng Python)}

% Placeholder cho bảng so sánh
\begin{table}[H]
\centering
\begin{tabular}{|l|c|c|c|}
\hline
\textbf{Thông số} & \textbf{Baseline} & \textbf{Tối ưu} & \textbf{Chênh lệch} \\
\hline
Lưu lượng sản phẩm đỉnh (L/h) & --- & --- & --- \\
\hline
Lưu lượng hồi lưu (L/h) & --- & --- & --- \\
\hline
Tỷ số hồi lưu & --- & --- & --- \\
\hline
Độ mở van làm mát (\%) & 100 & --- & --- \\
\hline
Tiêu thụ nước làm mát (L/h) & --- & --- & --- \\
\hline
\end{tabular}
\caption{So sánh hiệu quả năng suất giữa hai phương án (từ mô phỏng)}
\label{tab:sim_productivity_comparison}
\end{table}

\begin{figure}[H]
\centering
\fbox{\parbox{0.8\textwidth}{\centering\vspace{4cm}Đồ thị so sánh năng suất sản phẩm\\(Sẽ cập nhật từ kết quả mô phỏng Python)\vspace{4cm}}}
\caption{So sánh năng suất sản phẩm đỉnh giữa hai phương án vận hành}
\label{fig:sim_productivity_comparison}
\end{figure}

% ------------------------------------------------------------
\subsection{Thảo luận}
% ------------------------------------------------------------

\subsubsection{Phân tích kết quả mô phỏng}

\textit{(Phần này sẽ được cập nhật sau khi hoàn thành mô phỏng Python)}

Dự kiến kết quả mô phỏng sẽ cho thấy:
\begin{enumerate}
    \item Phương án điều khiển áp suất đỉnh tháp giúp tăng năng suất sản phẩm đỉnh.
    \item Độ mở van làm mát trung bình thấp hơn đáng kể so với baseline (100\%).
    \item Chất lượng điều khiển đạt yêu cầu: thời gian xác lập ngắn, độ quá điều chỉnh nhỏ.
    \item Hệ thống ổn định trong các điều kiện vận hành khác nhau.
\end{enumerate}

\subsubsection{Giải thích cơ chế tăng năng suất}

Khi van làm mát mở 100\%, công suất ngưng tụ vượt quá nhu cầu thực tế:
\begin{itemize}
    \item Hơi ethanol ngưng tụ quá mức, tạo lượng hồi lưu lớn hơn cần thiết.
    \item Tỷ số hồi lưu cao làm giảm năng suất sản phẩm đỉnh.
    \item Với cùng năng lượng cung cấp, hệ thống không đạt năng suất tối ưu.
\end{itemize}

Khi điều khiển áp suất ở mức tối ưu:
\begin{itemize}
    \item Van làm mát chỉ mở vừa đủ để duy trì áp suất ổn định.
    \item Giảm lượng ngưng tụ không cần thiết, giảm tỷ số hồi lưu.
    \item Năng suất sản phẩm đỉnh tăng trong khi giữ nguyên chất lượng và năng lượng.
\end{itemize}

\subsubsection{Tính toán năng suất từ lưu lượng hồi lưu}

Lưu lượng sản phẩm đỉnh được tính gián tiếp qua lưu lượng hồi lưu (đo được qua lưu lượng kế):
\begin{equation}
    D = \frac{L}{R}
\end{equation}

Trong đó:
\begin{itemize}
    \item $D$: Lưu lượng sản phẩm đỉnh (L/h)
    \item $L$: Lưu lượng hồi lưu (L/h) -- đo trực tiếp
    \item $R$: Tỷ số hồi lưu -- xác định từ thiết kế
\end{itemize}

\subsubsection{Hạn chế của mô phỏng}

\begin{itemize}
    \item Mô hình đơn giản hóa, chưa xét đầy đủ các yếu tố phi tuyến.
    \item Chưa xem xét ảnh hưởng của nhiễu loạn (thay đổi nồng độ nhập liệu, nhiệt độ nước làm mát).
    \item Cần kiểm chứng thực nghiệm để xác nhận kết quả mô phỏng.
\end{itemize}
