% ============================================================
% Chương 5: Mô phỏng và kết quả
% ============================================================

\newpage
\section{MÔ PHỎNG VÀ KẾT QUẢ}

% ------------------------------------------------------------
\subsection{Thiết lập mô phỏng}
% ------------------------------------------------------------

\subsubsection{Công cụ mô phỏng}

Mô phỏng được thực hiện bằng ngôn ngữ Python với các thư viện:
\begin{itemize}
    \item \texttt{numpy}: Tính toán số học
    \item \texttt{scipy}: Giải phương trình vi phân, tối ưu hóa
    \item \texttt{matplotlib}: Vẽ đồ thị
    \item \texttt{control}: Phân tích hệ thống điều khiển
\end{itemize}

\subsubsection{Thông số mô phỏng}

Các thông số hệ thống được sử dụng trong mô phỏng:

\begin{table}[H]
\centering
\begin{tabular}{|l|c|c|}
\hline
\textbf{Thông số} & \textbf{Giá trị} & \textbf{Đơn vị} \\
\hline
Công suất reboiler tối đa & 6000 & W \\
\hline
Lưu lượng nước làm mát tối đa & 7.2 & L/min \\
\hline
Diện tích trao đổi nhiệt condenser & 0.28 & m$^2$ \\
\hline
Áp suất vận hành & 1.0 & bar \\
\hline
Nồng độ nhập liệu & 10 & \% vol \\
\hline
Lưu lượng nhập liệu & 4.8 & L/h \\
\hline
\end{tabular}
\caption{Thông số hệ thống cho mô phỏng}
\label{tab:simulation_params}
\end{table}

% ------------------------------------------------------------
\subsection{Tính toán cân bằng năng lượng}
% ------------------------------------------------------------

Trước khi mô phỏng, cần tính toán cân bằng năng lượng để xác định công suất ngưng tụ tối ưu và lưu lượng nước làm mát tương ứng.

\subsubsection{Phương pháp McCabe-Thiele}

Từ phương pháp McCabe-Thiele (xem Chương 2), với các thông số đầu vào:
\begin{itemize}
    \item Nhập liệu: 4.8 L/h, 10\% vol ethanol, 70°C (subcooled)
    \item Sản phẩm đỉnh: 90\% vol ethanol
    \item Sản phẩm đáy: 1\% vol ethanol (giả thiết)
    \item Số mâm thực tế: 6
    \item Hiệu suất mâm giả định: 80\% (giá trị điển hình cho mâm xuyên lỗ)
    \item Số mâm lý thuyết tương đương: $6 \times 0.80 = 4.8$
\end{itemize}

Kết quả tính toán:
\begin{itemize}
    \item Hệ số q (feed quality): $q = 1.042$ (nhập liệu subcooled)
    \item Tỷ số hồi lưu tối thiểu: $R_{min} = 1.68$
    \item Tỷ số hồi lưu vận hành: $R = 16.8$ ($10 \times R_{min}$)
\end{itemize}

Từ cân bằng vật chất:
\begin{align}
    D &= F \cdot \frac{x_F - x_W}{x_D - x_W} = 0.197 \text{ L/h} \\
    W &= F - D = 4.603 \text{ L/h} \\
    L &= R \cdot D = 3.31 \text{ L/h} \\
    V &= (R+1) \cdot D = 3.51 \text{ L/h}
\end{align}

\subsubsection{Cân bằng năng lượng tại reboiler}

Công suất reboiler bao gồm hai thành phần:
\begin{equation}
    Q_{reboiler} = Q_{feed} + Q_{vaporize}
\end{equation}

\textbf{1. Nhiệt gia nhiệt nhập liệu ($Q_{feed}$):}

Nhiệt cần thiết để nâng nhiệt độ nhập liệu từ 70°C đến điểm sôi (92.5°C):
\begin{equation}
    Q_{feed} = \dot{m}_F \cdot C_p \cdot \Delta T
\end{equation}
với:
\begin{itemize}
    \item $\dot{m}_F = 4.8$ L/h $\times$ 977 kg/m$^3$ / 3600 = 1.30 g/s
    \item $C_p = 4039$ J/(kg·K) (hỗn hợp 10\% ethanol)
    \item $\Delta T = 92.5 - 70 = 22.5$ °C
\end{itemize}

Kết quả: $Q_{feed} = 118$ W

\textbf{2. Nhiệt hóa hơi ($Q_{vaporize}$):}

Nhiệt cần thiết để tạo ra lưu lượng hơi V trong phần cất:
\begin{equation}
    Q_{vaporize} = \dot{n}_V \cdot \lambda_{mix}
\end{equation}
với:
\begin{itemize}
    \item $V = 3.51$ L/h $\rightarrow$ $\dot{n}_V = 73.6$ mol/h
    \item $\lambda_{mix} = 39.1$ kJ/mol (tại 90\% vol ethanol)
\end{itemize}

Kết quả: $Q_{vaporize} = 799$ W

\textbf{Tổng công suất reboiler lý thuyết:}
\begin{equation}
    Q_{reboiler,theory} = 118 + 799 = 917 \text{ W}
\end{equation}

\subsubsection{Công suất ngưng tụ tối ưu}

Để vận hành tối ưu (không subcooling), công suất ngưng tụ bằng nhiệt hóa hơi:
\begin{equation}
    Q_{condenser,optimal} = Q_{vaporize} = 799 \text{ W}
\end{equation}

\subsubsection{Tính toán lưu lượng nước làm mát}

Từ phương trình truyền nhiệt:
\begin{equation}
    Q = U \cdot A \cdot \Delta T_{LMTD}
\end{equation}

Với condenser coil (A = 0.28 m$^2$), hệ số truyền nhiệt U phụ thuộc vào lưu lượng:
\begin{equation}
    U = U_{ref} \cdot \left(\frac{F}{F_{ref}}\right)^{0.8}
\end{equation}

Giải iterative để tìm F sao cho Q đạt giá trị yêu cầu.

\textbf{Tổn hao công suất reboiler:}

Công suất điện đầu vào không hoàn toàn chuyển thành nhiệt hữu ích do:
\begin{itemize}
    \item Sụt áp trên dây dẫn và tiếp điểm
    \item Tổn hao nhiệt ra môi trường
\end{itemize}

Ước tính tổn hao: 10\%. Công suất nhiệt hiệu dụng:
\begin{equation}
    Q_{effective} = Q_{electrical} \times (1 - 0.10) = 4500 \times 0.90 = 4050 \text{ W}
\end{equation}

\textbf{Kết quả với các mức công suất khác nhau:}

\begin{table}[H]
\centering
\begin{tabular}{|l|c|c|c|}
\hline
\textbf{Chế độ} & \textbf{$Q_{condenser}$} & \textbf{Valve} & \textbf{Lưu lượng} \\
\hline
Lý thuyết tối ưu & 799 W & 7\% & 0.5 L/min \\
\hline
Vận hành thực tế (4050 W hiệu dụng) & 4050 W & 60\% & 4.3 L/min \\
\hline
Baseline (100\% valve) & 6189 W & 100\% & 7.2 L/min \\
\hline
\end{tabular}
\caption{Công suất ngưng tụ và lưu lượng nước làm mát theo các chế độ}
\label{tab:condenser_modes}
\end{table}

\textbf{Nhận xét quan trọng:} Hệ thống thiết bị phòng thí nghiệm thường được thiết kế với hệ số an toàn cao. Công suất reboiler điện (4500 W), sau khi trừ 10\% tổn hao, cho công suất hiệu dụng 4050 W. Với công suất này, lưu lượng nước làm mát tối ưu là khoảng 60\% (4.3 L/min), tiết kiệm 40\% so với mở van 100\%.

% ------------------------------------------------------------
\subsection{Kết quả mô phỏng}
% ------------------------------------------------------------

Mô phỏng được chạy trong 60 phút (3600 giây) với bước thời gian $\Delta t = 1$ giây.

\subsubsection{Mô phỏng vận hành baseline (van mở 100\%)}

Trong chế độ vận hành baseline, van nước làm mát được mở hoàn toàn (100\%) -- đây là cách vận hành truyền thống không có điều khiển tự động.

\begin{figure}[H]
\centering
\includegraphics[width=0.9\textwidth]{../simulation/outputs/fig_baseline_combined.png}
\caption{Mô phỏng vận hành baseline: Áp suất và lưu lượng sản phẩm khi van làm mát mở 100\%}
\label{fig:sim_baseline}
\end{figure}

Kết quả mô phỏng baseline cho thấy:
\begin{itemize}
    \item Áp suất đỉnh tháp ổn định tại ~1.0 bar (tháp hở khí quyển).
    \item Van nước làm mát mở 100\% -- gây lãng phí nước làm mát.
    \item Công suất ngưng tụ: 6180 W (vượt quá nhu cầu thực tế 4050 W hiệu dụng từ reboiler).
    \item Lưu lượng sản phẩm đỉnh: 1.53 L/h.
    \item Nhiệt độ đỉnh ổn định tại 78.2°C.
\end{itemize}

\subsubsection{Mô phỏng với điều khiển áp suất PID}

Bộ điều khiển PID được cài đặt với các thông số:
\begin{itemize}
    \item $K_c = 400$ -- Độ lợi bộ điều khiển
    \item $T_i = 30$ s -- Hằng số thời gian tích phân
    \item $T_d = 0$ s -- Hằng số thời gian vi phân (điều khiển PI)
    \item Setpoint: 1.0 bar
\end{itemize}

Mô phỏng bao gồm nhiễu loạn: tăng 10\% công suất reboiler tại $t = 600$ s (phút thứ 10).

\begin{figure}[H]
\centering
\includegraphics[width=0.9\textwidth]{../simulation/outputs/fig_pid_combined.png}
\caption{Mô phỏng điều khiển PID: Áp suất, độ mở van và lưu lượng sản phẩm}
\label{fig:sim_pressure_response}
\end{figure}

\begin{figure}[H]
\centering
\includegraphics[width=0.9\textwidth]{../simulation/outputs/fig_pid_valve.png}
\caption{Đáp ứng độ mở van làm mát theo thời gian với điều khiển PID}
\label{fig:sim_valve_position}
\end{figure}

Kết quả mô phỏng PID cho thấy:
\begin{itemize}
    \item Áp suất được duy trì ổn định tại 1.0 bar (giống baseline).
    \item Van làm mát mở trung bình 63\% (so với 100\% baseline) -- tiết kiệm 37\% nước.
    \item Công suất ngưng tụ: 4277 W (phù hợp với reboiler 4050 W hiệu dụng).
    \item Khi có nhiễu (+10\% công suất reboiler tại $t = 600$ s), van tự động tăng từ 63\% lên 68\%.
    \item Độ lệch áp suất tối đa khi có nhiễu: < 0.2 mbar.
\end{itemize}

% ------------------------------------------------------------
\subsection{So sánh hiệu quả vận hành}
% ------------------------------------------------------------

Bảng \ref{tab:sim_productivity_comparison} so sánh kết quả mô phỏng giữa hai phương án vận hành.

\begin{table}[H]
\centering
\begin{tabular}{|l|c|c|c|}
\hline
\textbf{Thông số} & \textbf{Baseline} & \textbf{PID Control} & \textbf{Đơn vị} \\
\hline
Áp suất xác lập & 1.00 & 1.00 & bar \\
\hline
Độ mở van xác lập & 100 & 63 & \% \\
\hline
Công suất ngưng tụ & 6179 & 4277 & W \\
\hline
Lưu lượng nước làm mát & 7.2 & 4.5 & L/min \\
\hline
Lưu lượng sản phẩm D & 0.37 & 0.37 & L/h \\
\hline
Nhiệt độ đỉnh tháp & 78.2 & 78.3 & °C \\
\hline
\end{tabular}
\caption{So sánh kết quả mô phỏng giữa baseline và PID control}
\label{tab:sim_productivity_comparison}
\end{table}

\begin{figure}[H]
\centering
\includegraphics[width=0.85\textwidth]{../simulation/outputs/fig_comparison_pressure.png}
\caption{So sánh áp suất đỉnh tháp giữa baseline và điều khiển PID}
\label{fig:comparison_pressure}
\end{figure}

\begin{figure}[H]
\centering
\includegraphics[width=0.85\textwidth]{../simulation/outputs/fig_comparison_valve.png}
\caption{So sánh độ mở van nước làm mát giữa baseline và điều khiển PID}
\label{fig:comparison_valve}
\end{figure}

\begin{figure}[H]
\centering
\includegraphics[width=0.85\textwidth]{../simulation/outputs/fig_comparison_condenser.png}
\caption{So sánh công suất thiết bị ngưng tụ giữa baseline và điều khiển PID}
\label{fig:comparison_condenser}
\end{figure}

\begin{figure}[H]
\centering
\includegraphics[width=0.85\textwidth]{../simulation/outputs/fig_comparison_product.png}
\caption{So sánh lưu lượng sản phẩm D giữa baseline và điều khiển PID}
\label{fig:comparison_product}
\end{figure}

\textbf{Các chỉ số cải thiện chính:}
\begin{itemize}
    \item Giảm độ mở van: từ 100\% xuống 63\% (giảm 37\%)
    \item Giảm lưu lượng nước làm mát: từ 7.2 L/min xuống 4.5 L/min (tiết kiệm 37\%)
    \item Công suất ngưng tụ phù hợp với reboiler: 4277 W $\approx$ 4050 W (hiệu dụng)
    \item Lưu lượng sản phẩm D không đổi: 0.37 L/h (cùng năng lượng reboiler)
\end{itemize}

% ------------------------------------------------------------
\subsection{Thảo luận}
% ------------------------------------------------------------

\subsubsection{Phân tích kết quả mô phỏng}

Kết quả mô phỏng cho thấy:
\begin{enumerate}
    \item \textbf{Cân bằng năng lượng}: Điều khiển PID tự động điều chỉnh độ mở van để công suất ngưng tụ (4277 W) phù hợp với công suất reboiler hiệu dụng (4050 W = 4500 W $\times$ 0.9). Điều này đảm bảo quá trình ngưng tụ hiệu quả mà không lãng phí nước làm mát.
    
    \item \textbf{Tiết kiệm nước làm mát}: Van làm mát giảm từ 100\% xuống 63\%, tiết kiệm 37\% lưu lượng nước làm mát. Đây là kết quả của việc sử dụng đúng lượng nước cần thiết thay vì mở van tối đa.
    
    \item \textbf{Năng suất sản phẩm không đổi}: Lưu lượng sản phẩm D duy trì ở mức 0.37 L/h cho cả hai phương án, vì năng suất được quyết định bởi công suất reboiler hiệu dụng (4050 W) chứ không phải công suất ngưng tụ.
    
    \item \textbf{Áp suất ổn định}: Do tháp hở với khí quyển, áp suất đỉnh luôn xấp xỉ 1.0 bar. Điều này là đặc điểm cố hữu của tháp chưng cất quy mô phòng thí nghiệm.
\end{enumerate}

\subsubsection{Giải thích cơ chế hoạt động}

\textbf{Nguyên lý vật lý quan trọng:} Hệ số truyền nhiệt $U$ phụ thuộc vào lưu lượng nước làm mát:
\begin{equation}
    Q = U \cdot A \cdot \Delta T_{LMTD}
\end{equation}
trong đó $U$ phụ thuộc vào số Reynolds ($Re$) và số Nusselt ($Nu$):
\begin{itemize}
    \item Lưu lượng cao $\rightarrow$ $Re$ cao $\rightarrow$ $Nu$ cao $\rightarrow$ $U$ cao
    \item Với dòng chảy rối: $U \propto (\text{lưu lượng})^{0.8}$
\end{itemize}

Khi van mở 100\%:
\begin{itemize}
    \item Lưu lượng nước: 7.2 L/min $\rightarrow$ $U = 800$ W/(m²·K)
    \item Công suất ngưng tụ: $Q_c = 6179$ W $>$ $Q_{eff} = 4050$ W
    \item Dư thừa ~2100 W công suất làm mát -- lãng phí nước.
\end{itemize}

Với điều khiển PID (van 63\%):
\begin{itemize}
    \item Lưu lượng nước: 4.5 L/min $\rightarrow$ $U = 510$ W/(m²·K)
    \item Công suất ngưng tụ: $Q_c = 4277$ W $\approx$ $Q_{eff} = 4050$ W
    \item Cân bằng năng lượng tối ưu -- tiết kiệm 37\% nước làm mát.
\end{itemize}

\subsubsection{Tính toán năng suất từ lưu lượng hồi lưu}

Lưu lượng sản phẩm đỉnh được tính gián tiếp qua lưu lượng hồi lưu (đo được qua lưu lượng kế):
\begin{equation}
    D = \frac{L}{R}
\end{equation}

Trong đó:
\begin{itemize}
    \item $D$: Lưu lượng sản phẩm đỉnh (L/h)
    \item $L$: Lưu lượng hồi lưu (L/h) -- đo trực tiếp
    \item $R$: Tỷ số hồi lưu -- xác định từ thiết kế
\end{itemize}

\subsubsection{Hạn chế của mô phỏng}

\begin{itemize}
    \item Mô hình đơn giản hóa, chưa xét đầy đủ các yếu tố phi tuyến.
    \item Chưa xem xét ảnh hưởng của nhiễu loạn (thay đổi nồng độ nhập liệu, nhiệt độ nước làm mát).
    \item Cần kiểm chứng thực nghiệm để xác nhận kết quả mô phỏng.
\end{itemize}
