% ============================================================
% Chương 5: Kết luận và kế hoạch Khóa luận Tốt nghiệp
% ============================================================

\newpage
\section{KẾT LUẬN VÀ KẾ HOẠCH KHÓA LUẬN TỐT NGHIỆP}

% ------------------------------------------------------------
\subsection{Kết luận}
% ------------------------------------------------------------

Báo cáo đã thực hiện các nội dung sau:

\begin{enumerate}
    \item \textbf{Phân tích hiện trạng vận hành:}
    \begin{itemize}
        \item Xác định nguyên nhân hạn chế năng suất khi vận hành với van làm mát mở 100\%.
        \item Phân tích mối quan hệ giữa công suất làm mát, tỷ số hồi lưu và năng suất sản phẩm.
        \item Phát hiện hệ thống đang sử dụng dư thừa ~3700 W công suất làm mát so với nhu cầu thực tế.
    \end{itemize}
    
    \item \textbf{Xây dựng cơ sở lý thuyết:}
    \begin{itemize}
        \item Trình bày phương pháp McCabe-Thiele xác định số mâm lý thuyết.
        \item Định nghĩa hiệu suất mâm Murphree và ứng dụng trong tính toán.
        \item Thiết lập các phương trình cân bằng năng lượng cho hệ thống chưng cất.
    \end{itemize}
    
    \item \textbf{Đề xuất giải pháp điều khiển:}
    \begin{itemize}
        \item Đề xuất vòng điều khiển áp suất đỉnh tháp PC-01 với bộ điều khiển PID.
        \item Xây dựng mô hình toán học của hệ thống: hàm truyền condenser, van, cảm biến.
    \end{itemize}
    
    \item \textbf{Tính toán:}
    \begin{itemize}
        \item Xây dựng chương trình tính toán cân bằng năng lượng và vật chất.
        \item Tính toán: $Q_{reboiler} = 917$ W (lý thuyết), $Q_{eff} = 5400$ W (thực tế).
        \item So sánh hai phương án vận hành: baseline (van 100\%) và điều khiển tối ưu.
    \end{itemize}
\end{enumerate}

\textbf{Kết quả tính toán chính:}

\begin{table}[H]
\centering
\begin{tabular}{|l|c|c|c|}
\hline
\textbf{Thông số} & \textbf{Baseline} & \textbf{Có điều khiển} & \textbf{Cải thiện} \\
\hline
Độ mở van nước làm mát & 100\% & 54\% & --46\% \\
\hline
Lưu lượng nước làm mát & 7.2 L/min & 3.9 L/min & --46\% \\
\hline
Công suất ngưng tụ & 9084 W & 5400 W & Cân bằng \\
\hline
Áp suất đỉnh tháp & 1.0 bar & 1.0 bar & Ổn định \\
\hline
\end{tabular}
\caption{Tổng hợp kết quả tính toán}
\label{tab:conclusion_results}
\end{table}

\textbf{Kết luận tổng quát:}
\begin{itemize}
    \item Phương pháp điều khiển áp suất đỉnh tháp thông qua van nước làm mát cho phép tiết kiệm 46\% lưu lượng nước làm mát.
    \item Van chỉ cần mở 54\% thay vì 100\% để đạt cân bằng năng lượng với reboiler 5400 W hiệu dụng.
    \item Giải pháp không đòi hỏi thay đổi cấu hình thiết bị, chỉ cần bổ sung hệ thống điều khiển tự động.
    \item Kết quả cần được kiểm chứng thực nghiệm để xác nhận hiệu quả trong điều kiện vận hành thực tế.
\end{itemize}

% ------------------------------------------------------------
\subsection{Kế hoạch Khóa luận Tốt nghiệp}
% ------------------------------------------------------------

Để kiểm chứng kết quả tính toán và triển khai thực tế, kế hoạch Khóa luận Tốt nghiệp bao gồm các nội dung sau:

\begin{enumerate}
    \item \textbf{Thiết kế và lập trình hệ thống điều khiển:}
    \begin{itemize}
        \item Thiết kế bộ điều khiển PID cho vòng điều khiển áp suất PC-01.
        \item Lập trình PLC S7-1200 trên TIA Portal.
        \item Thiết kế giao diện HMI để giám sát và điều khiển.
    \end{itemize}
    
    \item \textbf{Thí nghiệm thực tế:}
    \begin{itemize}
        \item Triển khai thử nghiệm trên hệ thống chưng cất tại phòng thí nghiệm.
        \item Thu thập dữ liệu vận hành: áp suất, lưu lượng, nhiệt độ.
        \item So sánh kết quả thực nghiệm với kết quả tính toán.
    \end{itemize}
    
    \item \textbf{Tinh chỉnh và đánh giá:}
    \begin{itemize}
        \item Tinh chỉnh thông số PID dựa trên đáp ứng thực tế.
        \item Xác định hệ số truyền nhiệt thực tế của thiết bị ngưng tụ.
        \item Đánh giá hiệu quả tiết kiệm nước làm mát và tăng năng suất.
    \end{itemize}
\end{enumerate}

% ------------------------------------------------------------
% Bảng kế hoạch thực hiện
% ------------------------------------------------------------

\begin{table}[H]
\centering
\begin{tabular}{|c|l|p{5.5cm}|c|}
\hline
\textbf{Tuần} & \textbf{Thời gian} & \textbf{Nội dung công việc} & \textbf{Ghi chú} \\
\hline
1--3 & 03/02 -- 23/02/2025 & Thiết kế bộ điều khiển PID, lập trình PLC S7-1200 (TIA Portal), thiết kế HMI & \\
\hline
4--6 & 24/02 -- 16/03/2025 & Thí nghiệm thực tế, thu thập dữ liệu vận hành & \\
\hline
7--8 & 17/03 -- 30/03/2025 & Tinh chỉnh thông số PID trên thiết bị thực & \\
\hline
9 & 31/03 -- 06/04/2025 & Hoàn thiện báo cáo, chuẩn bị bảo vệ đồ án & \\
\hline
\end{tabular}
\caption{Kế hoạch thực hiện Khóa luận Tốt nghiệp}
\label{tab:schedule}
\end{table}
