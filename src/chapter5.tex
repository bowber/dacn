% ============================================================
% Chương 5: Kết luận và hướng phát triển
% ============================================================

\newpage
\section{KẾT LUẬN VÀ KIẾN NGHỊ}

% ------------------------------------------------------------
\subsection{Kết luận}
% ------------------------------------------------------------

Đồ án đã thực hiện các nội dung sau:

\begin{enumerate}
    \item \textbf{Phân tích hiện trạng vận hành:}
    \begin{itemize}
        \item Xác định nguyên nhân hạn chế năng suất khi vận hành với van làm mát mở 100\%.
        \item Phân tích mối quan hệ giữa công suất làm mát, tỷ số hồi lưu và năng suất sản phẩm.
        \item Phát hiện hệ thống đang sử dụng dư thừa ~3700 W công suất làm mát so với nhu cầu thực tế.
    \end{itemize}
    
    \item \textbf{Xây dựng cơ sở lý thuyết:}
    \begin{itemize}
        \item Trình bày phương pháp McCabe-Thiele xác định số mâm lý thuyết.
        \item Định nghĩa hiệu suất mâm Murphree và ứng dụng trong tính toán.
        \item Thiết lập các phương trình cân bằng năng lượng cho hệ thống chưng cất.
    \end{itemize}
    
    \item \textbf{Thiết kế vòng điều khiển áp suất đỉnh tháp:}
    \begin{itemize}
        \item Xây dựng vòng điều khiển PC-01 với bộ điều khiển PID.
        \item Xây dựng mô hình toán học của hệ thống: hàm truyền condenser, van, cảm biến.
        \item Đề xuất phương pháp xác định thông số PID (Ziegler-Nichols).
    \end{itemize}
    
    \item \textbf{Tính toán và mô phỏng:}
    \begin{itemize}
        \item Xây dựng chương trình tính toán bằng Python.
        \item Tính toán cân bằng năng lượng: $Q_{reboiler} = 917$ W (lý thuyết), $Q_{eff} = 5400$ W (thực tế).
        \item So sánh hai phương án vận hành: baseline (van 100\%) và điều khiển tối ưu.
    \end{itemize}
\end{enumerate}

\textbf{Kết quả tính toán chính:}

\begin{table}[H]
\centering
\begin{tabular}{|l|c|c|c|}
\hline
\textbf{Thông số} & \textbf{Baseline} & \textbf{Có điều khiển} & \textbf{Cải thiện} \\
\hline
Độ mở van nước làm mát & 100\% & 54\% & --46\% \\
\hline
Lưu lượng nước làm mát & 7.2 L/min & 3.9 L/min & --46\% \\
\hline
Công suất ngưng tụ & 9084 W & 5400 W & Cân bằng \\
\hline
Áp suất đỉnh tháp & 1.0 bar & 1.0 bar & Ổn định \\
\hline
\end{tabular}
\caption{Tổng hợp kết quả tính toán}
\label{tab:conclusion_results}
\end{table}

\textbf{Kết luận tổng quát:}
\begin{itemize}
    \item Phương pháp điều khiển áp suất đỉnh tháp thông qua van nước làm mát cho phép tiết kiệm 46\% lưu lượng nước làm mát.
    \item Van chỉ cần mở 54\% thay vì 100\% để đạt cân bằng năng lượng với reboiler 5400 W hiệu dụng.
    \item Giải pháp không đòi hỏi thay đổi cấu hình thiết bị, chỉ cần bổ sung hệ thống điều khiển tự động.
    \item Kết quả cần được kiểm chứng thực nghiệm để xác nhận hiệu quả trong điều kiện vận hành thực tế.
\end{itemize}

% ------------------------------------------------------------
\subsection{Kiến nghị và hướng phát triển}
% ------------------------------------------------------------

Để nâng cao hiệu quả và mở rộng phạm vi ứng dụng, đề xuất các hướng phát triển sau:

\begin{enumerate}
    \item \textbf{Kiểm chứng thực nghiệm:}
    \begin{itemize}
        \item Triển khai thử nghiệm trên hệ thống chưng cất thực tế tại phòng thí nghiệm.
        \item So sánh kết quả thực nghiệm với kết quả mô phỏng.
        \item Tinh chỉnh thông số PID dựa trên đáp ứng thực tế.
        \item Xác định hệ số truyền nhiệt thực tế của thiết bị ngưng tụ.
    \end{itemize}
    
    \textbf{Phân tích độ nhạy hiệu suất mâm Murphree:} Tỷ số hồi lưu vận hành $R = 10.4$ được tính toán với giả định hiệu suất mâm Murphree $E_{MV} = 80\%$. Tuy nhiên, trong thực tế hiệu suất mâm có thể dao động trong khoảng 70--80\% tùy thuộc vào điều kiện vận hành (lưu lượng, nhiệt độ, tình trạng thiết bị). Bảng \ref{tab:sensitivity_EMV} cho thấy ảnh hưởng của $E_{MV}$ đến tỷ số hồi lưu cần thiết:
    
    \begin{table}[H]
    \centering
    \begin{tabular}{|c|c|c|c|}
    \hline
    \textbf{$E_{MV}$} & \textbf{$N_{\text{lý thuyết}}$} & \textbf{$R$ cần thiết} & \textbf{Ghi chú} \\
    \hline
    70\% & 4.2 & 25.9 & Điều kiện xấu \\
    \hline
    75\% & 4.5 & 16.4 & Trung bình \\
    \hline
    80\% & 4.8 & 10.4 & Điều kiện tốt \\
    \hline
    85\% & 5.1 & 7.7 & Điều kiện rất tốt \\
    \hline
    \end{tabular}
    \caption{Ảnh hưởng của hiệu suất mâm Murphree đến tỷ số hồi lưu}
    \label{tab:sensitivity_EMV}
    \end{table}
    
    Từ bảng trên, nếu hiệu suất mâm thực tế thấp hơn 80\%, cần tăng tỷ số hồi lưu để đảm bảo đạt nồng độ sản phẩm 90\% vol. Do đó, việc xác định $E_{MV}$ thực tế qua thí nghiệm là cần thiết để tối ưu hóa $R$.
    
    \textbf{Phân tích độ nhạy hệ số truyền nhiệt:} Kết quả tính toán phụ thuộc vào hệ số truyền nhiệt tổng $U$ của thiết bị ngưng tụ. Giá trị $U_{ref} = 800$ W/(m²·K) được sử dụng trong tính toán dựa trên dữ liệu từ Engineering ToolBox cho hệ thống ngưng tụ hơi hữu cơ với nước làm mát (300--1200 W/(m²·K)). Bảng \ref{tab:sensitivity_U} cho thấy kết quả thay đổi đáng kể theo giá trị $U_{ref}$:
    
    \begin{table}[H]
    \centering
    \begin{tabular}{|c|c|c|c|c|}
    \hline
    \textbf{$U_{ref}$ [W/(m²·K)]} & \textbf{$Q_{100\%}$ [W]} & \textbf{Van tối ưu} & \textbf{Tiết kiệm} & \textbf{Ghi chú} \\
    \hline
    300 & 3890 & 100\% & 0\% & Bám bẩn nặng \\
    \hline
    500 & 6144 & 86\% & 14\% & \\
    \hline
    800 & 9084 & 54\% & 46\% & Giả định \\
    \hline
    1000 & 10792 & 45\% & 55\% & \\
    \hline
    1200 & 12317 & 39\% & 61\% & Ống sạch \\
    \hline
    \end{tabular}
    \caption{Ảnh hưởng của hệ số truyền nhiệt $U_{ref}$ đến kết quả tính toán}
    \label{tab:sensitivity_U}
    \end{table}
    
    Từ bảng trên, nếu $U_{ref} < 400$ W/(m²·K), condenser không đủ công suất để ngưng tụ hết hơi từ reboiler ngay cả khi van mở 100\%. Do đó, việc xác định $U_{ref}$ thực tế qua thí nghiệm là cần thiết để đánh giá chính xác hiệu quả tiết kiệm nước làm mát.
    
    \item \textbf{Mở rộng quy mô áp dụng:}
    \begin{itemize}
        \item Triển khai thử nghiệm trên hệ thống chưng cất quy mô pilot hoặc công nghiệp.
        \item Đánh giá hiệu quả tăng năng suất ở các quy mô khác nhau.
        \item Nghiên cứu ảnh hưởng của các yếu tố vận hành thực tế (nhiễu loạn, thay đổi tải).
    \end{itemize}
    
    \item \textbf{Cải tiến thuật toán điều khiển:}
    \begin{itemize}
        \item Áp dụng điều khiển thích nghi (Adaptive PID) để tự động điều chỉnh thông số theo điều kiện vận hành.
        \item Nghiên cứu điều khiển dự báo mô hình (MPC - Model Predictive Control) để tối ưu đa mục tiêu.
        \item Tích hợp trí tuệ nhân tạo (AI) để dự đoán và tối ưu hóa vận hành.
    \end{itemize}
    
    \item \textbf{Tối ưu hóa toàn diện hệ thống:}
    \begin{itemize}
        \item Kết hợp điều khiển áp suất với tối ưu tỷ số hồi lưu.
        \item Nghiên cứu tích hợp nhiệt (heat integration) giữa các dòng trong hệ thống.
        \item Phát triển hệ thống quản lý năng lượng và năng suất tổng thể cho nhà máy.
    \end{itemize}
    
    \item \textbf{Phát triển hệ thống giám sát và báo cáo:}
    \begin{itemize}
        \item Xây dựng dashboard giám sát năng suất theo thời gian thực.
        \item Tích hợp hệ thống IoT để thu thập và phân tích dữ liệu vận hành.
        \item Phát triển báo cáo tự động về hiệu quả năng suất.
    \end{itemize}
\end{enumerate}

% ------------------------------------------------------------
% Bảng kế hoạch thực hiện
% ------------------------------------------------------------

\begin{table}[H]
\centering
\begin{tabular}{|c|l|p{5.5cm}|c|}
\hline
\textbf{Tuần} & \textbf{Thời gian} & \textbf{Nội dung công việc} & \textbf{Ghi chú} \\
\hline
1 & 23/12 -- 29/12/2024 & Nghiên cứu tài liệu về điều khiển quá trình, PID, tối ưu năng suất chưng cất & Xong \\
\hline
2 & 30/12/2024 -- 05/01/2025 & Khảo sát hệ thống thực tế, đo đạc và thu thập thông số thiết bị & Xong \\
\hline
3 & 06/01 -- 12/01/2025 & Xây dựng mô hình toán học: hàm truyền condenser, van, cảm biến & Xong \\
\hline
4 & 13/01 -- 19/01/2025 & Lập trình Python: mô hình hóa quá trình, cân bằng năng lượng & Xong \\
\hline
5 & 20/01 -- 26/01/2025 & Lập trình Python: bộ điều khiển PID, vòng điều khiển kín & Xong \\
\hline
6 & 27/01 -- 02/02/2025 & Mô phỏng so sánh: baseline (van 100\%) và điều khiển PID & Xong \\
\hline
7 & 03/02 -- 09/02/2025 & Tối ưu thông số PID (Ziegler-Nichols, ISE/IAE), xuất đồ thị kết quả & Xong \\
\hline
8 & 10/02 -- 16/02/2025 & Viết báo cáo: Chương 1--3 (Mở đầu, Lý thuyết, Mô hình) & Xong \\
\hline
9 & 17/02 -- 23/02/2025 & Viết báo cáo: Chương 4--5 (Kết quả, Kết luận), hoàn thiện tài liệu & $\leftarrow$ Hiện tại \\
\hline
10 & 24/02 -- 02/03/2025 & Lập trình PLC S7-1200 (TIA Portal), thiết kế giao diện HMI & \\
\hline
11 & 03/03 -- 09/03/2025 & Thí nghiệm baseline: vận hành van 100\%, thu thập dữ liệu & \\
\hline
12 & 10/03 -- 16/03/2025 & Thí nghiệm điều khiển PID: áp dụng thông số từ mô phỏng & \\
\hline
13 & 17/03 -- 23/03/2025 & Tinh chỉnh thông số PID trên thiết bị thực, thí nghiệm bổ sung & \\
\hline
14 & 24/03 -- 30/03/2025 & Xử lý số liệu, so sánh kết quả mô phỏng và thực nghiệm & \\
\hline
15 & 31/03 -- 06/04/2025 & Chuẩn bị slide, poster và bảo vệ đồ án & \\
\hline
\end{tabular}
\caption{Kế hoạch thực hiện đồ án}
\label{tab:schedule}
\end{table}
