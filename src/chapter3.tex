% ============================================================
% Chương 3: Mô hình thí nghiệm
% ============================================================

\newpage
\section{MÔ HÌNH THÍ NGHIỆM}

% ------------------------------------------------------------
\subsection{Hệ thống chưng cất}
% ------------------------------------------------------------

\subsubsection{Quy trình công nghệ}

Hệ thống chưng cất ethanol -- nước quy mô phòng thí nghiệm tại phòng B2-105, Khoa Kỹ thuật Hóa học, Đại học Bách Khoa TP.HCM được thiết kế với các thành phần chính như sau:

\begin{figure}[H]
\centering
\includegraphics[width=0.9\textwidth]{../assets/PFD.jpg}
\caption{Sơ đồ P\&ID hệ thống chưng cất ethanol -- nước}
\label{fig:pid_diagram}
\end{figure}

Quy trình hoạt động:
\begin{enumerate}
    \item Hỗn hợp ethanol -- nước (nồng độ 10\% vol) được bơm qua bộ gia nhiệt và đưa vào tháp chưng cất tại mâm thứ 4 (từ dưới lên).
    \item Reboiler (sử dụng 2 điện trở gia nhiệt, tổng công suất 6 kW) cung cấp nhiệt để hóa hơi hỗn hợp đáy tháp.
    \item Hơi đi lên qua 6 mâm xuyên lỗ, tiếp xúc với dòng lỏng hồi lưu đi xuống.
    \item Hơi giàu ethanol thoát ra ở đỉnh tháp được ngưng tụ tại bộ ngưng tụ dạng ống xoắn ruột gà.
    \item Một phần sản phẩm ngưng tụ hồi lưu về tháp, phần còn lại là sản phẩm đỉnh (nồng độ mục tiêu 90\% vol).
    \item Sản phẩm đáy (nghèo ethanol) được lấy ra từ reboiler.
\end{enumerate}

\subsubsection{Thông số công nghệ của hệ thống chưng cất}

\begin{table}[htbp]
\centering
\begin{tabular}{|l|c|c|}
\hline
\textbf{Thông số} & \textbf{Giá trị} & \textbf{Đơn vị} \\
\hline
\multicolumn{3}{|l|}{\textbf{Tháp chưng cất}} \\
\hline
Đường kính tháp & 150 & mm \\
\hline
Chiều cao tháp & 800 & mm \\
\hline
Số mâm & 6 & mâm \\
\hline
Loại mâm & \multicolumn{2}{c|}{Mâm xuyên lỗ, 15 lỗ, có ống chảy truyền} \\
\hline
Khoảng cách mâm & 140 & mm \\
\hline
Vật liệu mâm & \multicolumn{2}{c|}{SS304} \\
\hline
Vị trí nhập liệu & \multicolumn{2}{c|}{Mâm thứ 4 (từ dưới lên)} \\
\hline
\multicolumn{3}{|l|}{\textbf{Reboiler}} \\
\hline
Loại reboiler & \multicolumn{2}{c|}{Điện trở gia nhiệt} \\
\hline
Công suất reboiler tối đa & 6000 (2$\times$3000) & W \\
\hline
\multicolumn{3}{|l|}{\textbf{Condenser}} \\
\hline
Loại condenser & \multicolumn{2}{c|}{Ống xoắn ruột gà} \\
\hline
Đường kính ống & 13 & mm \\
\hline
Số vòng xoắn & 22 & vòng \\
\hline
Đường kính xoắn & 100 & mm \\
\hline
Vật liệu & \multicolumn{2}{c|}{SS304} \\
\hline
Diện tích trao đổi nhiệt & 0.28 & m$^2$ \\
\hline
Lưu lượng nước làm mát & 7.2 & L/min \\
\hline
\multicolumn{3}{|l|}{\textbf{Bình hồi lưu}} \\
\hline
Kích thước & $\phi$20 $\times$ 245 & mm \\
\hline
Chỉ thị mức & \multicolumn{2}{c|}{Có} \\
\hline
\multicolumn{3}{|l|}{\textbf{Hệ thống nhập liệu}} \\
\hline
Bộ gia nhiệt nhập liệu & \multicolumn{2}{c|}{Có} \\
\hline
Công suất điện trở gia nhiệt & 300 & W \\
\hline
Lưu lượng nhập liệu & $\sim$4.8 (80 ml/min) & L/h \\
\hline
\multicolumn{3}{|l|}{\textbf{Van điều khiển nước làm mát}} \\
\hline
Loại van & \multicolumn{2}{c|}{Bürkert Type 2873 -- Van solenoid tỷ lệ} \\
\hline
Bộ điều khiển & \multicolumn{2}{c|}{Bürkert Type 8605 -- PWM controller} \\
\hline
Điện áp hoạt động & 24 & V DC \\
\hline
Dòng điện cuộn coil tối đa & 420 & mA \\
\hline
Tín hiệu điều khiển & 4-20 & mA \\
\hline
\multicolumn{3}{|l|}{\textbf{Điều kiện vận hành}} \\
\hline
Nồng độ nhập liệu & 10 & \% vol \\
\hline
Nồng độ sản phẩm đỉnh & 90 & \% vol \\
\hline
Áp suất vận hành & 1.0 & bar \\
\hline
\end{tabular}
\caption{Thông số thiết bị chính của hệ thống chưng cất}
\label{tab:equipment_specs}
\end{table}

\begin{figure}[H]
\centering
\includegraphics[width=0.5\textwidth]{../assets/reboiler_heaters.jpg}
\caption{Điện trở gia nhiệt reboiler (2 $\times$ 3000 W)}
\label{fig:reboiler_transistors}
\end{figure}

\textbf{Tính toán diện tích trao đổi nhiệt của condenser:}

Diện tích trao đổi nhiệt của condenser dạng ống xoắn ruột gà được tính theo công thức:
\begin{equation}
A = \pi \cdot d \cdot L = \pi \cdot d \cdot (n \cdot \pi \cdot D)
\end{equation}

Trong đó:
\begin{itemize}
    \item $d = 13$ mm -- đường kính ống
    \item $n = 22$ vòng -- số vòng xoắn
    \item $D = 100$ mm -- đường kính xoắn
\end{itemize}

Thay số:
\begin{equation}
A = \pi \times 0.013 \times (22 \times \pi \times 0.1) = \pi \times 0.013 \times 6.912 \approx 0.28 \text{ m}^2
\end{equation}

% ------------------------------------------------------------
\subsection{Thiết bị đo lường và điều khiển}
% ------------------------------------------------------------

\subsubsection{Thiết bị đo}

Hệ thống được trang bị các thiết bị đo sau:

\begin{table}[H]
\centering
\small
\begin{tabularx}{\textwidth}{|X|l|c|c|}
\hline
\textbf{Thiết bị} & \textbf{Loại/Model} & \textbf{Dải đo} & \textbf{Tín hiệu} \\
\hline
Cảm biến nhiệt độ đỉnh tháp & Pt100 + Transmitter & 0--200$^\circ$C & 4-20 mA \\
\hline
Cảm biến nhiệt độ nhập liệu & Pt100 + Transmitter & 0--200$^\circ$C & 4-20 mA \\
\hline
Cảm biến áp suất đỉnh tháp & SITRANS P200 & 1--60 bar & 4-20 mA \\
\hline
Cảm biến lưu lượng hồi lưu & DPM-1507-G1-L443 & 50--700 ml/min & 4-20 mA \\
\hline
Cảm biến mức bình hồi lưu & NMC-H1.2G603 & 0--1200 mm & 4-20 mA \\
\hline
\end{tabularx}
\caption{Danh sách thiết bị đo}
\label{tab:sensors}
\end{table}

\textbf{Cảm biến áp suất đỉnh tháp -- SITRANS P200:}

Cảm biến áp suất SITRANS P200 \cite{siemens_p200} được lắp đặt tại đường ống hơi ra khỏi đỉnh tháp, trước bộ ngưng tụ. Đây là cảm biến áp suất compact với các đặc điểm kỹ thuật:
\begin{itemize}
    \item Nguyên lý đo: Cảm biến áp điện trở (piezoresistive) với màng gốm ceramic
    \item Dải đo: 1--60 bar (áp suất tương đối và tuyệt đối)
    \item Tín hiệu ra: 4-20 mA hoặc 0-10 V DC
    \item Độ chính xác: $\pm$0.25\% của dải đo
    \item Vỏ bọc: Thép không gỉ, IP65/IP67
    \item Kết nối điện: Đầu nối M12 hoặc EN 175301-803-A
\end{itemize}

\textbf{Cảm biến lưu lượng hồi lưu -- Kobold DPM-1507-G1-L443:}

Cảm biến lưu lượng cánh quay thể tích nhỏ Kobold DPM \cite{kobold_dpm} được sử dụng để đo lưu lượng dòng hồi lưu:
\begin{itemize}
    \item Nguyên lý đo: Cánh quay (Rotating Vane) với cảm biến quang học
    \item Dải đo: 50--700 ml/min (nước)
    \item Vật liệu: Thép không gỉ 1.4404
    \item Áp suất làm việc tối đa: 16 bar
    \item Nhiệt độ làm việc: -40 đến +80$^\circ$C
    \item Tín hiệu ra: 4-20 mA (mã L443 -- đầu nối DIN 43650)
    \item Kết nối: G$\frac{1}{8}$ ren trong (mã G1)
    \item Độ chính xác: $\pm$1\% của thang đo
\end{itemize}

\begin{figure}[H]
\centering
\includegraphics[width=0.35\textwidth]{../assets/dpm.jpg}
\caption{Cảm biến lưu lượng Kobold DPM}
\label{fig:dpm_flowmeter}
\end{figure}

\textbf{Cảm biến mức bình hồi lưu -- Kobold NMC-H1.2G603:}

Cảm biến mức điện dung Kobold NMC (KOBOLD Messring GmbH, 2021) được lắp đặt tại bình chứa sản phẩm hồi lưu:
\begin{itemize}
    \item Nguyên lý đo: Điện dung (Capacitive)
    \item Chiều dài que đo: 1200 mm (mã H1.2)
    \item Dải áp suất: -1 đến +30 bar
    \item Nhiệt độ môi trường đo: -20 đến +200$^\circ$C (Model NMC-H)
    \item Sai số đo: < 1.5\% chiều dài que đo
    \item Tín hiệu ra: 4-20 mA
    \item Kết nối cơ khí: G1 ren ngoài, thép không gỉ (mã G6)
    \item Bảo vệ: IP65, phù hợp ứng dụng ATEX
\end{itemize}

\begin{figure}[H]
\centering
\includegraphics[width=0.35\textwidth]{../assets/level_meter.jpg}
\caption{Cảm biến mức điện dung Kobold NMC}
\label{fig:level_meter}
\end{figure}

\textbf{Cảm biến nhiệt độ -- Pt100 với Transmitter 4-20mA:}

Cảm biến nhiệt độ RTD Pt100 theo tiêu chuẩn IEC 60751 được sử dụng kết hợp với bộ chuyển đổi tín hiệu 4-20mA:
\begin{itemize}
    \item Loại cảm biến: Pt100 (100$\Omega$ tại 0$^\circ$C, $\alpha$ = 0.00385)
    \item Dải đo: 0--200$^\circ$C (có thể cấu hình)
    \item Cấp chính xác: Class A theo IEC 60751
    \item Tín hiệu ra: 4-20 mA (2-wire loop powered)
    \item Kết nối: 3-wire hoặc 4-wire RTD
    \item Nguồn cấp: 10-30 VDC
\end{itemize}

\begin{figure}[H]
\centering
\includegraphics[width=0.3\textwidth]{../assets/sitrans.jpg}
\caption{Cảm biến áp suất đỉnh tháp SITRANS P200}
\label{fig:pressure_sensor}
\end{figure}

\subsubsection{Van điều khiển nước làm mát}

Van điều khiển nước làm mát sử dụng van solenoid tỷ lệ Bürkert Type 2873 kết hợp với bộ điều khiển PWM Type 8605 để điều chỉnh lưu lượng nước làm mát vào bộ ngưng tụ.

\textbf{Van solenoid tỷ lệ Bürkert Type 2873:}

Van solenoid điều khiển tỷ lệ Type 2873 là van tác động trực tiếp 2/2, thường đóng (NC), được sử dụng làm cơ cấu chấp hành trong các vòng điều khiển. Với gioăng đàn hồi, van đóng kín hoàn toàn ở áp suất danh định. Piston của van được lắp đặt không ma sát, cho phép đặc tính điều chỉnh chính xác.

\begin{itemize}
    \item Loại van: Van solenoid tỷ lệ tác động trực tiếp, 2/2, thường đóng (NC)
    \item Điện áp hoạt động: 24 V DC
    \item Công suất tiêu thụ tối đa: 9 W
    \item Dòng điện cuộn coil tối đa: 420 mA (tại 9 W và 24 V)
    \item Tần số PWM: 1200 Hz
    \item Kích thước lỗ (orifice): 0.8 -- 6 mm (tùy model)
    \item Dải áp suất: 0 -- 16 bar (tùy kích thước lỗ)
    \item Vật liệu thân van: Đồng thau hoặc thép không gỉ
    \item Vật liệu gioăng: FKM, EPDM
    \item Nhiệt độ môi chất: -10$^\circ$C đến +90$^\circ$C (FKM)
    \item Cấp bảo vệ: IP65
\end{itemize}

Đặc tính điều khiển của van Type 2873:
\begin{itemize}
    \item Độ trễ (Hysteresis): < 5\%
    \item Độ lặp lại: < 0.5\% giá trị cuối
    \item Độ nhạy đáp ứng: < 0.25\% giá trị cuối
    \item Dải điều chỉnh: 1:200
    \item Thời gian tác động (10--90\%): < 20 ms
\end{itemize}

\begin{figure}[H]
\centering
\includegraphics[width=0.3\textwidth]{../assets/burkert_2873.jpg}
\caption{Van solenoid tỷ lệ Bürkert Type 2873}
\label{fig:burkert_2873}
\end{figure}

\textbf{Bộ điều khiển PWM Bürkert Type 8605:}

Bộ điều khiển điện tử Type 8605 được sử dụng để vận hành van solenoid tỷ lệ trong dải công suất 40 -- 2000 mA. Thiết bị chuyển đổi tín hiệu chuẩn đầu vào (4-20 mA hoặc 0-10 V) thành tín hiệu PWM (Pulse Width Modulation) để điều khiển vô cấp độ mở van.

\begin{itemize}
    \item Tín hiệu đầu vào: 0-20 mA, 4-20 mA hoặc 0-5 V, 0-10 V (có thể cấu hình)
    \item Tín hiệu đầu ra: PWM, tần số điều chỉnh được 80 Hz -- 6 kHz
    \item Dải dòng điện đầu ra: 40 -- 2000 mA (cấu hình được, Type 2873 cần tối đa 420 mA)
    \item Điện áp hoạt động: 12 -- 24 V DC
    \item Công suất tiêu thụ: $\sim$1 W (không tính van)
    \item Chức năng Ramp: Điều chỉnh được 0 -- 10 giây
    \item Màn hình hiển thị và phím bấm để cài đặt thông số
    \item Cấp bảo vệ: IP65 (dạng cable plug)
\end{itemize}

\begin{figure}[H]
\centering
\includegraphics[width=0.35\textwidth]{../assets/burkert_8605.jpg}
\caption{Bộ điều khiển PWM Bürkert Type 8605}
\label{fig:burkert_8605}
\end{figure}

Khi kết hợp van Type 2873 với bộ điều khiển Type 8605, tín hiệu điều khiển 4-20 mA từ PLC được chuyển đổi thành dòng điện cuộn coil tương ứng để điều chỉnh độ mở van. Type 8605 được cấu hình dải đầu ra phù hợp với Type 2873 (tối đa 420 mA). Tại tín hiệu 20 mA (100\% độ mở), dòng điện cuộn coil đạt khoảng 390 mA.

\subsubsection{Bộ điều khiển PLC S7-1200}

Hệ thống sử dụng PLC Siemens S7-1200 CPU 1214C làm bộ điều khiển trung tâm, với các đặc điểm:

\begin{itemize}
    \item CPU 1214C DC/DC/DC
    \item 14 DI / 10 DO tích hợp
    \item 2 AI tích hợp (0-10V)
    \item 2 module mở rộng SM 1231 AI (4 AI mỗi module)
    \item 1 module mở rộng SM 1232 AQ (2 AO)
    \item Giao tiếp PROFINET
\end{itemize}

\begin{figure}[H]
\centering
\includegraphics[width=0.5\textwidth]{../assets/plc.jpg}
\caption{PLC Siemens S7-1200 và các module mở rộng}
\label{fig:plc}
\end{figure}

\textbf{Cấu hình I/O:}

\begin{table}[H]
\centering
\begin{tabular}{|l|l|}
\hline
\textbf{Địa chỉ} & \textbf{Tên tín hiệu} \\
\hline
\%I0.0 & Công tắc chế độ tự động \\
\%IW96 & Mức bình chứa sản phẩm đỉnh \\
\%IW98 & Áp suất đỉnh tháp \\
\%IW100 & Lưu lượng hồi lưu \\
\%IW102 & Lưu lượng nhập liệu \\
\%IW104 & Nhiệt độ đỉnh tháp \\
\%IW106 & Nhiệt độ mâm 1 \\
\%IW108 & Nhiệt độ mâm 2 \\
\%IW110 & Nhiệt độ mâm 3 \\
\%IW112 & Nhiệt độ mâm 4 \\
\%IW114 & Nhiệt độ mâm 5 \\
\%IW116 & Nhiệt độ mâm 6 \\
\%IW118 & Nhiệt độ nhập liệu \\
\%IW120 & Nhiệt độ sản phẩm đỉnh \\
\%IW122 & Nhiệt độ nồi đun \\
\%IW124 & Nhiệt độ nước làm mát vào \\
\%IW126 & Nhiệt độ nước làm mát ra \\
\hline
\end{tabular}
\caption{Bảng tín hiệu đầu vào (Input)}
\label{tab:io_input}
\end{table}

\begin{table}[H]
\centering
\begin{tabular}{|l|l|}
\hline
\textbf{Địa chỉ} & \textbf{Tên tín hiệu} \\
\hline
\%Q0.0 & Relay van sản phẩm đỉnh \\
\%Q0.1 & Relay bơm nhập liệu \\
\%Q0.2 & Relay bơm hồi lưu \\
\%Q0.3 & Relay điện trở nồi đun 1 \\
\%Q0.4 & Relay điện trở bồn chứa \\
\%Q0.6 & Relay van nhập liệu \\
\%Q0.7 & Relay điện trở nồi đun 2 \\
\%Q1.0 & Van nước làm mát (On/Off) \\
\%Q1.1 & Van hồi lưu (On/Off) \\
\%QW128 & Tín hiệu điều khiển tốc độ bơm nhập liệu \\
\%QW130 & Tín hiệu điều khiển điện trở bồn chứa \\
\%QW132 & Độ mở van hồi lưu \\
\%QW134 & Độ mở van nước làm mát \\
\%QW290 & Độ mở van nồi đun \\
\%QW752 & Tín hiệu điều khiển tốc độ bơm hồi lưu \\
\hline
\end{tabular}
\caption{Bảng tín hiệu đầu ra (Output)}
\label{tab:io_output}
\end{table}

% ------------------------------------------------------------
\subsection{Thiết kế các vòng điều khiển}
% ------------------------------------------------------------

Hệ thống chưng cất được thiết kế với các vòng điều khiển sau:

\begin{enumerate}
    \item \textbf{Vòng điều khiển lưu lượng nhập liệu (FC-01):} Điều khiển tốc độ bơm để duy trì lưu lượng nhập liệu theo giá trị đặt.
    \item \textbf{Vòng điều khiển nhiệt độ nhập liệu (TC-01):} Điều khiển bộ gia nhiệt nhập liệu để đạt nhiệt độ mong muốn.
    \item \textbf{Vòng điều khiển nhiệt độ đỉnh tháp (TC-02):} Điều khiển công suất reboiler để duy trì nhiệt độ đỉnh tháp.
    \item \textbf{Vòng điều khiển áp suất đỉnh tháp (PC-01):} Điều khiển độ mở van nước làm mát để ổn định áp suất đỉnh tháp -- \textbf{đây là vòng điều khiển mới được thiết kế trong đồ án này}.
\end{enumerate}

% ------------------------------------------------------------
\subsection{Chiến lược điều khiển}
% ------------------------------------------------------------

\subsubsection{Phương án vận hành hiện tại (Baseline)}

Trong phương án vận hành hiện tại, van nước làm mát được mở cố định ở mức 100\% để đảm bảo áp suất đỉnh tháp luôn ổn định, đồng thời reboiler hoạt động ở công suất tối đa (ước tính 6000 W). Phương án này có ưu điểm là đơn giản, không cần hệ thống điều khiển phức tạp, nhưng gây lãng phí năng lượng đáng kể.

\begin{figure}[H]
\centering
\begin{tikzpicture}[node distance=2cm]
    % Process
    \node[block, minimum width=4cm] (condenser) {Bộ ngưng tụ};
    \node[left=2cm of condenser] (vapor) {Hơi đỉnh tháp};
    \node[right=2cm of condenser] (liquid) {Lỏng ngưng tụ};
    
    % Cooling water - fixed
    \node[above=2.5cm of condenser] (cw_in) {Nước làm mát (100\%)};
    \node[block, above=0.5cm of condenser, minimum width=2cm, fill=gray!30] (valve) {Van mở 100\%};
    
    % Arrows
    \draw[arrow] (vapor) -- (condenser);
    \draw[arrow] (condenser) -- (liquid);
    \draw[-Stealth, thick] (cw_in.south) -- (valve.north);
    \draw[line] (valve) -- (condenser);
    
\end{tikzpicture}
\caption{Sơ đồ vận hành hiện tại -- Van làm mát mở cố định 100\%}
\label{fig:baseline_control}
\end{figure}

\subsubsection{Phương án điều khiển tối ưu năng lượng (Đề xuất)}

Phương án đề xuất sử dụng bộ điều khiển PID để điều chỉnh độ mở van nước làm mát dựa trên áp suất đỉnh tháp. Van chỉ mở đủ để duy trì áp suất ở giá trị đặt, tiết kiệm nước làm mát không cần thiết.

\begin{figure}[H]
\centering
\begin{tikzpicture}[node distance=1.8cm]
    % Setpoint
    \node (sp) {SP$_P$};
    
    % Sum
    \node[sum, right=1cm of sp] (sum) {};
    
    % Controller
    \node[block, right=1.2cm of sum] (controller) {PID};
    
    % Valve
    \node[block, right=1.2cm of controller] (valve) {Van làm mát};
    
    % Condenser
    \node[block, right=1.2cm of valve, minimum width=2.5cm] (condenser) {Bộ ngưng tụ};
    
    % Pressure output
    \node[right=1cm of condenser] (output) {$P$};
    
    % Sensor
    \node[block, below=1cm of controller] (sensor) {Cảm biến áp suất};
    
    % Arrows
    \draw[arrow] (sp) -- node[above, font=\normalsize] {$+$} (sum);
    \draw[arrow] (sum) -- node[above, font=\normalsize] {$e$} (controller);
    \draw[arrow] (controller) -- node[above, font=\normalsize] {$u$} (valve);
    \draw[arrow] (valve) -- node[above, font=\normalsize] {$Q_C$} (condenser);
    \draw[arrow] (condenser) -- (output);
    
    % Feedback
    \coordinate (fb) at ($(condenser.east)!0.5!(output.west)$);
    \draw[line] (fb) |- (sensor);
    \draw[arrow] (sensor) -| node[pos=0.95, left, font=\normalsize] {$-$} (sum);
    
\end{tikzpicture}
\caption{Sơ đồ khối vòng điều khiển áp suất đỉnh tháp (PC-01)}
\label{fig:pressure_control_loop}
\end{figure}

\textbf{Nguyên lý hoạt động:}
\begin{itemize}
    \item Khi áp suất đỉnh tháp tăng cao hơn giá trị đặt: Bộ điều khiển tăng độ mở van làm mát $\rightarrow$ tăng công suất ngưng tụ $\rightarrow$ áp suất giảm.
    \item Khi áp suất đỉnh tháp giảm thấp hơn giá trị đặt: Bộ điều khiển giảm độ mở van làm mát $\rightarrow$ giảm công suất ngưng tụ $\rightarrow$ áp suất tăng.
\end{itemize}

\subsubsection{Điều khiển công suất reboiler bằng PWM}

Song song với vòng điều khiển áp suất, phương án điều khiển PWM công suất reboiler được đề xuất để tiết kiệm điện năng. Thay vì chạy reboiler ở 100\% công suất liên tục, duty cycle được điều chỉnh theo nhu cầu thực tế:

\begin{figure}[H]
\centering
\begin{tikzpicture}[node distance=1.8cm]
    % Setpoint
    \node (sp) {SP$_T$};
    
    % Sum
    \node[sum, right=1cm of sp] (sum) {};
    
    % Controller
    \node[block, right=1.2cm of sum] (controller) {PID};
    
    % Reboiler
    \node[block, right=1.2cm of controller] (reboiler) {Reboiler (PWM)};
    
    % Column
    \node[block, right=1.2cm of reboiler, minimum width=2cm] (column) {Tháp chưng cất};
    
    % Temperature output
    \node[right=1cm of column] (output) {$T$};
    
    % Sensor
    \node[block, below=1cm of controller] (sensor) {Cảm biến nhiệt độ};
    
    % Arrows
    \draw[arrow] (sp) -- node[above, font=\normalsize] {$+$} (sum);
    \draw[arrow] (sum) -- node[above, font=\normalsize] {$e$} (controller);
    \draw[arrow] (controller) -- node[above, font=\normalsize] {$u$} (reboiler);
    \draw[arrow] (reboiler) -- node[above, font=\normalsize] {$Q_R$} (column);
    \draw[arrow] (column) -- (output);
    
    % Feedback
    \coordinate (fb) at ($(column.east)!0.5!(output.west)$);
    \draw[line] (fb) |- (sensor);
    \draw[arrow] (sensor) -| node[pos=0.95, left, font=\normalsize] {$-$} (sum);
    
\end{tikzpicture}
\caption{Sơ đồ khối vòng điều khiển công suất reboiler bằng PWM}
\label{fig:temp_control_loop}
\end{figure}

\textbf{Nguyên lý hoạt động:}
\begin{itemize}
    \item Bộ điều khiển PID điều chỉnh duty cycle của tín hiệu PWM dựa trên nhiệt độ đỉnh tháp.
    \item Khi nhiệt độ thấp hơn giá trị đặt: Tăng duty cycle $\rightarrow$ tăng công suất reboiler.
    \item Khi nhiệt độ cao hơn giá trị đặt: Giảm duty cycle $\rightarrow$ giảm công suất reboiler.
    \item Công suất trung bình: $P_{avg} = P_{max} \times \text{duty cycle}$
\end{itemize}

\textbf{Ưu điểm của điều khiển PWM reboiler:}
\begin{itemize}
    \item Tiết kiệm điện năng khi nhu cầu nhiệt thấp hơn công suất tối đa.
    \item Duy trì chất lượng sản phẩm ổn định thông qua điều khiển nhiệt độ.
    \item Tương thích với hệ thống PLC hiện có (sử dụng relay đóng ngắt điện trở).
\end{itemize}
