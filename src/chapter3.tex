% ============================================================
% Chương 3: Mô hình thí nghiệm
% ============================================================

\newpage
\section{MÔ HÌNH THÍ NGHIỆM}

% ------------------------------------------------------------
\subsection{Hệ thống chưng cất}
% ------------------------------------------------------------

\subsubsection{Quy trình công nghệ}

Hệ thống chưng cất ethanol -- nước quy mô phòng thí nghiệm tại phòng B2-105, Khoa Kỹ thuật Hóa học, Đại học Bách Khoa TP.HCM được thiết kế với các thành phần chính như sau:

% Placeholder cho sơ đồ công nghệ
\begin{figure}[H]
\centering
\fbox{\parbox{0.6\textwidth}{\centering\vspace{3cm}Sơ đồ công nghệ hệ thống chưng cất ethanol -- nước\\(Sẽ cập nhật sau)\vspace{3cm}}}
\caption{Sơ đồ công nghệ hệ thống chưng cất ethanol -- nước}
\label{fig:process_flow}
\end{figure}

Quy trình hoạt động:
\begin{enumerate}
    \item Hỗn hợp ethanol -- nước (nồng độ 10\% vol) được bơm qua bộ gia nhiệt và đưa vào tháp chưng cất tại mâm thứ 4 (từ dưới lên).
    \item Reboiler (sử dụng 2 điện trở gia nhiệt, tổng công suất 6 kW) cung cấp nhiệt để hóa hơi hỗn hợp đáy tháp.
    \item Hơi đi lên qua 6 mâm xuyên lỗ, tiếp xúc với dòng lỏng hồi lưu đi xuống.
    \item Hơi giàu ethanol thoát ra ở đỉnh tháp được ngưng tụ tại bộ ngưng tụ dạng ống xoắn ruột gà.
    \item Một phần sản phẩm ngưng tụ hồi lưu về tháp, phần còn lại là sản phẩm đỉnh (nồng độ mục tiêu 90\% vol).
    \item Sản phẩm đáy (nghèo ethanol) được lấy ra từ reboiler.
\end{enumerate}

\subsubsection{Thông số công nghệ của hệ thống chưng cất}

\begin{table}[htbp]
\centering
\begin{tabular}{|l|c|c|}
\hline
\textbf{Thông số} & \textbf{Giá trị} & \textbf{Đơn vị} \\
\hline
\multicolumn{3}{|l|}{\textit{Tháp chưng cất}} \\
\hline
Đường kính tháp & 150 & mm \\
\hline
Chiều cao tháp & 800 & mm \\
\hline
Số mâm & 6 & mâm \\
\hline
Loại mâm & \multicolumn{2}{c|}{Mâm xuyên lỗ, 15 lỗ, có ống chảy truyền} \\
\hline
Đường kính lỗ & \multicolumn{2}{c|}{---} \\
\hline
Khoảng cách mâm & 140 & mm \\
\hline
Vật liệu mâm & \multicolumn{2}{c|}{SS304} \\
\hline
Vật liệu thân tháp & \multicolumn{2}{c|}{---} \\
\hline
Vị trí nhập liệu & \multicolumn{2}{c|}{Mâm thứ 4 (từ dưới lên)} \\
\hline
\multicolumn{3}{|l|}{\textit{Reboiler}} \\
\hline
Loại reboiler & \multicolumn{2}{c|}{Điện trở gia nhiệt} \\
\hline
Công suất reboiler tối đa & 6000 (2$\times$3000) & W \\
\hline
\multicolumn{3}{|l|}{\textit{Condenser}} \\
\hline
Loại condenser & \multicolumn{2}{c|}{Ống xoắn ruột gà} \\
\hline
Đường kính ống & 13 & mm \\
\hline
Số vòng xoắn & 22 & vòng \\
\hline
Đường kính xoắn & 100 & mm \\
\hline
Vật liệu & \multicolumn{2}{c|}{SS304} \\
\hline
Diện tích trao đổi nhiệt & 0.28 & m$^2$ \\
\hline
Lưu lượng nước làm mát & 7.2 & L/min \\
\hline
\multicolumn{3}{|l|}{\textit{Bình hồi lưu}} \\
\hline
Kích thước & $\phi$20 $\times$ 245 & mm \\
\hline
Chỉ thị mức & \multicolumn{2}{c|}{Có} \\
\hline
\multicolumn{3}{|l|}{\textit{Hệ thống nhập liệu}} \\
\hline
Bộ gia nhiệt nhập liệu & \multicolumn{2}{c|}{Có (thông số ---)} \\
\hline
Lưu lượng nhập liệu & $\sim$4.8 (80 ml/min) & L/h \\
\hline
\multicolumn{3}{|l|}{\textit{Van điều khiển nước làm mát}} \\
\hline
Loại van & \multicolumn{2}{c|}{---} \\
\hline
Điều khiển & \multicolumn{2}{c|}{Tự động} \\
\hline
\multicolumn{3}{|l|}{\textit{Điều kiện vận hành}} \\
\hline
Nồng độ nhập liệu & 10 & \% vol \\
\hline
Nồng độ sản phẩm đỉnh & 90 & \% vol \\
\hline
Áp suất vận hành & 1.0 & bar \\
\hline
\end{tabular}
\caption{Thông số thiết bị chính của hệ thống chưng cất}
\label{tab:equipment_specs}
\end{table}

\textbf{Tính toán diện tích trao đổi nhiệt của condenser:}

Diện tích trao đổi nhiệt của condenser dạng ống xoắn ruột gà được tính theo công thức:
\begin{equation}
A = \pi \cdot d \cdot L = \pi \cdot d \cdot (n \cdot \pi \cdot D)
\end{equation}

Trong đó:
\begin{itemize}
    \item $d = 13$ mm -- đường kính ống
    \item $n = 22$ vòng -- số vòng xoắn
    \item $D = 100$ mm -- đường kính xoắn
\end{itemize}

Thay số:
\begin{equation}
A = \pi \times 0.013 \times (22 \times \pi \times 0.1) = \pi \times 0.013 \times 6.912 \approx 0.28 \text{ m}^2
\end{equation}

% ------------------------------------------------------------
\subsection{Thiết bị đo lường và điều khiển}
% ------------------------------------------------------------

\subsubsection{Thiết bị đo}

Hệ thống được trang bị các thiết bị đo sau:

\begin{table}[H]
\centering
\small
\begin{tabularx}{\textwidth}{|X|l|c|c|}
\hline
\textbf{Thiết bị} & \textbf{Loại/Model} & \textbf{Dải đo} & \textbf{Tín hiệu} \\
\hline
Cảm biến nhiệt độ đỉnh tháp & Pt100 + Transmitter & 0 -- 200$^\circ$C & 4-20 mA \\
\hline
Cảm biến nhiệt độ nhập liệu & Pt100 + Transmitter & 0 -- 200$^\circ$C & 4-20 mA \\
\hline
Cảm biến áp suất đỉnh tháp & SITRANS P200 & 1 -- 60 bar & 4-20 mA \\
\hline
Cảm biến lưu lượng hồi lưu & DPM-1507-G1-L443 & 50 -- 700 ml/min & 4-20 mA \\
\hline
Cảm biến mức bình hồi lưu & NMC-H1.2G603 & 0 -- 1200 mm & 4-20 mA \\
\hline
\end{tabularx}
\caption{Danh sách thiết bị đo}
\label{tab:sensors}
\end{table}

\textbf{Cảm biến áp suất đỉnh tháp -- SITRANS P200:}

Cảm biến áp suất SITRANS P200 [8] được lắp đặt tại đường ống hơi ra khỏi đỉnh tháp, trước bộ ngưng tụ. Đây là cảm biến áp suất compact với các đặc điểm kỹ thuật:
\begin{itemize}
    \item Nguyên lý đo: Cảm biến áp điện trở (piezoresistive) với màng gốm ceramic
    \item Dải đo: 1 -- 60 bar (áp suất tương đối và tuyệt đối)
    \item Tín hiệu ra: 4-20 mA hoặc 0-10 V DC
    \item Độ chính xác: $\pm$0.25\% của dải đo
    \item Vỏ bọc: Thép không gỉ, IP65/IP67
    \item Kết nối điện: Đầu nối M12 hoặc EN 175301-803-A
\end{itemize}

\textbf{Cảm biến lưu lượng hồi lưu -- Kobold DPM-1507-G1-L443:}

Lưu lượng kế cánh quay thể tích nhỏ Kobold DPM [9] được sử dụng để đo lưu lượng dòng hồi lưu:
\begin{itemize}
    \item Nguyên lý đo: Cánh quay (Rotating Vane) với cảm biến quang học
    \item Dải đo: 50 -- 700 ml/min (nước)
    \item Vật liệu: Thép không gỉ 1.4404
    \item Áp suất làm việc tối đa: 16 bar
    \item Nhiệt độ làm việc: -40 đến +80$^\circ$C
    \item Tín hiệu ra: 4-20 mA (mã L443 -- đầu nối DIN 43650)
    \item Kết nối: G$\frac{1}{8}$ ren trong (mã G1)
    \item Độ chính xác: $\pm$1\% của thang đo
\end{itemize}

\textbf{Cảm biến mức bình hồi lưu -- Kobold NMC-H1.2G603:}

Cảm biến mức điện dung Kobold NMC [10] được lắp đặt tại bình chứa sản phẩm hồi lưu:
\begin{itemize}
    \item Nguyên lý đo: Điện dung (Capacitive)
    \item Chiều dài que đo: 1200 mm (mã H1.2)
    \item Dải áp suất: -1 đến +30 bar
    \item Nhiệt độ môi trường đo: -20 đến +200$^\circ$C (Model NMC-H)
    \item Sai số đo: < 1.5\% chiều dài que đo
    \item Tín hiệu ra: 4-20 mA
    \item Kết nối cơ khí: G1 ren ngoài, thép không gỉ (mã G6)
    \item Bảo vệ: IP65, phù hợp ứng dụng ATEX
\end{itemize}

\textbf{Cảm biến nhiệt độ -- Pt100 với Transmitter 4-20mA:}

Cảm biến nhiệt độ RTD Pt100 theo tiêu chuẩn IEC 60751 được sử dụng kết hợp với bộ chuyển đổi tín hiệu 4-20mA:
\begin{itemize}
    \item Loại cảm biến: Pt100 (100$\Omega$ tại 0$^\circ$C, $\alpha$ = 0.00385)
    \item Dải đo: 0 -- 200$^\circ$C (có thể cấu hình)
    \item Cấp chính xác: Class A theo IEC 60751
    \item Tín hiệu ra: 4-20 mA (2-wire loop powered)
    \item Kết nối: 3-wire hoặc 4-wire RTD
    \item Nguồn cấp: 10-30 VDC
\end{itemize}

% Placeholder cho hình ảnh thực tế
\begin{figure}[H]
\centering
\fbox{\parbox{0.6\textwidth}{\centering\vspace{3cm}Hình ảnh cảm biến áp suất SITRANS P200\\(Sẽ cập nhật sau)\vspace{3cm}}}
\caption{Cảm biến áp suất đỉnh tháp SITRANS P200}
\label{fig:pressure_sensor}
\end{figure}

\subsubsection{Bộ điều khiển PLC S7-1200}

Hệ thống sử dụng PLC Siemens S7-1200 CPU 1214C làm bộ điều khiển trung tâm, với các đặc điểm:

\begin{itemize}
    \item CPU 1214C DC/DC/DC
    \item 14 DI / 10 DO tích hợp
    \item 2 AI tích hợp (0-10V)
    \item 2 module mở rộng SM 1231 AI (4 AI mỗi module)
    \item 1 module mở rộng SM 1232 AQ (2 AO)
    \item Giao tiếp PROFINET
\end{itemize}

\textbf{Cấu hình I/O:}

\begin{table}[H]
\centering
\begin{tabular}{|l|l|l|}
\hline
\textbf{Tín hiệu} & \textbf{Địa chỉ} & \textbf{Mô tả} \\
\hline
AI0 & IW64 & Nhiệt độ đỉnh tháp \\
\hline
AI1 & IW66 & Nhiệt độ nhập liệu \\
\hline
AI2 & IW68 & Áp suất đỉnh tháp \\
\hline
AI3 & IW70 & Lưu lượng nhập liệu \\
\hline
AO0 & QW64 & Công suất reboiler \\
\hline
AO1 & QW66 & Độ mở van làm mát \\
\hline
AO2 & QW68 & Tốc độ bơm nhập liệu \\
\hline
\end{tabular}
\caption{Cấu hình I/O của PLC}
\label{tab:io_config}
\end{table}

% ------------------------------------------------------------
\subsection{Thiết kế các vòng điều khiển}
% ------------------------------------------------------------

Hệ thống chưng cất được thiết kế với các vòng điều khiển sau:

\begin{enumerate}
    \item \textbf{Vòng điều khiển lưu lượng nhập liệu (FC-01):} Điều khiển tốc độ bơm để duy trì lưu lượng nhập liệu theo giá trị đặt.
    \item \textbf{Vòng điều khiển nhiệt độ nhập liệu (TC-01):} Điều khiển bộ gia nhiệt nhập liệu để đạt nhiệt độ mong muốn.
    \item \textbf{Vòng điều khiển nhiệt độ đỉnh tháp (TC-02):} Điều khiển công suất reboiler để duy trì nhiệt độ đỉnh tháp.
    \item \textbf{Vòng điều khiển áp suất đỉnh tháp (PC-01):} Điều khiển độ mở van nước làm mát để ổn định áp suất đỉnh tháp -- \textbf{đây là vòng điều khiển mới được thiết kế trong đồ án này}.
\end{enumerate}

% ------------------------------------------------------------
\subsection{Chiến lược điều khiển}
% ------------------------------------------------------------

\subsubsection{Phương án vận hành hiện tại (Baseline)}

Trong phương án vận hành hiện tại, van nước làm mát được mở cố định ở mức 100\% để đảm bảo áp suất đỉnh tháp luôn ổn định. Phương án này có ưu điểm là đơn giản, không cần hệ thống điều khiển phức tạp, nhưng gây ngưng tụ quá mức và hạn chế năng suất sản phẩm.

\begin{figure}[H]
\centering
\begin{tikzpicture}[node distance=2cm]
    % Process
    \node[block, minimum width=4cm] (condenser) {Bộ ngưng tụ};
    \node[left=2cm of condenser] (vapor) {Hơi đỉnh tháp};
    \node[right=2cm of condenser] (liquid) {Lỏng ngưng tụ};
    
    % Cooling water - fixed
    \node[above=2.5cm of condenser] (cw_in) {Nước làm mát (100\%)};
    \node[block, above=0.5cm of condenser, minimum width=2cm, fill=gray!30] (valve) {Van mở 100\%};
    
    % Arrows
    \draw[arrow] (vapor) -- (condenser);
    \draw[arrow] (condenser) -- (liquid);
    \draw[-Stealth, thick] (cw_in.south) -- (valve.north);
    \draw[line] (valve) -- (condenser);
    
\end{tikzpicture}
\caption{Sơ đồ vận hành hiện tại -- Van làm mát mở cố định 100\%}
\label{fig:baseline_control}
\end{figure}

\subsubsection{Phương án điều khiển tối ưu năng suất (Đề xuất)}

Phương án đề xuất sử dụng bộ điều khiển PID để điều chỉnh độ mở van nước làm mát dựa trên áp suất đỉnh tháp. Van chỉ mở đủ để duy trì áp suất ở giá trị đặt, giảm lượng ngưng tụ không cần thiết, từ đó tăng năng suất sản phẩm đỉnh trong khi giữ nguyên năng lượng và chất lượng.

\begin{figure}[H]
\centering
\begin{tikzpicture}[node distance=1.8cm]
    % Setpoint
    \node (sp) {SP$_P$};
    
    % Sum
    \node[sum, right=1cm of sp] (sum) {};
    
    % Controller
    \node[block, right=1.2cm of sum] (controller) {PID};
    
    % Valve
    \node[block, right=1.2cm of controller] (valve) {Van làm mát};
    
    % Condenser
    \node[block, right=1.2cm of valve, minimum width=2.5cm] (condenser) {Bộ ngưng tụ};
    
    % Pressure output
    \node[right=1cm of condenser] (output) {$P$};
    
    % Sensor
    \node[block, below=1cm of controller] (sensor) {Cảm biến áp suất};
    
    % Arrows
    \draw[arrow] (sp) -- node[above, font=\normalsize] {$+$} (sum);
    \draw[arrow] (sum) -- node[above, font=\normalsize] {$e$} (controller);
    \draw[arrow] (controller) -- node[above, font=\normalsize] {$u$} (valve);
    \draw[arrow] (valve) -- node[above, font=\normalsize] {$Q_C$} (condenser);
    \draw[arrow] (condenser) -- (output);
    
    % Feedback
    \coordinate (fb) at ($(condenser.east)!0.5!(output.west)$);
    \draw[line] (fb) |- (sensor);
    \draw[arrow] (sensor) -| node[pos=0.95, left, font=\normalsize] {$-$} (sum);
    
\end{tikzpicture}
\caption{Sơ đồ khối vòng điều khiển áp suất đỉnh tháp (PC-01)}
\label{fig:pressure_control_loop}
\end{figure}

\textbf{Nguyên lý hoạt động:}
\begin{itemize}
    \item Khi áp suất đỉnh tháp tăng cao hơn setpoint: Bộ điều khiển tăng độ mở van làm mát $\rightarrow$ tăng công suất ngưng tụ $\rightarrow$ áp suất giảm.
    \item Khi áp suất đỉnh tháp giảm thấp hơn setpoint: Bộ điều khiển giảm độ mở van làm mát $\rightarrow$ giảm công suất ngưng tụ $\rightarrow$ áp suất tăng.
\end{itemize}

\subsubsection{Điều khiển nhiệt độ đỉnh tháp}

Song song với vòng điều khiển áp suất, vòng điều khiển nhiệt độ đỉnh tháp (TC-02) duy trì nhiệt độ ổn định để đảm bảo chất lượng sản phẩm:

\begin{figure}[H]
\centering
\begin{tikzpicture}[node distance=1.8cm]
    % Setpoint
    \node (sp) {SP$_T$};
    
    % Sum
    \node[sum, right=1cm of sp] (sum) {};
    
    % Controller
    \node[block, right=1.2cm of sum] (controller) {PID};
    
    % Reboiler
    \node[block, right=1.2cm of controller] (reboiler) {Reboiler};
    
    % Column
    \node[block, right=1.2cm of reboiler, minimum width=2cm] (column) {Tháp chưng cất};
    
    % Temperature output
    \node[right=1cm of column] (output) {$T$};
    
    % Sensor
    \node[block, below=1cm of controller] (sensor) {Cảm biến nhiệt độ};
    
    % Arrows
    \draw[arrow] (sp) -- node[above, font=\normalsize] {$+$} (sum);
    \draw[arrow] (sum) -- node[above, font=\normalsize] {$e$} (controller);
    \draw[arrow] (controller) -- node[above, font=\normalsize] {$u$} (reboiler);
    \draw[arrow] (reboiler) -- node[above, font=\normalsize] {$Q_R$} (column);
    \draw[arrow] (column) -- (output);
    
    % Feedback
    \coordinate (fb) at ($(column.east)!0.5!(output.west)$);
    \draw[line] (fb) |- (sensor);
    \draw[arrow] (sensor) -| node[pos=0.95, left, font=\normalsize] {$-$} (sum);
    
\end{tikzpicture}
\caption{Sơ đồ khối vòng điều khiển nhiệt độ đỉnh tháp (TC-02)}
\label{fig:temp_control_loop}
\end{figure}
