% ============================================================
% Lab 1: Khảo sát cảm biến nhiệt độ
% ============================================================

\setcounter{section}{1}
\setcounter{subsection}{0}
\section*{BÀI THÍ NGHIỆM 1: KHẢO SÁT CẢM BIẾN NHIỆT ĐỘ}
\addcontentsline{toc}{section}{Bài thí nghiệm 1: Khảo sát cảm biến nhiệt độ}

% ------------------------------------------------------------
\subsection{Mục tiêu}
% ------------------------------------------------------------

\begin{itemize}
    \item Hiểu nguyên lý hoạt động của các loại cảm biến nhiệt độ công nghiệp.
    \item Khảo sát đặc tính của cảm biến nhiệt điện trở (RTD) và cặp nhiệt điện (Thermocouple).
    \item Thực hiện hiệu chuẩn cảm biến và xây dựng đường cong hiệu chuẩn.
\end{itemize}

% ------------------------------------------------------------
\subsection{Cơ sở lý thuyết}
% ------------------------------------------------------------

\subsubsection{Cảm biến nhiệt điện trở (RTD)}

RTD (Resistance Temperature Detector) là cảm biến đo nhiệt độ dựa trên nguyên lý thay đổi điện trở của kim loại theo nhiệt độ. Loại phổ biến nhất là Pt100, sử dụng dây platinum với điện trở 100$\Omega$ tại 0°C.

Phương trình quan hệ điện trở - nhiệt độ:
\begin{equation}
R_T = R_0 (1 + \alpha T)
\end{equation}

Trong đó:
\begin{itemize}
    \item $R_T$: Điện trở tại nhiệt độ T ($\Omega$)
    \item $R_0$: Điện trở tại 0°C ($\Omega$)
    \item $\alpha$: Hệ số nhiệt điện trở (°C$^{-1}$)
    \item $T$: Nhiệt độ (°C)
\end{itemize}

\subsubsection{Cặp nhiệt điện (Thermocouple)}

Thermocouple hoạt động dựa trên hiệu ứng Seebeck: khi hai kim loại khác nhau được nối với nhau và có chênh lệch nhiệt độ giữa hai đầu, sẽ sinh ra một suất điện động (EMF).

\begin{equation}
E = S_{AB} \cdot (T_1 - T_2)
\end{equation}

Trong đó:
\begin{itemize}
    \item $E$: Suất điện động (mV)
    \item $S_{AB}$: Hệ số Seebeck của cặp kim loại A-B ($\mu$V/°C)
    \item $T_1, T_2$: Nhiệt độ tại hai đầu nối (°C)
\end{itemize}

Các loại thermocouple phổ biến:
\begin{itemize}
    \item Type K (Chromel-Alumel): -200°C đến 1260°C
    \item Type J (Iron-Constantan): -40°C đến 760°C
    \item Type T (Copper-Constantan): -200°C đến 350°C
\end{itemize}

% ------------------------------------------------------------
\subsection{Thiết bị thí nghiệm}
% ------------------------------------------------------------

\begin{itemize}
    \item PLC Siemens S7-1200 với module analog input
    \item Cảm biến Pt100 (RTD)
    \item Cặp nhiệt điện Type K
    \item Bộ gia nhiệt điều khiển được
    \item Nhiệt kế chuẩn
    \item Phần mềm TIA Portal V16
\end{itemize}

% ------------------------------------------------------------
\subsection{Quy trình thực hiện}
% ------------------------------------------------------------

\subsubsection{Chuẩn bị}
\begin{enumerate}
    \item Kết nối cảm biến với PLC S7-1200.
    \item Cấu hình module analog input trong TIA Portal.
    \item Kiểm tra kết nối và tín hiệu đầu vào.
\end{enumerate}

\subsubsection{Khảo sát cảm biến Pt100}
\begin{enumerate}
    \item Đặt nhiệt độ chuẩn tại các điểm: 30°C, 40°C, 50°C, 60°C, 70°C, 80°C.
    \item Ghi nhận giá trị điện trở hoặc tín hiệu analog tương ứng.
    \item Xây dựng đường cong hiệu chuẩn.
\end{enumerate}

\subsubsection{Khảo sát cặp nhiệt điện Type K}
\begin{enumerate}
    \item Thực hiện tương tự với các điểm nhiệt độ chuẩn.
    \item Ghi nhận giá trị EMF hoặc tín hiệu analog.
    \item So sánh với bảng chuẩn của thermocouple Type K.
\end{enumerate}

% ------------------------------------------------------------
\subsection{Kết quả thí nghiệm}
% ------------------------------------------------------------

\subsubsection{Đo nhiệt độ môi trường}

Tiến hành đo nhiệt độ môi trường trong 2 phút để kiểm tra độ ổn định của cảm biến Pt100.

\begin{figure}[H]
\centering
\begin{tikzpicture}
\begin{axis}[
    width=0.9\textwidth,
    height=6cm,
    xlabel={Thời gian (s)},
    ylabel={Nhiệt độ ($^\circ$C)},
    grid=both,
    grid style={line width=.1pt, draw=gray!30},
    major grid style={line width=.2pt,draw=gray!50},
    xmin=0, xmax=130,
    ymin=25.0, ymax=25.5,
    legend pos=north east,
]
\addplot[blue, thick, mark=*, mark size=1pt] table[x=time_s, y=PV, col sep=comma] {../tmp/TN1_processed.csv};
\addlegendentry{Nhiệt độ đo (Pt100)}
\addplot[red, dashed, domain=0:130] {25.304};
\addlegendentry{Giá trị trung bình = $25.30^\circ$C}
\end{axis}
\end{tikzpicture}
\caption{Kết quả đo nhiệt độ môi trường bằng cảm biến Pt100}
\label{fig:tn1_baseline}
\end{figure}

\textbf{Kết quả phân tích:}
\begin{itemize}
    \item Nhiệt độ trung bình: $25.30^\circ$C
    \item Độ lệch chuẩn: $< 0.01^\circ$C
    \item Dải dao động: $25.30 - 25.32^\circ$C
    \item Kết luận: Cảm biến Pt100 hoạt động ổn định, độ nhiễu rất thấp
\end{itemize}

\subsubsection{Kết quả khảo sát Pt100}

\begin{table}[H]
\centering
\begin{tabular}{|c|c|c|c|}
\hline
\textbf{Nhiệt độ chuẩn (°C)} & \textbf{Điện trở ($\Omega$)} & \textbf{Giá trị đo (°C)} & \textbf{Sai số (\%)} \\
\hline
25 (môi trường) & 109.73 & 25.30 & -- \\
\hline
30 & 111.67 & -- & -- \\
\hline
40 & 115.54 & -- & -- \\
\hline
50 & 119.40 & -- & -- \\
\hline
60 & 123.24 & -- & -- \\
\hline
70 & 127.07 & -- & -- \\
\hline
80 & 130.89 & -- & -- \\
\hline
\end{tabular}
\caption{Kết quả khảo sát cảm biến Pt100 (điện trở tham khảo theo tiêu chuẩn IEC 60751)}
\end{table}

\textit{Ghi chú: Điện trở lý thuyết của Pt100 tại nhiệt độ T được tính theo công thức: $R_T = 100 \times (1 + 0.00385 \times T)$}

\subsubsection{Kết quả khảo sát Thermocouple Type K}

\begin{table}[H]
\centering
\begin{tabular}{|c|c|c|c|}
\hline
\textbf{Nhiệt độ chuẩn (°C)} & \textbf{EMF (mV)} & \textbf{Giá trị đo (°C)} & \textbf{Sai số (\%)} \\
\hline
25 (môi trường) & 1.00 & -- & -- \\
\hline
30 & 1.20 & -- & -- \\
\hline
40 & 1.61 & -- & -- \\
\hline
50 & 2.02 & -- & -- \\
\hline
60 & 2.44 & -- & -- \\
\hline
70 & 2.85 & -- & -- \\
\hline
80 & 3.27 & -- & -- \\
\hline
\end{tabular}
\caption{Kết quả khảo sát cặp nhiệt điện Type K (EMF tham khảo theo bảng NIST)}
\end{table}

% ------------------------------------------------------------
\subsection{Bàn luận}
% ------------------------------------------------------------

\subsubsection{Độ chính xác của cảm biến Pt100}

Từ kết quả đo nhiệt độ môi trường (Hình \ref{fig:tn1_baseline}):
\begin{itemize}
    \item Cảm biến Pt100 cho độ ổn định rất cao với độ lệch chuẩn $< 0.01^\circ$C
    \item Tín hiệu đo gần như không có nhiễu, thể hiện chất lượng cảm biến tốt
    \item Phù hợp cho các ứng dụng yêu cầu độ chính xác cao trong điều khiển nhiệt độ
\end{itemize}

\subsubsection{So sánh RTD và Thermocouple}

\begin{table}[H]
\centering
\begin{tabular}{|l|c|c|}
\hline
\textbf{Tiêu chí} & \textbf{RTD (Pt100)} & \textbf{Thermocouple} \\
\hline
Độ chính xác & Cao ($\pm 0.1^\circ$C) & Trung bình ($\pm 1-2^\circ$C) \\
\hline
Dải đo & $-200$ đến $850^\circ$C & $-200$ đến $1800^\circ$C \\
\hline
Thời gian đáp ứng & Chậm (vài giây) & Nhanh ($< 1$s) \\
\hline
Chi phí & Cao & Thấp \\
\hline
Độ tuyến tính & Tốt & Kém (cần bù) \\
\hline
\end{tabular}
\caption{So sánh RTD và Thermocouple}
\end{table}

\subsubsection{Nguồn sai số trong phép đo}

\begin{itemize}
    \item \textbf{Sai số hệ thống:} Do hiệu chuẩn, điện trở dây dẫn (đặc biệt với RTD)
    \item \textbf{Sai số ngẫu nhiên:} Nhiễu điện từ, dao động nguồn cấp
    \item \textbf{Sai số do môi trường:} Ảnh hưởng của độ ẩm, áp suất
\end{itemize}

% ------------------------------------------------------------
\subsection{Kết luận}
% ------------------------------------------------------------

Qua thí nghiệm khảo sát cảm biến nhiệt độ, rút ra các kết luận:

\begin{enumerate}
    \item Cảm biến Pt100 có độ ổn định và chính xác cao, phù hợp cho điều khiển nhiệt độ trong công nghiệp.
    \item Kết quả đo nhiệt độ môi trường $25.30^\circ$C với độ dao động $< 0.02^\circ$C chứng tỏ hệ thống đo lường hoạt động tốt.
    \item RTD phù hợp cho các ứng dụng yêu cầu độ chính xác cao trong dải nhiệt độ thấp/trung bình, trong khi Thermocouple phù hợp cho nhiệt độ cao và đáp ứng nhanh.
\end{enumerate}

% ------------------------------------------------------------
\subsection{Câu hỏi kiểm tra}
% ------------------------------------------------------------

\textbf{Câu 1: So sánh ưu nhược điểm của RTD và Thermocouple?}

\begin{itemize}
    \item \textbf{RTD:} Ưu điểm: độ chính xác cao, độ tuyến tính tốt, ổn định lâu dài. Nhược điểm: đắt tiền, đáp ứng chậm, dải đo hẹp hơn, cần nguồn cấp.
    \item \textbf{Thermocouple:} Ưu điểm: giá rẻ, đáp ứng nhanh, dải đo rộng, không cần nguồn cấp. Nhược điểm: độ chính xác thấp hơn, cần bù nhiệt độ đầu lạnh, độ phi tuyến cao.
\end{itemize}

\textbf{Câu 2: Tại sao cần hiệu chuẩn cảm biến?}

Hiệu chuẩn cảm biến cần thiết vì:
\begin{itemize}
    \item Bù trừ sai số chế tạo và sai số lắp đặt
    \item Đảm bảo độ chính xác của phép đo theo yêu cầu
    \item Xác định đường cong hiệu chuẩn để chuyển đổi tín hiệu sang đơn vị đo
    \item Phát hiện sự suy giảm chất lượng cảm biến theo thời gian
\end{itemize}

\textbf{Câu 3: Giải thích nguyên lý hoạt động của Pt100?}

Pt100 hoạt động dựa trên nguyên lý thay đổi điện trở của kim loại Platinum theo nhiệt độ:
\begin{itemize}
    \item Điện trở của Platinum tăng gần tuyến tính với nhiệt độ
    \item Tại $0^\circ$C, điện trở là $100\Omega$ (gốc của tên gọi Pt100)
    \item Hệ số nhiệt điện trở $\alpha = 0.00385\, ^\circ\text{C}^{-1}$ (theo tiêu chuẩn IEC 60751)
    \item Công thức: $R_T = R_0(1 + \alpha T)$, với $T$ là nhiệt độ tính bằng $^\circ$C
\end{itemize}

% ------------------------------------------------------------
\subsection{Tài liệu tham khảo}
% ------------------------------------------------------------

\begin{enumerate}[label={[\arabic*]}]
    \item B. N. Pha, \textit{Thiết bị Đo lường và Điều khiển}. Trường Đại học Bách khoa, ĐHQG-HCM.
\end{enumerate}

\newpage
