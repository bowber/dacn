% ============================================================
% Lab 1: Làm quen với TIA Portal và thiết kế HMI
% ============================================================

\setcounter{section}{1}
\setcounter{subsection}{0}
\section*{BÀI THÍ NGHIỆM 1: LÀM QUEN VỚI TIA PORTAL VÀ THIẾT KẾ HMI}
\addcontentsline{toc}{section}{Bài thí nghiệm 1: Làm quen với TIA Portal và thiết kế HMI}

% ------------------------------------------------------------
\subsection{Mục tiêu}
% ------------------------------------------------------------

\begin{itemize}
    \item Làm quen với phần mềm TIA Portal V16 và cách tạo project mới.
    \item Cấu hình PLC S7-1200 và kết nối với cảm biến nhiệt độ.
    \item Thiết kế giao diện HMI để hiển thị giá trị nhiệt độ đo được.
\end{itemize}

% ------------------------------------------------------------
\subsection{Cơ sở lý thuyết}
% ------------------------------------------------------------

\subsubsection{Giới thiệu TIA Portal}

TIA Portal (Totally Integrated Automation Portal) là nền tảng phần mềm tích hợp của Siemens, cho phép cấu hình, lập trình và giám sát các thiết bị tự động hóa như PLC, HMI, và drive trong một môi trường thống nhất.

Các thành phần chính của TIA Portal:
\begin{itemize}
    \item \textbf{STEP 7:} Lập trình PLC (LAD, FBD, SCL, STL)
    \item \textbf{WinCC:} Thiết kế giao diện HMI/SCADA
\end{itemize}

\subsubsection{PLC Siemens S7-1200}

S7-1200 là dòng PLC compact của Siemens, phù hợp cho các ứng dụng tự động hóa vừa và nhỏ. Đặc điểm:
\begin{itemize}
    \item CPU tích hợp các cổng I/O số và analog
    \item Hỗ trợ giao tiếp PROFINET (Ethernet)
    \item Có thể mở rộng thêm module I/O, truyền thông
\end{itemize}

\subsubsection{Giao diện HMI}

HMI (Human-Machine Interface) là giao diện giữa người vận hành và hệ thống điều khiển. Trong TIA Portal, WinCC cho phép thiết kế các màn hình hiển thị với:
\begin{itemize}
    \item Hiển thị giá trị biến (số, đồ thị, gauge)
    \item Nút nhấn, công tắc điều khiển
    \item Cảnh báo và báo động (alarm)
\end{itemize}

% ------------------------------------------------------------
\subsection{Thiết bị thí nghiệm}
% ------------------------------------------------------------

\begin{itemize}
    \item Máy tính cài đặt TIA Portal V16
    \item PLC Siemens S7-1200 CPU 1214C
    \item Cảm biến nhiệt độ Pt100
    \item Kết nối WiFi giữa PLC và máy tính
\end{itemize}

% ------------------------------------------------------------
\subsection{Quy trình thực hiện}
% ------------------------------------------------------------

\subsubsection{Tạo project mới trong TIA Portal}
\begin{enumerate}
    \item Mở TIA Portal V16, chọn \texttt{Create new project}.
    \item Đặt tên project và chọn đường dẫn lưu trữ.
    \item Chọn \texttt{Configure a device} để thêm PLC.
\end{enumerate}

\subsubsection{Cấu hình PLC S7-1200}
\begin{enumerate}
    \item Thêm PLC S7-1200 CPU 1214C DC/DC/DC vào project.
    \item Cấu hình địa chỉ IP cho PLC (ví dụ: 192.168.0.1).
    \item Tạo tag cho biến nhiệt độ (PV - Process Value).
\end{enumerate}

\subsubsection{Thiết kế giao diện HMI}
\begin{enumerate}
    \item Thêm HMI mô phỏng (WinCC RT Advanced) vào project.
    \item Tạo màn hình hiển thị với các thành phần:
    \begin{itemize}
        \item Text hiển thị tiêu đề
        \item I/O Field hiển thị giá trị nhiệt độ (liên kết với tag PV)
        \item Trend graph để theo dõi biến thiên nhiệt độ theo thời gian
        \item Đơn vị đo (°C)
    \end{itemize}
    \item Cấu hình kết nối HMI với PLC qua PROFINET.
\end{enumerate}

\subsubsection{Download và kiểm tra}
\begin{enumerate}
    \item Compile project để kiểm tra lỗi.
    \item Download chương trình xuống PLC.
    \item Chạy WinCC RT trên máy tính và quan sát giá trị nhiệt độ hiển thị.
\end{enumerate}

% ------------------------------------------------------------
\subsection{Kết quả thí nghiệm}
% ------------------------------------------------------------

\subsubsection{Giao diện HMI}

\begin{figure}[H]
\centering
\includegraphics[width=0.8\textwidth]{../assets/HMI_TN1.jpg}
\caption{Giao diện HMI hiển thị giá trị nhiệt độ đo được}
\label{fig:hmi_tn1}
\end{figure}

Giao diện HMI được thiết kế với các thành phần:
\begin{itemize}
    \item Hiển thị giá trị nhiệt độ hiện tại (PV) từ cảm biến Pt100
    \item Trend graph theo dõi biến thiên nhiệt độ theo thời gian
    \item Đơn vị đo: °C
\end{itemize}

\subsubsection{Kết quả đo nhiệt độ}

Tiến hành đo nhiệt độ môi trường trong 2 phút để kiểm tra hoạt động của hệ thống.

\begin{figure}[H]
\centering
\begin{tikzpicture}
\begin{axis}[
    width=0.9\textwidth,
    height=6cm,
    xlabel={Thời gian (s)},
    ylabel={Nhiệt độ ($^\circ$C)},
    grid=both,
    grid style={line width=.1pt, draw=gray!30},
    major grid style={line width=.2pt,draw=gray!50},
    xmin=0, xmax=130,
    ymin=25.0, ymax=25.5,
    legend pos=north east,
]
\addplot[blue, thick, mark=*, mark size=1pt] table[x=time_s, y=PV, col sep=comma] {../data/TN1_processed.csv};
\addlegendentry{Nhiệt độ đo được}
\addplot[red, dashed, domain=0:130] {25.304};
\addlegendentry{Giá trị trung bình = $25.30^\circ$C}
\end{axis}
\end{tikzpicture}
\caption{Kết quả đo nhiệt độ môi trường}
\label{fig:tn1_result}
\end{figure}

\textbf{Kết quả:}
\begin{itemize}
    \item Nhiệt độ trung bình: $25.30^\circ$C
    \item Dải dao động: $25.30 - 25.32^\circ$C
    \item Hệ thống hoạt động ổn định, giá trị hiển thị chính xác trên HMI
\end{itemize}

% ------------------------------------------------------------
\subsection{Bàn luận}
% ------------------------------------------------------------

\begin{itemize}
    \item TIA Portal cung cấp môi trường tích hợp thuận tiện cho việc cấu hình PLC và thiết kế HMI.
    \item Việc kết nối giữa HMI và PLC qua PROFINET đơn giản, chỉ cần cấu hình địa chỉ IP và liên kết tag.
    \item Giao diện HMI cho phép giám sát giá trị nhiệt độ theo thời gian thực.
    \item Dữ liệu đo được có thể ghi lại (logging) để phân tích sau này.
\end{itemize}

% ------------------------------------------------------------
\subsection{Kết luận}
% ------------------------------------------------------------

Qua bài thí nghiệm, đã thực hiện được:

\begin{enumerate}
    \item Tạo project mới trong TIA Portal V16 và cấu hình PLC S7-1200.
    \item Thiết kế giao diện HMI để hiển thị giá trị nhiệt độ từ cảm biến.
    \item Kiểm tra hoạt động của hệ thống với kết quả đo nhiệt độ môi trường ổn định tại $25.30^\circ$C.
\end{enumerate}

% ------------------------------------------------------------
\subsection{Câu hỏi kiểm tra}
% ------------------------------------------------------------

\textbf{Câu 1: TIA Portal là gì? Các thành phần chính?}

TIA Portal (Totally Integrated Automation Portal) là nền tảng phần mềm tích hợp của Siemens để cấu hình, lập trình và giám sát các thiết bị tự động hóa. Các thành phần chính:
\begin{itemize}
    \item STEP 7: Lập trình PLC
    \item WinCC: Thiết kế HMI/SCADA
\end{itemize}

\textbf{Câu 2: Các bước cơ bản để tạo giao diện HMI trong TIA Portal?}

\begin{enumerate}
    \item Thêm thiết bị HMI vào project
    \item Tạo kết nối (connection) giữa HMI và PLC
    \item Thiết kế màn hình với các đối tượng (text, I/O field, button...)
    \item Liên kết các đối tượng với tag của PLC
    \item Compile và download xuống thiết bị
\end{enumerate}

\textbf{Câu 3: PROFINET là gì?}

PROFINET (Process Field Network) là chuẩn Ethernet công nghiệp của Siemens, cho phép:
\begin{itemize}
    \item Truyền thông thời gian thực giữa các thiết bị tự động hóa
    \item Tốc độ cao (100 Mbps trở lên)
    \item Tích hợp với mạng IT tiêu chuẩn
    \item Hỗ trợ các giao thức TCP/IP thông thường
\end{itemize}

% ------------------------------------------------------------
\subsection{Tài liệu tham khảo}
% ------------------------------------------------------------

\begin{enumerate}[label={[\arabic*]}]
    \item B. N. Pha, \textit{Thiết bị Đo lường và Điều khiển}. Trường Đại học Bách khoa, ĐHQG-HCM.
    \item Siemens, \textit{S7-1200 Programmable Controller System Manual}. Siemens AG, 2020.
\end{enumerate}

\newpage
