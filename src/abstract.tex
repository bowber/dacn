% ============================================================
% Tóm tắt đồ án
% ============================================================

\phantomsection
\addcontentsline{toc}{section}{TÓM TẮT}
\section*{TÓM TẮT}

Trong hệ thống chưng cất ethanol -- nước quy mô phòng thí nghiệm, việc ổn định áp suất đỉnh tháp hiện đang được thực hiện bằng cách mở van nước làm mát ở mức tối đa (100\%). Phương pháp này đảm bảo áp suất ổn định nhưng gây lãng phí nước làm mát. Đồng thời, reboiler luôn hoạt động ở công suất tối đa (ước tính 6000 W) bất kể nhu cầu thực tế, dẫn đến tiêu thụ năng lượng không hiệu quả.

Đồ án này nghiên cứu và đề xuất giải pháp tối ưu hóa năng lượng cung cấp cho quá trình chưng cất thông qua hai phương án:
\begin{enumerate}
    \item \textbf{Vòng điều khiển áp suất đỉnh tháp (PC-01):} Sử dụng bộ điều khiển PID để điều chỉnh độ mở van nước làm mát, duy trì áp suất đỉnh tháp ổn định với lượng nước làm mát vừa đủ.
    \item \textbf{Điều khiển PWM công suất reboiler:} Điều chỉnh công suất reboiler theo nhu cầu thực tế thay vì chạy 100\% liên tục, tiết kiệm điện năng tiêu thụ.
\end{enumerate}

Các nội dung chính của đồ án bao gồm:
\begin{itemize}
    \item Phân tích hiện trạng vận hành và xác định các nguồn lãng phí năng lượng
    \item Thiết kế vòng điều khiển áp suất đỉnh tháp (PC-01) với bộ điều khiển PID
    \item Đề xuất phương án điều khiển PWM công suất reboiler
    \item Tính toán và đánh giá hiệu quả tiết kiệm năng lượng
\end{itemize}

Kết quả tính toán cho thấy phương pháp điều khiển van làm mát có thể tiết kiệm 47\% lưu lượng nước làm mát. Việc triển khai điều khiển PWM reboiler sẽ được thực hiện trong giai đoạn Khóa luận Tốt nghiệp để xác định mức tiết kiệm điện năng thực tế.

\newpage
