% ============================================================
% Chương 1: Mở đầu
% ============================================================

\newpage
\pagenumbering{arabic}
\setcounter{page}{1}

\section{TỔNG QUÁT ĐỀ TÀI}

% ------------------------------------------------------------
\subsection{Đặt vấn đề}
% ------------------------------------------------------------

Chưng cất là một trong những quá trình phân tách quan trọng nhất trong công nghiệp hóa học, đặc biệt trong sản xuất ethanol. Hệ thống chưng cất ethanol -- nước được sử dụng rộng rãi trong các nhà máy sản xuất nhiên liệu sinh học, đồ uống có cồn và dung môi công nghiệp. Trong vận hành hệ thống chưng cất, việc tối ưu hóa năng lượng là một trong những mục tiêu quan trọng nhất để giảm chi phí vận hành và nâng cao hiệu quả kinh tế.

Trong hệ thống chưng cất quy mô phòng thí nghiệm tại Khoa Kỹ thuật Hóa học, Đại học Bách Khoa TP.HCM, việc vận hành hiện tại còn tồn tại một số vấn đề về hiệu quả sử dụng năng lượng:

\begin{itemize}
    \item \textbf{Lãng phí nước làm mát:} Van nước làm mát được mở cố định ở mức 100\% để đảm bảo áp suất đỉnh tháp ổn định, dẫn đến lưu lượng nước sử dụng vượt xa nhu cầu thực tế.
    \item \textbf{Lãng phí điện năng reboiler:} Reboiler luôn hoạt động ở công suất tối đa (ước tính 6000 W) bất kể nhu cầu thực tế của quá trình.
    \item \textbf{Thiếu hệ thống điều khiển tự động:} Việc vận hành thủ công không cho phép tối ưu hóa năng lượng theo điều kiện vận hành thực tế.
\end{itemize}

Vấn đề này đặt ra yêu cầu cần thiết phải nghiên cứu và phát triển phương pháp điều khiển thông minh hơn, có khả năng điều chỉnh năng lượng cung cấp phù hợp với nhu cầu thực tế của quá trình.

% ------------------------------------------------------------
\subsection{Mục tiêu nghiên cứu}
% ------------------------------------------------------------

Đồ án này nhằm đạt được các mục tiêu sau:

\begin{enumerate}
    \item Phân tích và đánh giá hiện trạng vận hành của hệ thống chưng cất ethanol -- nước quy mô phòng thí nghiệm, xác định các nguồn lãng phí năng lượng.
    \item Thiết kế vòng điều khiển áp suất đỉnh tháp (PC-01) để tối ưu hóa lưu lượng nước làm mát.
    \item Đề xuất phương án điều khiển PWM công suất reboiler để tiết kiệm điện năng.
    \item Tính toán và đánh giá hiệu quả tiết kiệm năng lượng của các phương pháp điều khiển mới.
\end{enumerate}

% ------------------------------------------------------------
\subsection{Nội dung thực hiện}
% ------------------------------------------------------------

Để đạt được các mục tiêu đề ra, đồ án thực hiện các nội dung sau:

\begin{enumerate}
    \item \textbf{Nghiên cứu lý thuyết:}
    \begin{itemize}
        \item Tìm hiểu cơ sở lý thuyết quá trình chưng cất và đặc tính hệ ethanol -- nước.
        \item Nghiên cứu các phương pháp điều khiển quá trình và tối ưu hóa năng lượng.
    \end{itemize}
    
    \item \textbf{Khảo sát hệ thống hiện tại:}
    \begin{itemize}
        \item Phân tích cấu hình thiết bị và sơ đồ công nghệ.
        \item Đánh giá hiện trạng tiêu thụ năng lượng khi vận hành với van làm mát mở 100\% và reboiler chạy 100\%.
    \end{itemize}
    
    \item \textbf{Thiết kế hệ thống điều khiển:}
    \begin{itemize}
        \item Xây dựng vòng điều khiển áp suất đỉnh tháp (PC-01) thông qua van nước làm mát.
        \item Đề xuất phương án điều khiển PWM công suất reboiler.
        \item Xác định thông số bộ điều khiển PID phù hợp.
    \end{itemize}
    
    \item \textbf{Tính toán và đánh giá:}
    \begin{itemize}
        \item Tính toán cân bằng năng lượng cho hệ thống.
        \item Đánh giá hiệu quả tiết kiệm năng lượng của các phương án điều khiển.
    \end{itemize}
\end{enumerate}

% ------------------------------------------------------------
\subsection{Phạm vi nghiên cứu}
% ------------------------------------------------------------

Đồ án được thực hiện trong phạm vi sau:

\begin{itemize}
    \item \textbf{Đối tượng nghiên cứu:} Hệ thống chưng cất ethanol -- nước quy mô phòng thí nghiệm tại phòng B2-105, Khoa Kỹ thuật Hóa học, Đại học Bách Khoa TP.HCM.
    \item \textbf{Biến điều khiển:} Độ mở van nước làm mát bộ ngưng tụ và công suất reboiler (qua PWM).
    \item \textbf{Biến được điều khiển:} Áp suất đỉnh tháp chưng cất.
    \item \textbf{Mục tiêu tối ưu:} Giảm tiêu thụ năng lượng (nước làm mát và điện năng reboiler) trong khi duy trì chất lượng sản phẩm và áp suất đỉnh tháp ổn định.
    \item \textbf{Giới hạn:} Không thay đổi cấu hình thiết bị hiện có, chỉ thay đổi chiến lược điều khiển.
\end{itemize}

% ------------------------------------------------------------
\subsection{Sản phẩm đồ án}
% ------------------------------------------------------------

Sản phẩm của đồ án chuyên ngành (đề cương Khóa luận Tốt nghiệp) bao gồm:

\begin{enumerate}
    \item Báo cáo phân tích hiện trạng vận hành và tiêu thụ năng lượng của hệ thống chưng cất.
    \item Cơ sở lý thuyết về quá trình chưng cất và điều khiển quá trình.
    \item Thiết kế vòng điều khiển áp suất đỉnh tháp (PC-01) với bộ điều khiển PID.
    \item Đề xuất phương án điều khiển PWM công suất reboiler.
    \item Kết quả tính toán cân bằng năng lượng và đánh giá tiềm năng tiết kiệm.
    \item Kế hoạch triển khai chi tiết cho Khóa luận Tốt nghiệp.
\end{enumerate}

% ------------------------------------------------------------
\subsection{Bố cục dự kiến Khóa luận Tốt nghiệp}
% ------------------------------------------------------------

Khóa luận Tốt nghiệp được phát triển từ đề cương này sẽ có bố cục dự kiến như sau:

\begin{enumerate}
    \item \textbf{Chương 1: Tổng quan}
    \begin{itemize}
        \item Giới thiệu đề tài và đặt vấn đề
        \item Mục tiêu và phạm vi nghiên cứu
    \end{itemize}
    
    \item \textbf{Chương 2: Cơ sở lý thuyết}
    \begin{itemize}
        \item Lý thuyết quá trình chưng cất
        \item Tính chất hệ ethanol -- nước
        \item Điều khiển quá trình và bộ điều khiển PID
    \end{itemize}
    
    \item \textbf{Chương 3: Mô hình thí nghiệm}
    \begin{itemize}
        \item Mô tả hệ thống chưng cất thí nghiệm
        \item Thiết bị đo lường và điều khiển
        \item Thiết kế vòng điều khiển áp suất PC-01
    \end{itemize}
    
    \item \textbf{Chương 4: Triển khai và thực nghiệm}
    \begin{itemize}
        \item Lập trình PLC S7-1200 và thiết kế HMI
        \item Xác định thông số PID (phương pháp Ziegler-Nichols)
        \item Triển khai điều khiển PWM công suất reboiler
        \item Thu thập dữ liệu vận hành thực tế
    \end{itemize}
    
    \item \textbf{Chương 5: Kết quả và thảo luận}
    \begin{itemize}
        \item So sánh kết quả thực nghiệm với tính toán lý thuyết
        \item Đánh giá hiệu quả tiết kiệm năng lượng
        \item Phân tích độ nhạy và giới hạn của phương pháp
    \end{itemize}
    
    \item \textbf{Chương 6: Kết luận và kiến nghị}
    \begin{itemize}
        \item Tổng kết kết quả đạt được
        \item Hạn chế và hướng phát triển
    \end{itemize}
\end{enumerate}

% ------------------------------------------------------------
\subsection{Tiến độ thực hiện Khóa luận Tốt nghiệp}
% ------------------------------------------------------------

Sau khi đề cương được phê duyệt, Khóa luận Tốt nghiệp sẽ được triển khai theo kế hoạch sau:

\begin{table}[H]
\centering
\begin{tabular}{|c|l|p{6.5cm}|}
\hline
\textbf{Tuần} & \textbf{Thời gian} & \textbf{Nội dung công việc} \\
\hline
1--2 & 03/02 -- 16/02/2025 & Sửa chữa/thay thế cảm biến hỏng, hiệu chuẩn thiết bị đo \\
\hline
3--4 & 17/02 -- 02/03/2025 & Thiết kế bộ điều khiển PID, lập trình PLC S7-1200 (TIA Portal), thiết kế HMI \\
\hline
5--6 & 03/03 -- 16/03/2025 & Thí nghiệm vòng điều khiển áp suất PC-01, thu thập dữ liệu, tinh chỉnh PID \\
\hline
7--8 & 17/03 -- 30/03/2025 & Triển khai điều khiển PWM reboiler, đánh giá hiệu quả tiết kiệm năng lượng \\
\hline
9 & 31/03 -- 06/04/2025 & Hoàn thiện báo cáo, chuẩn bị bảo vệ đồ án \\
\hline
\end{tabular}
\caption{Kế hoạch thực hiện Khóa luận Tốt nghiệp}
\label{tab:schedule_ch1}
\end{table}

% ------------------------------------------------------------
\subsection{Tóm tắt kết quả tính toán}
% ------------------------------------------------------------

Dựa trên các tính toán cân bằng năng lượng và truyền nhiệt (chi tiết tại Chương 4), đồ án đã xác định được các thông số vận hành tối ưu cho hệ thống:

\begin{table}[H]
\centering
\begin{tabular}{|l|c|c|}
\hline
\textbf{Thông số} & \textbf{Baseline} & \textbf{Tối ưu} \\
\hline
Độ mở van nước làm mát & 100\% & 28\% \\
\hline
Lưu lượng nước làm mát & 7.2 L/min & 2.0 L/min \\
\hline
Công suất ngưng tụ & 9113 W & 5282 W \\
\hline
Áp suất đỉnh tháp & 1.0 bar & 1.0 bar \\
\hline
\end{tabular}
\caption{So sánh kết quả tính toán giữa chế độ baseline và tối ưu}
\label{tab:summary_results}
\end{table}

\textbf{Kết quả chính:}
\begin{itemize}
    \item Phương pháp vận hành hiện tại (van 100\%) có công suất ngưng tụ dư thừa so với nhu cầu, gây lãng phí 72\% lưu lượng nước làm mát.
    \item Với điều khiển tự động vòng PC-01, van chỉ cần mở 28\% để đạt cân bằng năng lượng, tiết kiệm 72\% lưu lượng nước làm mát.
    \item Việc triển khai điều khiển PWM reboiler sẽ được thực hiện trong giai đoạn Khóa luận Tốt nghiệp để xác định mức tiết kiệm điện năng thực tế.
\end{itemize}
