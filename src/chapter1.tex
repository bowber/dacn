% ============================================================
% Chương 1: Mở đầu
% ============================================================

\newpage
\pagenumbering{arabic}
\setcounter{page}{1}

\section{TỔNG QUÁT ĐỀ TÀI}

% ------------------------------------------------------------
\subsection{Đặt vấn đề}
% ------------------------------------------------------------

Chưng cất là một trong những quá trình phân tách quan trọng nhất trong công nghiệp hóa học, đặc biệt trong sản xuất ethanol. Hệ thống chưng cất ethanol -- nước được sử dụng rộng rãi trong các nhà máy sản xuất nhiên liệu sinh học, đồ uống có cồn và dung môi công nghiệp. Trong vận hành hệ thống chưng cất, việc tối ưu năng suất sản phẩm là một trong những mục tiêu quan trọng nhất.

Trong hệ thống chưng cất quy mô phòng thí nghiệm tại Khoa Kỹ thuật Hóa học, Đại học Bách Khoa TP.HCM, việc ổn định áp suất đỉnh tháp hiện đang được thực hiện bằng cách mở van nước làm mát ở mức tối đa (100\%). Phương pháp này đảm bảo áp suất đỉnh tháp luôn ổn định nhưng gây ra một số vấn đề:

\begin{itemize}
    \item \textbf{Ngưng tụ quá mức:} Khi công suất làm mát quá lớn, hơi ethanol ngưng tụ nhiều hơn mức cần thiết, làm tăng lượng hồi lưu và giảm năng suất sản phẩm đỉnh.
    \item \textbf{Lãng phí nước làm mát:} Lưu lượng nước làm mát sử dụng vượt xa nhu cầu thực tế, gây tốn kém chi phí vận hành.
    \item \textbf{Hạn chế năng suất:} Hệ thống không đạt được năng suất sản phẩm tối ưu do tỷ số hồi lưu cao hơn cần thiết.
\end{itemize}

Vấn đề này đặt ra yêu cầu cần thiết phải nghiên cứu và phát triển một phương pháp điều khiển thông minh hơn, có khả năng điều chỉnh lưu lượng nước làm mát phù hợp với nhu cầu thực tế của quá trình, từ đó tối ưu hóa năng suất sản phẩm.

% ------------------------------------------------------------
\subsection{Mục tiêu nghiên cứu}
% ------------------------------------------------------------

Đồ án này nhằm đạt được các mục tiêu sau:

\begin{enumerate}
    \item Phân tích và đánh giá hiện trạng vận hành của hệ thống chưng cất ethanol -- nước quy mô phòng thí nghiệm.
    \item Nghiên cứu mối quan hệ giữa lưu lượng nước làm mát, áp suất đỉnh tháp và năng suất sản phẩm.
    \item Đề xuất và thiết kế bộ điều khiển tự động cho van nước làm mát nhằm ổn định áp suất đỉnh tháp với năng suất sản phẩm tối ưu.
    \item Tính toán và đánh giá hiệu quả tăng năng suất của phương pháp điều khiển mới so với phương pháp vận hành hiện tại.
\end{enumerate}

% ------------------------------------------------------------
\subsection{Nội dung thực hiện}
% ------------------------------------------------------------

Để đạt được các mục tiêu đề ra, đồ án thực hiện các nội dung sau:

\begin{enumerate}
    \item \textbf{Nghiên cứu lý thuyết:}
    \begin{itemize}
        \item Tìm hiểu cơ sở lý thuyết quá trình chưng cất và đặc tính hệ ethanol -- nước.
        \item Nghiên cứu các phương pháp điều khiển quá trình và tối ưu hóa năng suất.
    \end{itemize}
    
    \item \textbf{Khảo sát hệ thống hiện tại:}
    \begin{itemize}
        \item Phân tích cấu hình thiết bị và sơ đồ công nghệ.
        \item Đánh giá hiện trạng năng suất khi vận hành với van làm mát mở 100\%.
    \end{itemize}
    
    \item \textbf{Thiết kế hệ thống điều khiển:}
    \begin{itemize}
        \item Xây dựng vòng điều khiển áp suất đỉnh tháp thông qua van nước làm mát.
        \item Xác định thông số bộ điều khiển PID phù hợp.
    \end{itemize}
    
    \item \textbf{Thí nghiệm và đánh giá:}
    \begin{itemize}
        \item Vận hành hệ thống với bộ điều khiển mới.
        \item So sánh năng suất sản phẩm trước và sau khi áp dụng giải pháp tối ưu.
    \end{itemize}
\end{enumerate}

% ------------------------------------------------------------
\subsection{Phạm vi nghiên cứu}
% ------------------------------------------------------------

Đồ án được thực hiện trong phạm vi sau:

\begin{itemize}
    \item \textbf{Đối tượng nghiên cứu:} Hệ thống chưng cất ethanol -- nước quy mô phòng thí nghiệm tại phòng B2-105, Khoa Kỹ thuật Hóa học, Đại học Bách Khoa TP.HCM.
    \item \textbf{Biến điều khiển:} Độ mở van nước làm mát bộ ngưng tụ.
    \item \textbf{Biến được điều khiển:} Áp suất đỉnh tháp chưng cất.
    \item \textbf{Mục tiêu tối ưu:} Tăng năng suất sản phẩm đỉnh trong khi giữ nguyên năng lượng tiêu thụ và chất lượng sản phẩm. Lưu lượng sản phẩm được tính toán gián tiếp thông qua lưu lượng hồi lưu (có cảm biến lưu lượng) và tỷ số hồi lưu.
    \item \textbf{Giới hạn:} Không thay đổi cấu hình thiết bị hiện có, chỉ thay đổi chiến lược điều khiển.
\end{itemize}

% ------------------------------------------------------------
\subsection{Sản phẩm đồ án}
% ------------------------------------------------------------

Sản phẩm của đồ án chuyên ngành (đề cương Khóa luận Tốt nghiệp) bao gồm:

\begin{enumerate}
    \item Báo cáo phân tích hiện trạng vận hành của hệ thống chưng cất.
    \item Cơ sở lý thuyết về quá trình chưng cất và điều khiển quá trình.
    \item Thiết kế sơ bộ vòng điều khiển áp suất đỉnh tháp (PC-01) với bộ điều khiển PID.
    \item Kết quả tính toán cân bằng năng lượng và đánh giá tiềm năng tối ưu hóa.
    \item Kế hoạch triển khai chi tiết cho Khóa luận Tốt nghiệp.
\end{enumerate}

% ------------------------------------------------------------
\subsection{Bố cục dự kiến Khóa luận Tốt nghiệp}
% ------------------------------------------------------------

Khóa luận Tốt nghiệp được phát triển từ đề cương này sẽ có bố cục dự kiến như sau:

\begin{enumerate}
    \item \textbf{Chương 1: Tổng quan}
    \begin{itemize}
        \item Giới thiệu đề tài và đặt vấn đề
        \item Mục tiêu và phạm vi nghiên cứu
    \end{itemize}
    
    \item \textbf{Chương 2: Cơ sở lý thuyết}
    \begin{itemize}
        \item Lý thuyết quá trình chưng cất
        \item Tính chất hệ ethanol -- nước
        \item Điều khiển quá trình và bộ điều khiển PID
    \end{itemize}
    
    \item \textbf{Chương 3: Mô hình thí nghiệm}
    \begin{itemize}
        \item Mô tả hệ thống chưng cất thí nghiệm
        \item Thiết bị đo lường và điều khiển
        \item Thiết kế vòng điều khiển áp suất PC-01
    \end{itemize}
    
    \item \textbf{Chương 4: Triển khai và thực nghiệm}
    \begin{itemize}
        \item Lập trình PLC S7-1200 và thiết kế HMI
        \item Xác định thông số PID (phương pháp Ziegler-Nichols)
        \item Thu thập dữ liệu vận hành thực tế
        \item Triển khai điều khiển PWM công suất reboiler
    \end{itemize}
    
    \item \textbf{Chương 5: Kết quả và thảo luận}
    \begin{itemize}
        \item So sánh kết quả thực nghiệm với tính toán lý thuyết
        \item Đánh giá hiệu quả tối ưu hóa năng suất
        \item Phân tích độ nhạy và giới hạn của phương pháp
    \end{itemize}
    
    \item \textbf{Chương 6: Kết luận và kiến nghị}
    \begin{itemize}
        \item Tổng kết kết quả đạt được
        \item Hạn chế và hướng phát triển
    \end{itemize}
\end{enumerate}

% ------------------------------------------------------------
\subsection{Tiến độ thực hiện Khóa luận Tốt nghiệp}
% ------------------------------------------------------------

Sau khi đề cương được phê duyệt, Khóa luận Tốt nghiệp sẽ được triển khai theo kế hoạch sau:

\begin{table}[H]
\centering
\begin{tabular}{|c|l|p{6.5cm}|}
\hline
\textbf{Tuần} & \textbf{Thời gian} & \textbf{Nội dung công việc} \\
\hline
1--2 & 03/02 -- 16/02/2025 & Thiết kế bộ điều khiển PID, lập trình PLC S7-1200 (TIA Portal), thiết kế HMI \\
\hline
3--4 & 17/02 -- 02/03/2025 & Thí nghiệm vòng điều khiển áp suất PC-01, thu thập dữ liệu \\
\hline
5--6 & 03/03 -- 16/03/2025 & Kiểm chứng $\Delta T_{subcool}$, tinh chỉnh thông số PID \\
\hline
7--8 & 17/03 -- 30/03/2025 & Triển khai và thí nghiệm điều khiển PWM reboiler \\
\hline
9 & 31/03 -- 06/04/2025 & Hoàn thiện báo cáo, chuẩn bị bảo vệ đồ án \\
\hline
\end{tabular}
\caption{Kế hoạch thực hiện Khóa luận Tốt nghiệp}
\label{tab:schedule_ch1}
\end{table}

% ------------------------------------------------------------
\subsection{Tóm tắt kết quả tính toán}
% ------------------------------------------------------------

Dựa trên các tính toán cân bằng năng lượng và truyền nhiệt (chi tiết tại Chương 4), đồ án đã xác định được các thông số vận hành tối ưu cho hệ thống:

\begin{table}[H]
\centering
\begin{tabular}{|l|c|c|}
\hline
\textbf{Thông số} & \textbf{Baseline} & \textbf{Tối ưu} \\
\hline
Độ mở van nước làm mát & 100\% & 53\% \\
\hline
Lưu lượng nước làm mát & 7.2 L/min & 3.8 L/min \\
\hline
Công suất ngưng tụ & 9113 W & 5282 W \\
\hline
Áp suất đỉnh tháp & 1.0 bar & 1.0 bar \\
\hline
\end{tabular}
\caption{So sánh kết quả tính toán giữa chế độ baseline và tối ưu}
\label{tab:summary_results}
\end{table}

\textbf{Kết quả chính:}
\begin{itemize}
    \item Phương pháp vận hành hiện tại (van 100\%) có công suất ngưng tụ dư thừa ~3800 W so với nhu cầu, gây lãng phí 47\% lưu lượng nước làm mát.
    \item Với điều khiển tự động, van chỉ cần mở 53\% để đạt cân bằng năng lượng, tiết kiệm 47\% lưu lượng nước làm mát.
    \item Áp suất đỉnh tháp duy trì ổn định tại 1.0 bar trong cả hai chế độ.
\end{itemize}
