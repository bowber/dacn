% ============================================================
% Chương 1: Mở đầu
% ============================================================

\newpage
\pagenumbering{arabic}
\setcounter{page}{1}

\section{TỔNG QUÁT ĐỀ TÀI}

% ------------------------------------------------------------
\subsection{Đặt vấn đề}
% ------------------------------------------------------------

Chưng cất là một trong những quá trình phân tách quan trọng nhất trong công nghiệp hóa học, đặc biệt trong sản xuất ethanol. Hệ thống chưng cất ethanol -- nước được sử dụng rộng rãi trong các nhà máy sản xuất nhiên liệu sinh học, đồ uống có cồn và dung môi công nghiệp. Trong vận hành hệ thống chưng cất, việc tối ưu năng suất sản phẩm là một trong những mục tiêu quan trọng nhất.

Trong hệ thống chưng cất quy mô phòng thí nghiệm tại Khoa Kỹ thuật Hóa học, Đại học Bách Khoa TP.HCM, việc ổn định áp suất đỉnh tháp hiện đang được thực hiện bằng cách mở van nước làm mát ở mức tối đa (100\%). Phương pháp này đảm bảo áp suất đỉnh tháp luôn ổn định nhưng gây ra một số vấn đề:

\begin{itemize}
    \item \textbf{Ngưng tụ quá mức:} Khi công suất làm mát quá lớn, hơi ethanol ngưng tụ nhiều hơn mức cần thiết, làm tăng lượng hồi lưu và giảm năng suất sản phẩm đỉnh.
    \item \textbf{Lãng phí nước làm mát:} Lưu lượng nước làm mát sử dụng vượt xa nhu cầu thực tế, gây tốn kém chi phí vận hành.
    \item \textbf{Hạn chế năng suất:} Hệ thống không đạt được năng suất sản phẩm tối ưu do tỷ số hồi lưu cao hơn cần thiết.
\end{itemize}

Vấn đề này đặt ra yêu cầu cần thiết phải nghiên cứu và phát triển một phương pháp điều khiển thông minh hơn, có khả năng điều chỉnh lưu lượng nước làm mát phù hợp với nhu cầu thực tế của quá trình, từ đó tối ưu hóa năng suất sản phẩm.

% ------------------------------------------------------------
\subsection{Mục tiêu nghiên cứu}
% ------------------------------------------------------------

Đồ án này nhằm đạt được các mục tiêu sau:

\begin{enumerate}
    \item Phân tích và đánh giá hiện trạng vận hành của hệ thống chưng cất ethanol -- nước quy mô phòng thí nghiệm.
    \item Nghiên cứu mối quan hệ giữa lưu lượng nước làm mát, áp suất đỉnh tháp và năng suất sản phẩm.
    \item Thiết kế và triển khai bộ điều khiển tự động cho van nước làm mát nhằm ổn định áp suất đỉnh tháp với năng suất sản phẩm tối ưu.
    \item Đánh giá hiệu quả tăng năng suất của phương pháp điều khiển mới so với phương pháp vận hành hiện tại.
\end{enumerate}

% ------------------------------------------------------------
\subsection{Nội dung thực hiện}
% ------------------------------------------------------------

Để đạt được các mục tiêu đề ra, đồ án thực hiện các nội dung sau:

\begin{enumerate}
    \item \textbf{Nghiên cứu lý thuyết:}
    \begin{itemize}
        \item Tìm hiểu cơ sở lý thuyết quá trình chưng cất và đặc tính hệ ethanol -- nước.
        \item Nghiên cứu các phương pháp điều khiển quá trình và tối ưu hóa năng suất.
    \end{itemize}
    
    \item \textbf{Khảo sát hệ thống hiện tại:}
    \begin{itemize}
        \item Phân tích cấu hình thiết bị và sơ đồ công nghệ.
        \item Đánh giá hiện trạng năng suất khi vận hành với van làm mát mở 100\%.
    \end{itemize}
    
    \item \textbf{Thiết kế hệ thống điều khiển:}
    \begin{itemize}
        \item Xây dựng vòng điều khiển áp suất đỉnh tháp thông qua van nước làm mát.
        \item Xác định thông số bộ điều khiển PID phù hợp.
    \end{itemize}
    
    \item \textbf{Thí nghiệm và đánh giá:}
    \begin{itemize}
        \item Vận hành hệ thống với bộ điều khiển mới.
        \item So sánh năng suất sản phẩm trước và sau khi áp dụng giải pháp tối ưu.
    \end{itemize}
\end{enumerate}

% ------------------------------------------------------------
\subsection{Phạm vi nghiên cứu}
% ------------------------------------------------------------

Đồ án được thực hiện trong phạm vi sau:

\begin{itemize}
    \item \textbf{Đối tượng nghiên cứu:} Hệ thống chưng cất ethanol -- nước quy mô phòng thí nghiệm tại phòng B2-105, Khoa Kỹ thuật Hóa học, Đại học Bách Khoa TP.HCM.
    \item \textbf{Biến điều khiển:} Độ mở van nước làm mát bộ ngưng tụ.
    \item \textbf{Biến được điều khiển:} Áp suất đỉnh tháp chưng cất.
    \item \textbf{Mục tiêu tối ưu:} Tăng năng suất sản phẩm đỉnh trong khi giữ nguyên năng lượng tiêu thụ và chất lượng sản phẩm. Lưu lượng sản phẩm được tính toán gián tiếp thông qua lưu lượng hồi lưu (có lưu lượng kế) và tỷ số hồi lưu.
    \item \textbf{Giới hạn:} Không thay đổi cấu hình thiết bị hiện có, chỉ thay đổi chiến lược điều khiển.
\end{itemize}

% ------------------------------------------------------------
\subsection{Sản phẩm đồ án}
% ------------------------------------------------------------

Sản phẩm của đồ án bao gồm:

\begin{enumerate}
    \item Báo cáo phân tích hiện trạng vận hành của hệ thống chưng cất.
    \item Chương trình điều khiển PLC cho vòng điều khiển áp suất đỉnh tháp.
    \item Giao diện HMI giám sát và vận hành hệ thống.
    \item Kết quả thí nghiệm và đánh giá hiệu quả tăng năng suất.
    \item Báo cáo đồ án chuyên ngành hoàn chỉnh.
\end{enumerate}
