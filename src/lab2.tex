% ============================================================
% Lab 2: Thiết bị điều khiển và lập trình vi điều khiển
% ============================================================

\setcounter{section}{2}
\setcounter{subsection}{0}
\section*{BÀI THÍ NGHIỆM 2: THIẾT BỊ ĐIỀU KHIỂN VÀ LẬP TRÌNH VI ĐIỀU KHIỂN}
\addcontentsline{toc}{section}{Bài thí nghiệm 2: Thiết bị điều khiển và lập trình vi điều khiển}

% ------------------------------------------------------------
\subsection{Tóm tắt}
% ------------------------------------------------------------

Thí nghiệm này nhằm giúp chúng ta hiểu sâu hơn về chức năng của bộ vi điều khiển và cách lập trình các thuật toán điều khiển. Bằng cách thiết lập sơ đồ kết nối các thành phần trong hệ thống điều khiển và thiết kế hệ thống tự động để điều khiển quá trình chưng cất, chúng ta có thể áp dụng kiến thức của mình vào các ứng dụng thực tế. Mục tiêu cuối cùng là xây dựng một hệ thống tự động hoạt động hiệu quả, từ đó giúp tăng cường kiến thức và kỹ năng trong lĩnh vực điều khiển quá trình này.

% ------------------------------------------------------------
\subsection{Cơ sở lý thuyết}
% ------------------------------------------------------------

\subsubsection{Bộ vi điều khiển}

Bộ vi điều khiển (Microcontroller) là một vi mạch tích hợp bao gồm bộ xử lý trung tâm (CPU), bộ nhớ, và các thiết bị ngoại vi I/O trong một chip duy nhất. Vi điều khiển được thiết kế để thực hiện các tác vụ điều khiển cụ thể trong các hệ thống nhúng.

Các thành phần chính của vi điều khiển:
\begin{itemize}
    \item \textbf{CPU:} Thực hiện các lệnh và tính toán.
    \item \textbf{Bộ nhớ:} Lưu trữ chương trình (Flash) và dữ liệu (RAM).
    \item \textbf{I/O:} Giao tiếp với các thiết bị bên ngoài thông qua các chân số (Digital) và tương tự (Analog).
    \item \textbf{Timer/Counter:} Đếm thời gian và sự kiện.
    \item \textbf{Giao tiếp:} UART, SPI, I2C để truyền nhận dữ liệu.
\end{itemize}

\subsubsection{Điều khiển trình tự (Sequence Control)}

Điều khiển trình tự là phương pháp điều khiển trong đó các bước thực hiện được tiến hành theo một thứ tự định sẵn. Mỗi bước chỉ được thực hiện khi bước trước đó hoàn thành. Phương pháp này thường được sử dụng trong các quy trình khởi động và dừng hệ thống công nghiệp.

% ------------------------------------------------------------
\subsection{Mô tả thiết bị thí nghiệm}
% ------------------------------------------------------------

Trong lĩnh vực công nghiệp hoá chất và dầu khí, hệ thống điều khiển đóng vai trò quan trọng trong việc điều chỉnh, giám sát và tự động hóa các quy trình sản xuất. Các chức năng của hệ thống điều khiển bao gồm điều chỉnh, điều khiển trình tự, điều khiển rời rạc và giám sát vận hành.

Trong bài thí nghiệm này, chúng ta sử dụng bộ điều khiển Arduino, một bo mạch vi điều khiển chuẩn hóa, được sử dụng rộng rãi và được hỗ trợ bởi cộng đồng người dùng ở Việt Nam và toàn cầu.

Phần mềm Arduino IDE cùng với công cụ mô phỏng trực tuyến www.tinkercad.com đóng vai trò quan trọng trong việc lập trình và mô phỏng các thuật toán điều khiển.

\subsubsection{Bo mạch Arduino}

Các mạch Arduino của hãng Arduino có nhiều phiên bản khác nhau như Mega, Uno, Micro, Nano v.v. Mặc dù có những điểm khác biệt về kích thước và tính năng, nhưng cấu trúc chung của chúng vẫn tương tự nhau. Dưới đây là mô tả về mạch Nano:

\begin{itemize}
    \item \textbf{Khối Nguồn (POWER):} Mạch Arduino có thể được cấp nguồn từ bên ngoài với điện áp trong khoảng từ 5V đến 9V.
    \item \textbf{Khối Chức Năng Nhận và Xuất Tín Hiệu (INPUT và OUTPUT):} Mạch có các chân tín hiệu (PIN) để nhận và xuất tín hiệu.
    \item \textbf{Khối Chức Năng Truyền Dữ Liệu (Serial):} Mạch hỗ trợ giao tiếp dữ liệu thông qua giao tiếp Serial.
\end{itemize}

\subsubsection{Arduino IDE}

Intergrated Development Environment (IDE) là môi trường phát triển tích hợp được sử dụng để lập trình cho các bo mạch Arduino. IDE này có thể được tải về miễn phí từ trang web: www.arduino.cc và cài đặt trên máy tính.

\textbf{Các lệnh cơ bản:}
\begin{itemize}
    \item \texttt{pinMode(pin, mode)}: Thiết lập chế độ của một chân (OUTPUT hoặc INPUT).
    \item \texttt{digitalWrite(pin, value)}: Xuất tín hiệu điện từ một chân (HIGH hoặc LOW).
    \item \texttt{digitalRead(pin)}: Đọc giá trị điện áp từ một chân (HIGH hoặc LOW).
    \item \texttt{delay(ms)}: Tạm dừng chương trình trong khoảng thời gian xác định (mili giây).
\end{itemize}

% ------------------------------------------------------------
\subsection{Quy trình thực hiện}
% ------------------------------------------------------------

\subsubsection{Chuẩn bị thí nghiệm}
\begin{enumerate}
    \item Khởi động phần mềm Arduino IDE hoặc Tinkercad.
    \item Kết nối bo mạch Arduino với máy tính qua cáp USB.
    \item Kiểm tra kết nối và chọn đúng cổng COM.
    \item Chuẩn bị các linh kiện: LED, điện trở, công tắc.
\end{enumerate}

\subsubsection{Thí nghiệm 3: Mô phỏng vận hành điều khiển hệ thống chưng cất}

Hệ thống chưng cất trong bài thí nghiệm này được thể hiện trên Hình \ref{fig:distillation_system}.

\begin{figure}[H]
\centering
\includegraphics[width=0.85\textwidth]{../assets/distillation_system.png}
\caption{Sơ đồ hệ thống chưng cất}
\label{fig:distillation_system}
\end{figure}

Điều khiển hệ thống chưng cất bao gồm tự động chu trình khởi động (ON). Tiếp theo sau đó là giai đoạn duy trì trạng thái làm việc ổn định – gọi là chu trình ổn định (SS) – được thực hiện bởi các hệ thống điều chỉnh tự động. Cuối cùng là chu trình tự động dừng (OFF) hệ thống thiết bị chưng cất.

Sử dụng công tắc ON/OFF viết chương trình điều khiển tự động 3 chu trình cho hệ thống chưng cất:
\begin{itemize}
    \item Bật nút ON hệ thống khởi động theo quy trình định sẵn, sau đó hệ chuyển sang trạng thái làm việc.
    \item Bật nút OFF hệ thống chuyển trạng thái dừng theo quy trình.
\end{itemize}

\textbf{Chu trình khởi động ON:}

\begin{table}[H]
\centering
\begin{tabular}{|l|l|c|}
\hline
\textbf{Các bước thực hiện} & \textbf{Sự kiện} & \textbf{Thời gian chờ (giây)} \\
\hline
B1 – Khởi động & Nhấn nút ON & -- \\
\hline
B2 – Nhận dung dịch vào nồi đun & Bật bơm 1, Mở van 0 & 30 \\
\hline
B3 – Gia nhiệt nồi đun & Tắt van 0, Mở điện trở R0, Mở van V2 & 60 \\
\hline
B4 – Khởi động dòng nhập liệu & Mở van V1, Bật điện trở R1 & 10 \\
\hline
B5 – Khởi động dòng hoàn lưu & Bật bơm 2, Bật điện trở R1 & 10 \\
\hline
B6 – Tháo sản phẩm đáy & Mở Van V3, Bật bơm 3 & -- \\
\hline
\end{tabular}
\caption{Chu trình khởi động hệ thống chưng cất}
\end{table}

\textbf{Thiết kế chương trình:}
\begin{itemize}
    \item Sử dụng các LED để mô phỏng trạng thái của bơm, van và điện trở.
    \item Công tắc nối vào chân D2 sử dụng điện trở kéo về GND.
    \item Sử dụng \texttt{digitalWrite()} để bật/tắt các thiết bị theo trình tự.
    \item Sử dụng \texttt{delay()} để tạo thời gian chờ giữa các bước.
\end{itemize}

% ------------------------------------------------------------
\subsection{Kết quả và bàn luận}
% ------------------------------------------------------------

\subsubsection{Kết quả thí nghiệm}

Với các bước thực hiện thiết kế chương trình trên, cùng với đoạn code hoàn chỉnh và chạy thực tế trên mô hình tại phòng thí nghiệm. Ta thấy đoạn code đưa ra kết quả hiệu quả và có kết quả chính xác đáp ứng theo ý đồ điều khiển của chúng ta với các chu trình ON, OFF định sẵn.

Hệ thống hoạt động đúng theo trình tự:
\begin{enumerate}
    \item Khi nhấn nút ON, hệ thống bắt đầu chu trình khởi động.
    \item Các thiết bị được bật/tắt theo đúng thứ tự và thời gian quy định.
    \item Khi nhấn nút OFF, hệ thống dừng theo quy trình an toàn.
\end{enumerate}

\subsubsection{Bàn luận}

Qua thí nghiệm, có thể thấy rằng vi điều khiển Arduino là công cụ hiệu quả để triển khai các thuật toán điều khiển trình tự. Việc sử dụng các lệnh đơn giản như \texttt{digitalWrite()}, \texttt{delay()} cho phép điều khiển linh hoạt các thiết bị chấp hành theo trình tự định sẵn.

% ------------------------------------------------------------
\subsection{Kết luận và khuyến nghị}
% ------------------------------------------------------------

Qua quá trình tìm hiểu và thực hiện thí nghiệm, nhóm rút ra được nhiều nội dung ý nghĩa:

\begin{itemize}
    \item \textbf{Hiểu được chức năng bộ vi điều khiển:} Thí nghiệm giúp người tham gia hiểu rõ về chức năng và vai trò của bộ vi điều khiển trong hệ thống điều khiển tự động.
    \item Biết lập trình các thuật toán điều khiển trình tự cho bộ điều khiển.
    \item Thiết lập sơ đồ kết nối các thành phần của hệ thống điều khiển.
    \item Thiết kế hệ thống điều khiển tự động để khởi động, dừng cho quá trình chưng cất.
\end{itemize}

\textbf{Khuyến nghị:} Cần nghiên cứu thêm về các phương pháp điều khiển nâng cao và xử lý các tình huống lỗi trong quá trình vận hành để cải thiện độ tin cậy của hệ thống.

% ------------------------------------------------------------
\subsection{Trả lời câu hỏi kiểm tra}
% ------------------------------------------------------------

\textbf{Câu 1: Mô tả cấu trúc và chức năng của bộ vi điều khiển Arduino?}

Arduino là bo mạch vi điều khiển bao gồm: CPU (xử lý lệnh), bộ nhớ Flash (lưu chương trình), RAM (lưu dữ liệu), các chân I/O số và analog, giao tiếp Serial/SPI/I2C, và các timer/counter.

\textbf{Câu 2: Điều khiển trình tự là gì? Ứng dụng trong công nghiệp?}

Điều khiển trình tự là phương pháp điều khiển trong đó các bước được thực hiện theo thứ tự định sẵn. Ứng dụng phổ biến trong công nghiệp bao gồm: khởi động/dừng hệ thống, quy trình sản xuất theo mẻ, điều khiển dây chuyền lắp ráp.

\textbf{Câu 3: Giải thích quy trình khởi động hệ thống chưng cất?}

Quy trình khởi động hệ thống chưng cất bao gồm các bước tuần tự: nạp nguyên liệu vào nồi đun, gia nhiệt nồi đun, khởi động dòng nhập liệu, khởi động dòng hoàn lưu, và cuối cùng là tháo sản phẩm đáy. Mỗi bước có thời gian chờ xác định để đảm bảo hệ thống hoạt động ổn định.

% ------------------------------------------------------------
\subsection{Tài liệu tham khảo}
% ------------------------------------------------------------

\begin{enumerate}
    \item Điều khiển Quá trình Công nghệ Hóa học - Cơ sở điều khiển Quá trình – Quyển 1.
    \item Điều khiển Quá trình Công nghệ Hóa học – Hướng dẫn thí nghiệm, Thực hành cơ sở Điều khiển – Quyển 2.
    \item Arduino Official Documentation, https://www.arduino.cc/reference/en/
\end{enumerate}

\newpage
