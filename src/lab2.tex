% ============================================================
% Lab 2: Thiết bị điều khiển và lập trình vi điều khiển
% ============================================================

\setcounter{section}{2}
\setcounter{subsection}{0}
\section*{BÀI THÍ NGHIỆM 2: THIẾT BỊ ĐIỀU KHIỂN VÀ LẬP TRÌNH VI ĐIỀU KHIỂN}
\addcontentsline{toc}{section}{Bài thí nghiệm 2: Thiết bị điều khiển và lập trình vi điều khiển}

% ------------------------------------------------------------
\subsection{Tóm tắt}
% ------------------------------------------------------------

Bài thí nghiệm này tập trung vào việc mô phỏng vận hành điều khiển hệ thống chưng cất sử dụng vi điều khiển Arduino. Mục tiêu là thiết kế hệ thống điều khiển tự động để thực hiện 3 chu trình: khởi động (ON), làm việc ổn định (SS), và dừng (OFF) cho quá trình chưng cất. Thông qua thí nghiệm, chúng ta hiểu rõ hơn về điều khiển trình tự và cách áp dụng vào các hệ thống công nghiệp thực tế.

% ------------------------------------------------------------
\subsection{Cơ sở lý thuyết}
% ------------------------------------------------------------

\subsubsection{Điều khiển trình tự (Sequence Control)}

Điều khiển trình tự là phương pháp điều khiển trong đó các bước thực hiện được tiến hành theo một thứ tự định sẵn. Mỗi bước chỉ được thực hiện khi bước trước đó hoàn thành. Phương pháp này thường được sử dụng trong các quy trình khởi động và dừng hệ thống công nghiệp.

Trong hệ thống chưng cất, điều khiển trình tự đảm bảo các thiết bị (bơm, van, điện trở gia nhiệt) được bật/tắt theo đúng thứ tự và thời gian quy định, đảm bảo an toàn và hiệu quả vận hành.

% ------------------------------------------------------------
\subsection{Mô tả thiết bị thí nghiệm}
% ------------------------------------------------------------

Hệ thống chưng cất trong bài thí nghiệm này được thể hiện trên Hình \ref{fig:distillation_system}.

\begin{figure}[H]
\centering
\includegraphics[width=0.85\textwidth]{../assets/distillation_system.png}
\caption{Sơ đồ hệ thống chưng cất}
\label{fig:distillation_system}
\end{figure}

Các thành phần chính của hệ thống bao gồm:
\begin{itemize}
    \item \textbf{Bơm 1, 2, 3:} Cung cấp và vận chuyển dung dịch trong hệ thống.
    \item \textbf{Van V0, V1, V2, V3, V4:} Điều khiển dòng chảy của các dòng vật chất.
    \item \textbf{Điện trở R0, R1, R2:} Cung cấp nhiệt cho quá trình chưng cất.
    \item \textbf{Thiết bị ngưng tụ:} Làm mát và ngưng tụ hơi sản phẩm.
\end{itemize}

Bộ điều khiển Arduino được sử dụng để lập trình điều khiển trình tự, với các LED mô phỏng trạng thái của bơm, van và điện trở. Phần mềm Arduino IDE và công cụ mô phỏng Tinkercad hỗ trợ việc lập trình và kiểm tra thuật toán.

% ------------------------------------------------------------
\subsection{Quy trình thực hiện}
% ------------------------------------------------------------

Điều khiển hệ thống chưng cất bao gồm 3 chu trình:
\begin{itemize}
    \item \textbf{Chu trình khởi động (ON):} Tự động khởi động hệ thống theo quy trình định sẵn.
    \item \textbf{Chu trình làm việc (SS):} Duy trì trạng thái làm việc ổn định.
    \item \textbf{Chu trình dừng (OFF):} Tự động dừng hệ thống an toàn.
\end{itemize}

Sử dụng công tắc ON/OFF viết chương trình điều khiển tự động:
\begin{itemize}
    \item Bật nút ON: hệ thống khởi động theo quy trình, sau đó chuyển sang trạng thái làm việc.
    \item Bật nút OFF: hệ thống chuyển sang trạng thái dừng theo quy trình.
\end{itemize}

\subsubsection{Chu trình khởi động ON}

\begin{table}[H]
\centering
\begin{tabular}{|l|l|c|}
\hline
\textbf{Các bước thực hiện} & \textbf{Sự kiện} & \textbf{Thời gian chờ (giây)} \\
\hline
B1 – Khởi động & Nhấn nút ON & -- \\
\hline
B2 – Nhận dung dịch vào nồi đun & Bật bơm 1, Mở van V0 & 30 \\
\hline
B3 – Gia nhiệt nồi đun & Tắt van V0, Bật điện trở R0, Mở van V4 & 60 \\
\hline
B4 – Khởi động dòng nhập liệu & Mở van V1, Bật điện trở R1 & 10 \\
\hline
B5 – Khởi động dòng hoàn lưu & Mở van V2, Bật bơm 2 & 10 \\
\hline
B6 – Tháo sản phẩm đáy & Mở van V3, Bật bơm 3 & -- \\
\hline
\end{tabular}
\caption{Chu trình khởi động hệ thống chưng cất}
\end{table}

\subsubsection{Trạng thái làm việc SS}

Tất cả các thiết bị ở trạng thái làm việc (ON), riêng van V0 đóng. Hệ thống duy trì trạng thái này cho đến khi nhận lệnh dừng.

\subsubsection{Chu trình dừng OFF}

\begin{table}[H]
\centering
\begin{tabular}{|l|l|c|}
\hline
\textbf{Các bước thực hiện} & \textbf{Sự kiện} & \textbf{Thời gian chờ (giây)} \\
\hline
B1 – Dừng & Nhấn nút OFF & -- \\
\hline
B2 – Tắt tất cả nguồn nhiệt & Tắt điện trở R0, R1 và R2 & 60 \\
\hline
B3 – Ngưng dòng nhập liệu và hồi lưu & Tắt bơm 1, 2, 3; Đóng van V1, V2, V3 & 60 \\
\hline
B4 – Ngưng dòng nước lạnh & Đóng van V4 & -- \\
\hline
\end{tabular}
\caption{Chu trình dừng hệ thống chưng cất}
\end{table}

\subsubsection{Thiết kế chương trình}

Sử dụng các LED để mô phỏng trạng thái của bơm, van và điện trở. Công tắc nối vào chân D2 sử dụng điện trở kéo về GND.

\textbf{Sơ đồ kết nối chân Arduino:}

\begin{table}[H]
\centering
\begin{tabular}{|c|l||c|l|}
\hline
\textbf{Chân} & \textbf{Thiết bị} & \textbf{Chân} & \textbf{Thiết bị} \\
\hline
D2 & Công tắc (INPUT) & D9 & Điện trở R2 \\
\hline
D3 & Bơm 3 & D10 & Điện trở R1 \\
\hline
D4 & Bơm 2 & D11 & Điện trở R0 \\
\hline
D5 & Bơm 1 & D12 & Van V3 \\
\hline
D6 & Van V2 & D13 & Van V4 \\
\hline
D7 & Van V1 & & \\
\hline
D8 & Van V0 & & \\
\hline
\end{tabular}
\caption{Sơ đồ kết nối chân Arduino}
\end{table}

\textbf{Đoạn code chương trình:}

\begin{lstlisting}
int congtac = 2;
int pump3 = 3, pump2 = 4, pump1 = 5;
int van2 = 6, van1 = 7, van0 = 8;
int re2 = 9, re1 = 10, re0 = 11;
int van3 = 12, van4 = 13;
int i = 0;

void chutrinhON() {
  // B2: Nap dung dich vao noi dun (30s)
  digitalWrite(pump1, 1); digitalWrite(van0, 1); delay(30000);
  // B3: Gia nhiet noi dun, cap nuoc lam mat (60s)
  digitalWrite(van0, 0); digitalWrite(re0, 1); digitalWrite(van4, 1); delay(60000);
  // B4: Khoi dong dong nhap lieu (10s)
  digitalWrite(van1, 1); digitalWrite(re1, 1); delay(10000);
  // B5: Khoi dong dong hoan luu (10s)
  digitalWrite(van2, 1); digitalWrite(pump2, 1); delay(10000);
  // B6: Thao san pham day
  digitalWrite(van3, 1); digitalWrite(pump3, 1);
}

void chutrinhOFF() {
  // B2: Tat tat ca nguon nhiet (60s)
  digitalWrite(re0, 0); digitalWrite(re1, 0); digitalWrite(re2, 0); delay(60000);
  // B3: Ngung dong nhap lieu va hoi luu (60s)
  digitalWrite(pump1, 0); digitalWrite(pump2, 0); digitalWrite(pump3, 0);
  digitalWrite(van1, 0); digitalWrite(van2, 0); digitalWrite(van3, 0);
  delay(60000);
  // B4: Ngung dong nuoc lanh
  digitalWrite(van4, 0);
}

void setup() {
  pinMode(congtac, INPUT);
  pinMode(pump1, OUTPUT); pinMode(pump2, OUTPUT); pinMode(pump3, OUTPUT);
  pinMode(van0, OUTPUT); pinMode(van1, OUTPUT); pinMode(van2, OUTPUT);
  pinMode(van3, OUTPUT); pinMode(van4, OUTPUT);
  pinMode(re0, OUTPUT); pinMode(re1, OUTPUT); pinMode(re2, OUTPUT);
}

void loop() {
  if (digitalRead(congtac) == 1) {
    if (i == 0) {
      chutrinhON();
      i = i + 1;
    }
  }
  if (digitalRead(congtac) == 0) {
    if (i == 1) {
      chutrinhOFF();
      i = i - 1;
    }
  }
}
\end{lstlisting}

% ------------------------------------------------------------
\subsection{Kết quả và bàn luận}
% ------------------------------------------------------------

\subsubsection{Kết quả thí nghiệm}

Chương trình điều khiển đã được thiết kế và chạy thực tế trên mô hình tại phòng thí nghiệm. Kết quả cho thấy hệ thống hoạt động đúng theo trình tự:

\begin{enumerate}
    \item Khi nhấn nút ON, hệ thống bắt đầu chu trình khởi động với các bước tuần tự từ B1 đến B6.
    \item Sau khi hoàn thành chu trình ON, hệ thống chuyển sang trạng thái làm việc SS và duy trì ổn định.
    \item Khi nhấn nút OFF, hệ thống thực hiện chu trình dừng an toàn với các bước từ B1 đến B4.
\end{enumerate}

\subsubsection{Bàn luận}

Qua thí nghiệm, có thể thấy rằng điều khiển trình tự là phương pháp hiệu quả để vận hành các hệ thống công nghiệp phức tạp như hệ thống chưng cất. Việc tuân thủ đúng trình tự và thời gian chờ giữa các bước đảm bảo:
\begin{itemize}
    \item An toàn cho thiết bị và người vận hành.
    \item Hiệu quả của quá trình chưng cất.
    \item Dễ dàng giám sát và phát hiện sự cố.
\end{itemize}

% ------------------------------------------------------------
\subsection{Kết luận và khuyến nghị}
% ------------------------------------------------------------

Qua quá trình thực hiện thí nghiệm, nhóm rút ra được các nội dung sau:

\begin{itemize}
    \item Hiểu được nguyên lý điều khiển trình tự và ứng dụng trong vận hành hệ thống chưng cất.
    \item Biết thiết kế chu trình khởi động (ON), làm việc (SS), và dừng (OFF) cho hệ thống công nghiệp.
    \item Nắm vững cách sử dụng vi điều khiển Arduino để triển khai thuật toán điều khiển trình tự.
\end{itemize}

\textbf{Khuyến nghị:} Cần nghiên cứu thêm về xử lý các tình huống lỗi và interlock (khóa liên động) để cải thiện độ an toàn và tin cậy của hệ thống.

% ------------------------------------------------------------
\subsection{Trả lời câu hỏi kiểm tra}
% ------------------------------------------------------------

\textbf{Câu 1: Điều khiển trình tự là gì? Ứng dụng trong công nghiệp?}

Điều khiển trình tự là phương pháp điều khiển trong đó các bước được thực hiện theo thứ tự định sẵn, mỗi bước chỉ thực hiện khi bước trước hoàn thành. Ứng dụng phổ biến: khởi động/dừng hệ thống, quy trình sản xuất theo mẻ, điều khiển dây chuyền lắp ráp.

\textbf{Câu 2: Giải thích quy trình khởi động hệ thống chưng cất?}

Quy trình khởi động gồm các bước tuần tự: (1) Nạp nguyên liệu vào nồi đun qua bơm 1 và van V0, (2) Gia nhiệt nồi đun bằng điện trở R0 và cấp nước làm mát qua van V4, (3) Khởi động dòng nhập liệu qua van V1, (4) Khởi động dòng hoàn lưu qua van V2 và bơm 2, (5) Tháo sản phẩm đáy qua van V3 và bơm 3.

\textbf{Câu 3: Tại sao cần có thời gian chờ giữa các bước trong chu trình?}

Thời gian chờ đảm bảo: (1) Thiết bị có đủ thời gian ổn định trước khi bước tiếp theo thực hiện, (2) Tránh quá tải cho hệ thống điện, (3) Đảm bảo các điều kiện công nghệ được đáp ứng (ví dụ: nồi đun đủ nóng trước khi nhập liệu).

% ------------------------------------------------------------
\subsection{Tài liệu tham khảo}
% ------------------------------------------------------------

\begin{enumerate}
    \item Điều khiển Quá trình Công nghệ Hóa học - Cơ sở điều khiển Quá trình – Quyển 1.
    \item Điều khiển Quá trình Công nghệ Hóa học – Hướng dẫn thí nghiệm, Thực hành cơ sở Điều khiển – Quyển 2.
\end{enumerate}

\newpage
