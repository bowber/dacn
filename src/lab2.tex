% ============================================================
% Lab 2: Điều khiển ON/OFF
% ============================================================

\setcounter{section}{2}
\setcounter{subsection}{0}
\section*{BÀI THÍ NGHIỆM 2: BỘ ĐIỀU KHIỂN ON/OFF}
\addcontentsline{toc}{section}{Bài thí nghiệm 2: Bộ điều khiển ON/OFF}

% ------------------------------------------------------------
\subsection{Mục tiêu}
% ------------------------------------------------------------

\begin{itemize}
    \item Hiểu nguyên lý hoạt động của bộ điều khiển ON/OFF.
    \item Thiết kế và lập trình hệ thống điều khiển nhiệt độ ON/OFF trên PLC S7-1200.
    \item Khảo sát đặc tính dao động của hệ thống điều khiển ON/OFF.
\end{itemize}

% ------------------------------------------------------------
\subsection{Cơ sở lý thuyết}
% ------------------------------------------------------------

\subsubsection{Bộ điều khiển ON/OFF}

Bộ điều khiển ON/OFF (còn gọi là điều khiển bang-bang) là loại điều khiển đơn giản nhất, trong đó tín hiệu điều khiển chỉ có hai trạng thái: bật (ON) hoặc tắt (OFF).

\begin{figure}[H]
\centering
\begin{tikzpicture}
    % ON-OFF characteristic with deadband
    \draw[->] (-2,0) -- (3,0) node[right] {$e$};
    \draw[->] (0,-1.5) -- (0,2) node[above] {$u$};
    
    % Deadband loop
    \draw[thick, blue, ->] (-1.5,-1) -- (-0.5,-1) -- (-0.5,1) -- (1.5,1);
    \draw[thick, red, ->] (1.5,1) -- (0.5,1) -- (0.5,-1) -- (-1.5,-1);
    
    % Labels
    \node[font=\small] at (2.5,1) {$u_{max}$};
    \node[font=\small] at (2.5,-1) {$u_{min}$};
    \node[font=\small] at (-0.5,-1.5) {$-h$};
    \node[font=\small] at (0.5,-1.5) {$+h$};
\end{tikzpicture}
\caption{Đặc tính bộ điều khiển ON/OFF với deadband (dải chết)}
\label{fig:onoff_deadband}
\end{figure}

Phương trình mô tả:
\begin{equation}
u = \begin{cases}
u_{max} & \text{nếu } e > +h \\
u_{min} & \text{nếu } e < -h \\
\text{giữ nguyên} & \text{nếu } -h \leq e \leq +h
\end{cases}
\end{equation}

Trong đó $h$ là độ rộng dải chết (deadband), giúp giảm tần số đóng ngắt của thiết bị chấp hành.

\subsubsection{Đặc điểm của điều khiển ON/OFF}

\begin{itemize}
    \item \textbf{Ưu điểm:} Đơn giản, dễ triển khai, chi phí thấp.
    \item \textbf{Nhược điểm:} Biến điều khiển dao động quanh setpoint, không đạt được điều khiển chính xác.
    \item \textbf{Ứng dụng:} Điều khiển nhiệt độ lò nung, bình nước nóng, máy điều hòa...
\end{itemize}

% ------------------------------------------------------------
\subsection{Thiết bị thí nghiệm}
% ------------------------------------------------------------

\begin{itemize}
    \item PLC Siemens S7-1200 CPU 1213C
    \item Module analog input/output
    \item Cảm biến nhiệt độ Pt100
    \item Bộ gia nhiệt (Heater)
    \item Phần mềm TIA Portal V16
\end{itemize}

% ------------------------------------------------------------
\subsection{Quy trình thực hiện}
% ------------------------------------------------------------

\subsubsection{Cấu hình phần cứng}
\begin{enumerate}
    \item Tạo project mới trong TIA Portal V16.
    \item Thêm PLC S7-1200 và cấu hình module analog.
    \item Kết nối cảm biến Pt100 với analog input.
    \item Kết nối bộ gia nhiệt với digital output.
\end{enumerate}

\subsubsection{Lập trình điều khiển ON/OFF}
\begin{enumerate}
    \item Đọc giá trị nhiệt độ từ cảm biến.
    \item So sánh với setpoint và tính sai lệch $e = SP - PV$.
    \item Áp dụng logic ON/OFF với deadband:
    \begin{itemize}
        \item Nếu $PV < SP - h$: Bật heater (ON)
        \item Nếu $PV > SP + h$: Tắt heater (OFF)
    \end{itemize}
    \item Quan sát đáp ứng của hệ thống.
\end{enumerate}

% ------------------------------------------------------------
\subsection{Kết quả thí nghiệm}
% ------------------------------------------------------------

\subsubsection{Giao diện HMI}

\begin{figure}[H]
\centering
\includegraphics[width=0.9\textwidth]{../assets/HMI_TN2.jpg}
\caption{Giao diện HMI thí nghiệm điều khiển ON/OFF}
\label{fig:hmi_tn2}
\end{figure}

\subsubsection{Đáp ứng hệ thống điều khiển ON/OFF}

Thí nghiệm điều khiển ON/OFF với các thông số:
\begin{itemize}
    \item Setpoint (SV) = $40^\circ$C
    \item Deadband (dải chết) = $\pm 1^\circ$C (heater ON khi PV $< 39^\circ$C, OFF khi PV $> 41^\circ$C)
    \item Nhiệt độ ban đầu: $\approx 29^\circ$C
\end{itemize}

\begin{figure}[H]
\centering
\begin{tikzpicture}
\begin{axis}[
    width=0.95\textwidth,
    height=8cm,
    xlabel={Thời gian (s)},
    ylabel={Nhiệt độ ($^\circ$C)},
    grid=both,
    grid style={line width=.1pt, draw=gray!30},
    major grid style={line width=.2pt,draw=gray!50},
    xmin=0, xmax=800,
    ymin=25, ymax=50,
    legend pos=north east,
]
% PV line
\addplot[blue, thick] table[x=time_s, y=PV, col sep=comma] {../data/TN2_processed.csv};
\addlegendentry{PV (Nhiệt độ đo được)}

% SV line
\addplot[red, dashed, thick] table[x=time_s, y=SV, col sep=comma] {../data/TN2_processed.csv};
\addlegendentry{SV = $40^\circ$C}

% Deadband boundaries
\addplot[orange, dotted, thick, domain=0:800] {41};
\addlegendentry{Ngưỡng OFF ($41^\circ$C)}
\addplot[green!60!black, dotted, thick, domain=0:800] {39};
\addlegendentry{Ngưỡng ON ($39^\circ$C)}

% Annotations
\draw[gray, dashed] (axis cs:131,25) -- (axis cs:131,50);
\node[font=\tiny] at (axis cs:131,27) {Heater OFF};

\draw[gray, dashed] (axis cs:176,25) -- (axis cs:176,50);
\node[font=\tiny] at (axis cs:176,47) {Peak};

\draw[gray, dashed] (axis cs:276,25) -- (axis cs:276,50);
\node[font=\tiny] at (axis cs:276,27) {Heater ON};

\end{axis}
\end{tikzpicture}
\caption{Đáp ứng điều khiển ON/OFF với SV = $40^\circ$C, deadband = $\pm 1^\circ$C}
\label{fig:onoff_response_tn2}
\end{figure}

\subsubsection{Phân tích đáp ứng}

\begin{itemize}
    \item \textbf{Giai đoạn gia nhiệt (0-131s):} Heater ON liên tục, nhiệt độ tăng từ $29^\circ$C đến $41^\circ$C với tốc độ $\approx 5.5^\circ$C/phút
    \item \textbf{Quá điều chỉnh (131-176s):} Heater OFF nhưng nhiệt độ tiếp tục tăng do quán tính nhiệt, đạt đỉnh $45.5^\circ$C (vượt SV $5.5^\circ$C, tương đương 50\% so với biên độ bước $11^\circ$C)
    \item \textbf{Giai đoạn làm mát (176-793s):} Hệ thống nguội tự nhiên, tốc độ $\approx 1.6^\circ$C/phút
    \item \textbf{Kích hoạt lại (tại $\approx$276s):} PV giảm xuống dưới $39^\circ$C, heater bật lại
\end{itemize}

\textbf{Các thông số quan trọng từ thí nghiệm:}
\begin{itemize}
    \item Chu kỳ dao động (Ultimate Period): $T_u \approx 250$s
    \item Biên độ dao động: $\approx 6.5^\circ$C (so với SV)
    \item Độ quá điều chỉnh: $\approx 50\%$ ($(45.5-40)/(40-29) \times 100$, tính theo biên độ bước)
    \item Thời gian đạt SV lần đầu: $\approx 126$s
\end{itemize}

% ------------------------------------------------------------
\subsection{Bàn luận}
% ------------------------------------------------------------

\subsubsection{Đặc tính dao động của điều khiển ON/OFF}

Từ kết quả thí nghiệm (Hình \ref{fig:onoff_response_tn2}), nhận thấy:
\begin{itemize}
    \item Hệ thống không thể duy trì nhiệt độ chính xác tại setpoint mà luôn dao động quanh nó
    \item Biên độ dao động ($\approx 6.5^\circ$C) lớn hơn nhiều so với deadband ($\pm 1^\circ$C) do quán tính nhiệt của hệ thống
    \item Độ quá điều chỉnh $\approx 50\%$ (so với biên độ bước $11^\circ$C) là do heater không tắt ngay khi đạt SV mà phải đợi vượt ngưỡng $41^\circ$C, kết hợp với quán tính nhiệt lớn của hệ thống
\end{itemize}

\subsubsection{Vai trò của deadband}

Về mặt lý thuyết, deadband ảnh hưởng đến chất lượng điều khiển:
\begin{itemize}
    \item \textbf{Deadband nhỏ:} Giảm biên độ dao động nhưng tăng tần số đóng ngắt, gây hao mòn thiết bị
    \item \textbf{Deadband lớn:} Giảm tần số đóng ngắt nhưng tăng biên độ dao động, giảm độ chính xác
    \item \textbf{Giá trị tối ưu:} Cần cân bằng giữa độ chính xác và tuổi thọ thiết bị
\end{itemize}

\subsubsection{Ứng dụng trong chỉnh định bộ điều khiển PID}

Dữ liệu từ điều khiển ON/OFF có thể dùng để tính thông số PID theo phương pháp relay (Ziegler-Nichols Ultimate Gain):
\begin{itemize}
    \item Chu kỳ dao động tới hạn: $T_u = 250$s
    \item Ultimate gain: $K_u = \frac{4d}{\pi a}$ với $d$ = độ rộng relay (50\%), $a$ = biên độ dao động
\end{itemize}

% ------------------------------------------------------------
\subsection{Kết luận}
% ------------------------------------------------------------

Qua thí nghiệm điều khiển ON/OFF hệ thống gia nhiệt, rút ra các kết luận:

\begin{enumerate}
    \item \textbf{Đặc tính cơ bản:} Điều khiển ON/OFF đơn giản, dễ triển khai nhưng không thể duy trì nhiệt độ ổn định do bản chất phi tuyến của bộ điều khiển.
    
    \item \textbf{Dao động và quá điều chỉnh:} Với deadband $\pm 1^\circ$C, hệ thống dao động với biên độ $\approx 6.5^\circ$C và chu kỳ $\approx 250$s. Độ quá điều chỉnh $\approx 50\%$ (so với biên độ bước) do quán tính nhiệt.
    
    \item \textbf{Vai trò của deadband:} Dải chết giúp giảm tần số đóng ngắt, bảo vệ thiết bị chấp hành, nhưng làm tăng biên độ dao động.
    
    \item \textbf{Phạm vi ứng dụng:} Phù hợp cho các hệ thống không yêu cầu độ chính xác cao như bình nước nóng, lò sấy công nghiệp, máy điều hòa dân dụng.
\end{enumerate}

% ------------------------------------------------------------
\subsection{Câu hỏi kiểm tra}
% ------------------------------------------------------------

\textbf{Câu 1: Tại sao cần deadband (dải chết) trong điều khiển ON/OFF?}

Deadband (dải chết) cần thiết vì:
\begin{itemize}
    \item \textbf{Tránh dao động tần số cao:} Không có deadband, relay sẽ đóng ngắt liên tục khi PV ở gần SP do nhiễu đo lường
    \item \textbf{Bảo vệ thiết bị:} Giảm số lần đóng ngắt, kéo dài tuổi thọ relay, contactor và heater
    \item \textbf{Tiết kiệm năng lượng:} Tránh khởi động heater quá thường xuyên, giảm dòng khởi động
    \item \textbf{Ổn định hệ thống:} Tạo vùng ổn định cho biến điều khiển
\end{itemize}

\textbf{Câu 2: Ưu nhược điểm của điều khiển ON/OFF so với PID?}

\textbf{Ưu điểm ON/OFF:}
\begin{itemize}
    \item Đơn giản, dễ triển khai, không cần chỉnh định thông số
    \item Chi phí thấp, không cần bộ điều khiển phức tạp
    \item Phù hợp với các hệ thống có quán tính lớn, yêu cầu độ chính xác không cao
\end{itemize}

\textbf{Nhược điểm ON/OFF:}
\begin{itemize}
    \item Biến điều khiển dao động, không đạt được giá trị chính xác
    \item Gây mài mòn cơ cấu chấp hành do đóng ngắt liên tục
    \item Không phù hợp với các quá trình yêu cầu điều khiển chính xác
\end{itemize}

\textbf{Câu 3: Ứng dụng thực tế của điều khiển ON/OFF?}

\begin{itemize}
    \item \textbf{Điều hòa không khí:} Bật/tắt máy nén dựa trên nhiệt độ phòng
    \item \textbf{Bình nước nóng:} Bật/tắt điện trở khi nhiệt độ nước thấp/cao
    \item \textbf{Tủ lạnh:} Điều khiển máy nén để duy trì nhiệt độ
    \item \textbf{Lò sấy công nghiệp:} Điều khiển heater với dung sai nhiệt độ cho phép
    \item \textbf{Bể ổn nhiệt:} Duy trì nhiệt độ trong dải cho phép
    \item \textbf{Hệ thống bơm nước:} Bật/tắt bơm theo mức nước trong bể
\end{itemize}

% ------------------------------------------------------------
\subsection{Tài liệu tham khảo}
% ------------------------------------------------------------

\begin{enumerate}[label={[\arabic*]}]
    \item B. N. Pha, \textit{Thiết bị Đo lường và Điều khiển}. Trường Đại học Bách khoa, ĐHQG-HCM.
    \item Siemens, \textit{S7-1200 Programmable Controller System Manual}. Siemens AG, 2020.
\end{enumerate}

\newpage
