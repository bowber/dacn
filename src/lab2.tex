% ============================================================
% Lab 2: Điều khiển ON/OFF
% ============================================================

\setcounter{section}{2}
\setcounter{subsection}{0}
\section*{BÀI THÍ NGHIỆM 2: ĐIỀU KHIỂN ON/OFF}
\addcontentsline{toc}{section}{Bài thí nghiệm 2: Điều khiển ON/OFF}

% ------------------------------------------------------------
\subsection{Mục tiêu}
% ------------------------------------------------------------

\begin{itemize}
    \item Hiểu nguyên lý hoạt động của bộ điều khiển ON/OFF.
    \item Thiết kế và lập trình hệ thống điều khiển nhiệt độ ON/OFF trên PLC S7-1200.
    \item Khảo sát đặc tính dao động và ảnh hưởng của hysteresis.
\end{itemize}

% ------------------------------------------------------------
\subsection{Cơ sở lý thuyết}
% ------------------------------------------------------------

\subsubsection{Bộ điều khiển ON/OFF}

Bộ điều khiển ON/OFF (còn gọi là điều khiển bang-bang) là loại điều khiển đơn giản nhất, trong đó tín hiệu điều khiển chỉ có hai trạng thái: bật (ON) hoặc tắt (OFF).

\begin{figure}[H]
\centering
\begin{tikzpicture}
    % ON-OFF characteristic with hysteresis
    \draw[->] (-2,0) -- (3,0) node[right] {$e$};
    \draw[->] (0,-1.5) -- (0,2) node[above] {$u$};
    
    % Hysteresis loop
    \draw[thick, blue, ->] (-1.5,-1) -- (-0.5,-1) -- (-0.5,1) -- (1.5,1);
    \draw[thick, red, ->] (1.5,1) -- (0.5,1) -- (0.5,-1) -- (-1.5,-1);
    
    % Labels
    \node[font=\small] at (2.5,1) {$u_{max}$};
    \node[font=\small] at (2.5,-1) {$u_{min}$};
    \node[font=\small] at (-0.5,-1.5) {$-h$};
    \node[font=\small] at (0.5,-1.5) {$+h$};
\end{tikzpicture}
\caption{Đặc tính bộ điều khiển ON/OFF với hysteresis}
\label{fig:onoff_hysteresis}
\end{figure}

Phương trình mô tả:
\begin{equation}
u = \begin{cases}
u_{max} & \text{nếu } e > +h \\
u_{min} & \text{nếu } e < -h \\
\text{giữ nguyên} & \text{nếu } -h \leq e \leq +h
\end{cases}
\end{equation}

Trong đó $h$ là độ trễ (hysteresis), giúp giảm tần số đóng ngắt của thiết bị chấp hành.

\subsubsection{Đặc điểm của điều khiển ON/OFF}

\begin{itemize}
    \item \textbf{Ưu điểm:} Đơn giản, dễ triển khai, chi phí thấp.
    \item \textbf{Nhược điểm:} Biến điều khiển dao động quanh setpoint, không đạt được điều khiển chính xác.
    \item \textbf{Ứng dụng:} Điều khiển nhiệt độ lò nung, bình nước nóng, máy điều hòa...
\end{itemize}

% ------------------------------------------------------------
\subsection{Thiết bị thí nghiệm}
% ------------------------------------------------------------

\begin{itemize}
    \item PLC Siemens S7-1200 CPU 1214C
    \item Module analog input/output
    \item Cảm biến nhiệt độ Pt100
    \item Bộ gia nhiệt (Heater)
    \item Phần mềm TIA Portal V16
\end{itemize}

% ------------------------------------------------------------
\subsection{Quy trình thực hiện}
% ------------------------------------------------------------

\subsubsection{Cấu hình phần cứng}
\begin{enumerate}
    \item Tạo project mới trong TIA Portal V16.
    \item Thêm PLC S7-1200 và cấu hình module analog.
    \item Kết nối cảm biến Pt100 với analog input.
    \item Kết nối bộ gia nhiệt với digital output.
\end{enumerate}

\subsubsection{Lập trình điều khiển ON/OFF}
\begin{enumerate}
    \item Đọc giá trị nhiệt độ từ cảm biến.
    \item So sánh với setpoint và tính sai lệch $e = SP - PV$.
    \item Áp dụng logic ON/OFF với hysteresis:
    \begin{itemize}
        \item Nếu $PV < SP - h$: Bật heater (ON)
        \item Nếu $PV > SP + h$: Tắt heater (OFF)
    \end{itemize}
    \item Quan sát đáp ứng của hệ thống.
\end{enumerate}

\subsubsection{Khảo sát ảnh hưởng của hysteresis}
\begin{enumerate}
    \item Thay đổi giá trị hysteresis: $h$ = 1°C, 2°C, 5°C.
    \item Ghi nhận biên độ dao động và tần số đóng ngắt.
\end{enumerate}

% ------------------------------------------------------------
\subsection{Kết quả thí nghiệm}
% ------------------------------------------------------------

\subsubsection{Đáp ứng hệ thống}

% TODO: Thêm đồ thị đáp ứng

\begin{table}[H]
\centering
\begin{tabular}{|c|c|c|c|}
\hline
\textbf{Hysteresis (°C)} & \textbf{Biên độ dao động (°C)} & \textbf{Chu kỳ dao động (s)} & \textbf{Tần số đóng ngắt} \\
\hline
1 & & & \\
\hline
2 & & & \\
\hline
5 & & & \\
\hline
\end{tabular}
\caption{Kết quả khảo sát ảnh hưởng của hysteresis}
\end{table}

% ------------------------------------------------------------
\subsection{Bàn luận}
% ------------------------------------------------------------

\begin{itemize}
    \item Nhận xét về đặc tính dao động của điều khiển ON/OFF.
    \item Phân tích ảnh hưởng của hysteresis đến chất lượng điều khiển.
    \item So sánh ưu nhược điểm với điều khiển liên tục (PID).
\end{itemize}

% ------------------------------------------------------------
\subsection{Kết luận}
% ------------------------------------------------------------

% TODO: Điền kết luận sau khi thực hiện thí nghiệm

% ------------------------------------------------------------
\subsection{Câu hỏi kiểm tra}
% ------------------------------------------------------------

\textbf{Câu 1: Tại sao cần hysteresis trong điều khiển ON/OFF?}

% TODO: Trả lời

\textbf{Câu 2: Ưu nhược điểm của điều khiển ON/OFF so với PID?}

% TODO: Trả lời

\textbf{Câu 3: Ứng dụng thực tế của điều khiển ON/OFF?}

% TODO: Trả lời

% ------------------------------------------------------------
\subsection{Tài liệu tham khảo}
% ------------------------------------------------------------

\begin{enumerate}[label={[\arabic*]}]
    \item B. N. Pha, \textit{Thiết bị Đo lường và Điều khiển}. Trường Đại học Bách khoa, ĐHQG-HCM.
    \item Siemens, \textit{S7-1200 Programmable Controller System Manual}. Siemens AG, 2020.
\end{enumerate}

\newpage
