% ============================================================
% Chương 6: Kết luận và hướng phát triển
% ============================================================

\newpage
\section{KẾT LUẬN VÀ KIẾN NGHỊ}

% ------------------------------------------------------------
\subsection{Kết luận}
% ------------------------------------------------------------

Đồ án đã thực hiện các nội dung sau:

\begin{enumerate}
    \item \textbf{Phân tích hiện trạng tiêu thụ năng lượng:}
    \begin{itemize}
        \item Xác định nguyên nhân lãng phí năng lượng khi vận hành với van làm mát mở 100\%.
        \item Phân tích mối quan hệ giữa công suất làm mát và năng lượng reboiler.
        \item Đánh giá lượng nước làm mát sử dụng dư thừa trong phương án vận hành hiện tại.
    \end{itemize}
    
    \item \textbf{Thiết kế vòng điều khiển áp suất đỉnh tháp:}
    \begin{itemize}
        \item Xây dựng vòng điều khiển PC-01 với bộ điều khiển PID.
        \item Đề xuất phương pháp xác định thông số PID (Ziegler-Nichols và mô phỏng).
        \item Xây dựng mô hình toán học của hệ thống để phục vụ mô phỏng.
    \end{itemize}
    
    \item \textbf{Mô hình hóa và mô phỏng:}
    \begin{itemize}
        \item Xây dựng mô hình hệ thống điều khiển bằng Python.
        \item Mô phỏng so sánh hai phương án: baseline (van 100\%) và điều khiển tối ưu.
        \item Đánh giá hiệu quả tiết kiệm năng lượng thông qua mô phỏng.
    \end{itemize}
\end{enumerate}

\textbf{Kết quả dự kiến:}

Phương pháp điều khiển áp suất đỉnh tháp thông qua van nước làm mát dự kiến mang lại:
\begin{itemize}
    \item Giảm đáng kể năng lượng cung cấp cho reboiler
    \item Giảm lượng nước làm mát sử dụng
    \item Duy trì chất lượng sản phẩm đỉnh
    \item Áp suất đỉnh tháp ổn định tại giá trị đặt
\end{itemize}

\textbf{Kết luận tổng quát:} Giải pháp điều khiển áp suất đỉnh tháp là phương pháp hiệu quả để tối ưu hóa năng lượng cung cấp cho quá trình chưng cất ethanol -- nước. Giải pháp này không đòi hỏi thay đổi cấu hình thiết bị, chỉ cần bổ sung hệ thống điều khiển tự động, có thể áp dụng cho các hệ thống chưng cất hiện có.

% ------------------------------------------------------------
\subsection{Kiến nghị và hướng phát triển}
% ------------------------------------------------------------

Để nâng cao hiệu quả và mở rộng phạm vi ứng dụng, đề xuất các hướng phát triển sau:

\begin{enumerate}
    \item \textbf{Kiểm chứng thực nghiệm:}
    \begin{itemize}
        \item Triển khai thử nghiệm trên hệ thống chưng cất thực tế tại phòng thí nghiệm.
        \item So sánh kết quả thực nghiệm với kết quả mô phỏng.
        \item Tinh chỉnh thông số PID dựa trên đáp ứng thực tế.
    \end{itemize}
    
    \item \textbf{Mở rộng quy mô áp dụng:}
    \begin{itemize}
        \item Triển khai thử nghiệm trên hệ thống chưng cất quy mô pilot hoặc công nghiệp.
        \item Đánh giá hiệu quả tiết kiệm năng lượng ở các quy mô khác nhau.
        \item Nghiên cứu ảnh hưởng của các yếu tố vận hành thực tế (nhiễu loạn, thay đổi tải).
    \end{itemize}
    
    \item \textbf{Cải tiến thuật toán điều khiển:}
    \begin{itemize}
        \item Áp dụng điều khiển thích nghi (Adaptive PID) để tự động điều chỉnh thông số theo điều kiện vận hành.
        \item Nghiên cứu điều khiển dự báo mô hình (MPC - Model Predictive Control) để tối ưu đa mục tiêu.
        \item Tích hợp trí tuệ nhân tạo (AI) để dự đoán và tối ưu hóa vận hành.
    \end{itemize}
    
    \item \textbf{Tối ưu hóa toàn diện hệ thống:}
    \begin{itemize}
        \item Kết hợp điều khiển áp suất với tối ưu tỉ lệ hồi lưu.
        \item Nghiên cứu tích hợp nhiệt (heat integration) giữa các dòng trong hệ thống.
        \item Phát triển hệ thống quản lý năng lượng tổng thể cho nhà máy.
    \end{itemize}
    
    \item \textbf{Phát triển hệ thống giám sát và báo cáo:}
    \begin{itemize}
        \item Xây dựng dashboard giám sát năng lượng theo thời gian thực.
        \item Tích hợp hệ thống IoT để thu thập và phân tích dữ liệu vận hành.
        \item Phát triển báo cáo tự động về hiệu quả năng lượng.
    \end{itemize}
\end{enumerate}

% ------------------------------------------------------------
% Bảng kế hoạch thực hiện
% ------------------------------------------------------------

\begin{table}[H]
\centering
\begin{tabular}{|c|l|p{7cm}|}
\hline
\textbf{Tuần} & \textbf{Thời gian} & \textbf{Nội dung công việc} \\
\hline
1 & 05/01 -- 11/01/2026 & Nghiên cứu tài liệu về điều khiển quá trình, PID, tối ưu năng lượng chưng cất \\
\hline
2 & 12/01 -- 18/01/2026 & Khảo sát hệ thống thực tế, đo đạc và thu thập thông số thiết bị \\
\hline
3 & 19/01 -- 25/01/2026 & Xây dựng mô hình toán học: hàm truyền condenser, van, cảm biến \\
\hline
4 & 02/02 -- 08/02/2026 & Lập trình Python: mô hình hóa quá trình, cân bằng năng lượng \\
\hline
5 & 09/02 -- 15/02/2026 & Lập trình Python: bộ điều khiển PID, vòng điều khiển kín \\
\hline
6 & 23/02 -- 01/03/2026 & Mô phỏng so sánh: baseline (van 100\%) và điều khiển PID \\
\hline
7 & 02/03 -- 08/03/2026 & Tối ưu thông số PID (Ziegler-Nichols, ISE/IAE), xuất đồ thị kết quả \\
\hline
8 & 09/03 -- 15/03/2026 & Lập trình PLC S7-1200 (TIA Portal), thiết kế giao diện HMI \\
\hline
9 & 16/03 -- 22/03/2026 & Thí nghiệm baseline: vận hành van 100\%, thu thập dữ liệu năng lượng \\
\hline
10 & 23/03 -- 29/03/2026 & Thí nghiệm điều khiển PID: áp dụng thông số từ mô phỏng \\
\hline
11 & 30/03 -- 05/04/2026 & Tinh chỉnh thông số PID trên thiết bị thực, thí nghiệm bổ sung \\
\hline
12 & 06/04 -- 12/04/2026 & Xử lý số liệu, so sánh kết quả mô phỏng và thực nghiệm \\
\hline
13 & 13/04 -- 19/04/2026 & Viết báo cáo: Chương 1--4 (Mở đầu, Lý thuyết, Mô hình, Giải pháp) \\
\hline
14 & 20/04 -- 26/04/2026 & Viết báo cáo: Chương 5--6 (Kết quả, Kết luận), hoàn thiện tài liệu \\
\hline
15 & 04/05 -- 10/05/2026 & Chuẩn bị slide, poster và bảo vệ đồ án \\
\hline
\end{tabular}
\caption{Kế hoạch thực hiện đồ án}
\label{tab:schedule}
\end{table}
