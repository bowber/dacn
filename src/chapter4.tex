% ============================================================
% Chương 4: Phương pháp tối ưu năng lượng
% ============================================================

\newpage
\section{GIẢI PHÁP TỐI ƯU HÓA NĂNG LƯỢNG}

% ------------------------------------------------------------
\subsection{Phân tích tiêu thụ năng lượng}
% ------------------------------------------------------------

\subsubsection{Năng lượng cung cấp cho reboiler}

Reboiler sử dụng điện trở gia nhiệt với công suất tối đa 3000W. Năng lượng tiêu thụ tại reboiler được tính theo công thức:

\begin{equation}
    E_R = P_R \times t
\end{equation}

Trong đó:
\begin{itemize}
    \item $E_R$: Năng lượng tiêu thụ (kWh)
    \item $P_R$: Công suất reboiler (kW)
    \item $t$: Thời gian vận hành (h)
\end{itemize}

Công suất reboiler được điều khiển thông qua bộ điều khiển nhiệt độ đỉnh tháp. Khi hệ thống hoạt động ổn định, công suất reboiler phụ thuộc vào:
\begin{itemize}
    \item Lưu lượng nhập liệu
    \item Nhiệt độ nhập liệu
    \item Công suất làm mát tại bộ ngưng tụ
    \item Nhiệt mất mát ra môi trường
\end{itemize}

\subsubsection{Năng lượng cho bộ ngưng tụ}

Bộ ngưng tụ sử dụng nước làm mát để ngưng tụ hơi ethanol từ đỉnh tháp. Công suất làm mát được tính:

\begin{equation}
    Q_C = \dot{m}_{cw} \times C_p \times \Delta T_{cw}
\end{equation}

Trong đó:
\begin{itemize}
    \item $Q_C$: Công suất làm mát (kW)
    \item $\dot{m}_{cw}$: Lưu lượng nước làm mát (kg/s)
    \item $C_p$: Nhiệt dung riêng của nước (4.18 kJ/kg.K)
    \item $\Delta T_{cw}$: Độ tăng nhiệt độ nước làm mát (K)
\end{itemize}

Lưu lượng nước làm mát phụ thuộc vào độ mở van:
\begin{equation}
    \dot{m}_{cw} = \dot{m}_{cw,max} \times \frac{u}{100}
\end{equation}

Trong đó $u$ là độ mở van (0--100\%).

\subsubsection{Cân bằng năng lượng tổng thể}

Ở trạng thái xác lập, cân bằng năng lượng cho hệ thống:

\begin{equation}
    Q_R = Q_C + Q_{product} + Q_{loss}
\end{equation}

Trong đó:
\begin{itemize}
    \item $Q_R$: Công suất reboiler
    \item $Q_C$: Công suất làm mát
    \item $Q_{product}$: Nhiệt lượng mang theo sản phẩm
    \item $Q_{loss}$: Nhiệt mất mát ra môi trường
\end{itemize}

Từ phương trình trên, ta thấy rằng khi giảm $Q_C$ (bằng cách giảm độ mở van làm mát), $Q_R$ cũng giảm tương ứng, với điều kiện các yếu tố khác không đổi.

% ------------------------------------------------------------
\subsection{Hiện trạng vận hành và vấn đề năng lượng}
% ------------------------------------------------------------

\subsubsection{Khảo sát vận hành với van làm mát mở 100\%}

Thí nghiệm khảo sát được thực hiện với các điều kiện:
\begin{itemize}
    \item Lưu lượng nhập liệu: 1.5 L/h
    \item Nồng độ nhập liệu: 15\% vol ethanol
    \item Nhiệt độ đỉnh tháp đặt: 78.5$^\circ$C
    \item Van làm mát: Mở 100\%
\end{itemize}

\begin{figure}[H]
\centering
\begin{tikzpicture}
\begin{axis}[
    width=0.9\textwidth,
    height=7cm,
    xlabel={Thời gian (phút)},
    ylabel={Công suất reboiler (\%)},
    grid=both,
    grid style={line width=.1pt, draw=gray!30},
    xmin=0, xmax=120,
    ymin=0, ymax=100,
    legend pos=north east,
]
% Công suất reboiler khi van mở 100%
\addplot[blue, thick] coordinates {
    (0, 0) (5, 45) (10, 68) (15, 75) (20, 78) (25, 80) (30, 82)
    (35, 83) (40, 84) (45, 84) (50, 85) (55, 85) (60, 85)
    (65, 85) (70, 85) (75, 85) (80, 85) (85, 85) (90, 85)
    (95, 85) (100, 85) (105, 85) (110, 85) (115, 85) (120, 85)
};
\addlegendentry{Van làm mát 100\%}

\end{axis}
\end{tikzpicture}
\caption{Công suất reboiler khi vận hành với van làm mát mở 100\%}
\label{fig:baseline_power}
\end{figure}

\textbf{Nhận xét:}
\begin{itemize}
    \item Công suất reboiler ổn định ở mức khoảng 85\% (tương đương 2550W).
    \item Áp suất đỉnh tháp duy trì ổn định ở mức thấp (khoảng 1.01 bar) do công suất làm mát dư thừa.
    \item Lượng nước làm mát sử dụng là tối đa, gây lãng phí.
\end{itemize}

\subsubsection{Phân tích nguyên nhân lãng phí năng lượng}

Khi van làm mát mở 100\%:
\begin{enumerate}
    \item Công suất ngưng tụ ($Q_C$) lớn hơn nhiều so với lượng hơi cần ngưng tụ.
    \item Hơi ethanol ngưng tụ nhanh, tạo chân không nhẹ tại đỉnh tháp.
    \item Áp suất đỉnh tháp giảm, kéo theo sự tăng lưu lượng hơi từ reboiler.
    \item Reboiler phải cung cấp thêm năng lượng để bù đắp.
\end{enumerate}

Ước tính năng lượng lãng phí:
\begin{equation}
    Q_{waste} = Q_C - Q_{C,required}
\end{equation}

% ------------------------------------------------------------
\subsection{Thiết kế hệ thống điều khiển tối ưu hóa năng lượng}
% ------------------------------------------------------------

\subsubsection{Cấu trúc vòng điều khiển áp suất}

Vòng điều khiển áp suất đỉnh tháp (PC-01) được thiết kế với:
\begin{itemize}
    \item \textbf{Biến được điều khiển (PV):} Áp suất đỉnh tháp
    \item \textbf{Biến điều khiển (MV):} Độ mở van nước làm mát
    \item \textbf{Giá trị đặt (SP):} 1.05 bar (áp suất tối ưu)
    \item \textbf{Thuật toán điều khiển:} PID
\end{itemize}

\textbf{Lựa chọn giá trị đặt áp suất:}

Giá trị đặt áp suất được chọn dựa trên các tiêu chí:
\begin{itemize}
    \item Đủ cao để đảm bảo hơi ngưng tụ hiệu quả
    \item Không quá cao gây nguy hiểm cho thiết bị
    \item Tối ưu cho quá trình phân tách
\end{itemize}

Giá trị SP = 1.05 bar được chọn dựa trên kinh nghiệm vận hành và khuyến cáo của nhà sản xuất thiết bị.

\subsubsection{Xác định thông số bộ điều khiển PID}

Thông số PID được xác định bằng phương pháp thực nghiệm (Ziegler-Nichols):

\textbf{Bước 1: Xác định đặc tính quá trình}

Thực hiện thí nghiệm step response bằng cách thay đổi độ mở van từ 50\% lên 60\% và ghi nhận đáp ứng áp suất:

\begin{figure}[H]
\centering
\begin{tikzpicture}
\begin{axis}[
    width=0.9\textwidth,
    height=6cm,
    xlabel={Thời gian (s)},
    ylabel={Áp suất (bar)},
    grid=both,
    grid style={line width=.1pt, draw=gray!30},
    xmin=0, xmax=300,
    ymin=1.04, ymax=1.08,
    legend pos=south east,
]
% Step response
\addplot[blue, thick] coordinates {
    (0, 1.07) (10, 1.07) (20, 1.068) (30, 1.065) (40, 1.062)
    (50, 1.059) (60, 1.057) (70, 1.055) (80, 1.054) (90, 1.053)
    (100, 1.052) (120, 1.051) (140, 1.050) (160, 1.050) (180, 1.050)
    (200, 1.050) (220, 1.050) (240, 1.050) (260, 1.050) (280, 1.050) (300, 1.050)
};
\addlegendentry{Áp suất đỉnh tháp}

% Step input annotation
\draw[dashed, gray] (axis cs:10, 1.04) -- (axis cs:10, 1.08);
\node[font=\small] at (axis cs:10, 1.042) {Step 50\%$\rightarrow$60\%};

\end{axis}
\end{tikzpicture}
\caption{Đáp ứng bước của áp suất đỉnh tháp khi thay đổi độ mở van}
\label{fig:step_response}
\end{figure}

Từ đáp ứng bước, xác định các thông số:
\begin{itemize}
    \item Độ lợi quá trình: $K_p = \frac{\Delta PV}{\Delta MV} = \frac{1.07 - 1.05}{60 - 50} = -0.002$ bar/\%
    \item Hằng số thời gian: $\tau \approx 60$ s
    \item Thời gian trễ: $\theta \approx 10$ s
\end{itemize}

\textbf{Bước 2: Tính toán thông số PID theo Ziegler-Nichols}

Áp dụng công thức Ziegler-Nichols cho hệ bậc nhất có trễ:

\begin{table}[H]
\centering
\begin{tabular}{|c|c|c|c|}
\hline
\textbf{Bộ điều khiển} & \textbf{$K_c$} & \textbf{$T_i$ (s)} & \textbf{$T_d$ (s)} \\
\hline
P & $\frac{\tau}{K_p \cdot \theta} = 300$ & -- & -- \\
\hline
PI & $\frac{0.9\tau}{K_p \cdot \theta} = 270$ & $3.3\theta = 33$ & -- \\
\hline
PID & $\frac{1.2\tau}{K_p \cdot \theta} = 360$ & $2\theta = 20$ & $0.5\theta = 5$ \\
\hline
\end{tabular}
\caption{Thông số PID theo Ziegler-Nichols}
\label{tab:pid_tuning}
\end{table}

\textbf{Thông số PID được chọn:} $K_c = 300$, $T_i = 25$ s, $T_d = 5$ s

(Các giá trị được tinh chỉnh sau khi chạy thử nghiệm thực tế)

\subsubsection{Mô phỏng và kiểm chứng}

Trước khi triển khai thực tế, hệ thống điều khiển được mô phỏng trên Matlab/Simulink để kiểm tra tính ổn định và hiệu quả.

\begin{figure}[H]
\centering
\begin{tikzpicture}
\begin{axis}[
    width=0.9\textwidth,
    height=6cm,
    xlabel={Thời gian (s)},
    ylabel={Áp suất (bar)},
    grid=both,
    grid style={line width=.1pt, draw=gray!30},
    xmin=0, xmax=600,
    ymin=1.00, ymax=1.15,
    legend pos=north east,
]
% Setpoint
\addplot[red, dashed, thick] coordinates {
    (0, 1.05) (600, 1.05)
};
\addlegendentry{Setpoint (1.05 bar)}

% PV với điều khiển PID
\addplot[blue, thick] coordinates {
    (0, 1.01) (20, 1.02) (40, 1.03) (60, 1.04) (80, 1.048)
    (100, 1.052) (120, 1.054) (140, 1.052) (160, 1.050) (180, 1.050)
    (200, 1.050) (220, 1.050) (240, 1.050) (260, 1.050) (280, 1.050)
    (300, 1.050) (320, 1.050) (340, 1.050) (360, 1.050) (380, 1.050)
    (400, 1.050) (420, 1.050) (440, 1.050) (460, 1.050) (480, 1.050)
    (500, 1.050) (520, 1.050) (540, 1.050) (560, 1.050) (580, 1.050) (600, 1.050)
};
\addlegendentry{PV với PID}

\end{axis}
\end{tikzpicture}
\caption{Kết quả mô phỏng vòng điều khiển áp suất}
\label{fig:simulation_result}
\end{figure}

\textbf{Kết quả mô phỏng:}
\begin{itemize}
    \item Thời gian đạt setpoint: khoảng 100s
    \item Độ quá điều chỉnh: 0.4\% (chấp nhận được)
    \item Sai lệch tĩnh: 0\% (nhờ thành phần I)
    \item Hệ thống ổn định, không dao động
\end{itemize}
