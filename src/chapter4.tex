% ============================================================
% Chương 4: Tính toán và kết quả
% ============================================================

\newpage
\section{TÍNH TOÁN VÀ KẾT QUẢ}

% ------------------------------------------------------------
\subsection{Thiết lập tính toán}
% ------------------------------------------------------------

\subsubsection{Công cụ tính toán}

Tính toán được thực hiện bằng ngôn ngữ Python với các thư viện:
\begin{itemize}
    \item \texttt{numpy}: Tính toán số học
    \item \texttt{scipy}: Giải phương trình, tối ưu hóa
    \item \texttt{matplotlib}: Vẽ đồ thị
\end{itemize}

\subsubsection{Thông số hệ thống}

Các thông số hệ thống được sử dụng trong tính toán:

\begin{table}[H]
\centering
\begin{tabular}{|l|c|c|}
\hline
\textbf{Thông số} & \textbf{Giá trị} & \textbf{Đơn vị} \\
\hline
Công suất reboiler tối đa & 6000 & W \\
\hline
Lưu lượng nước làm mát tối đa & 7.2 & L/min \\
\hline
Diện tích trao đổi nhiệt condenser & 0.28 & m$^2$ \\
\hline
Áp suất vận hành & 1.0 & bar \\
\hline
Nồng độ nhập liệu & 10 & \% vol \\
\hline
Lưu lượng nhập liệu & 4.8 & L/h \\
\hline
\end{tabular}
\caption{Thông số hệ thống cho tính toán}
\label{tab:simulation_params}
\end{table}

% ------------------------------------------------------------
\subsection{Tính toán cân bằng năng lượng}
% ------------------------------------------------------------

Tính toán cân bằng năng lượng để xác định công suất ngưng tụ tối ưu và lưu lượng nước làm mát tương ứng.

\subsubsection{Thông số vận hành}

Thông số vận hành của hệ thống:
\begin{itemize}
    \item Nhập liệu: 4.8 L/h, 10\% vol ethanol, 70$^\circ$C
    \item Sản phẩm đỉnh: 90\% vol ethanol
    \item Công suất reboiler điện: 6000 W (2 $\times$ 3000 W) -- giá trị ước tính, cần đo lại trong LVTN
\end{itemize}

\subsubsection{Cân bằng năng lượng tại reboiler}

Công suất reboiler bao gồm hai thành phần:
\begin{equation}
    Q_{reboiler} = Q_{feed} + Q_{vaporize}
\end{equation}

\textbf{Công suất reboiler hiệu dụng:}

Công suất điện đầu vào không hoàn toàn chuyển thành nhiệt hữu ích do sụt áp trên dây dẫn, tiếp điểm và tổn hao nhiệt ra môi trường. Ước tính tổn hao: 10\%.
\begin{equation}
    Q_{reboiler} = Q_{electrical} \times (1 - 0.10) = 6000 \times 0.90 = 5400 \text{ W}
\end{equation}

\textbf{Nhiệt gia nhiệt nhập liệu ($Q_{feed}$):}

Nhiệt cần thiết để nâng nhiệt độ nhập liệu từ 70$^\circ$C đến điểm sôi. Điểm sôi của hỗn hợp 10\% vol ethanol (3.32\% mol) tại 1 bar được tính từ phương trình Antoine kết hợp hệ số hoạt độ Van Laar \cite{gmehling1991}:
\begin{equation}
    T_{bubble} = 92.5 ^\circ\text{C}
\end{equation}

Nhiệt gia nhiệt:
\begin{equation}
    Q_{feed} = \dot{m}_F \cdot C_p \cdot \Delta T
\end{equation}
với:
\begin{itemize}
    \item $\dot{m}_F = 4.8$ L/h $\times$ 977 kg/m$^3$ / 3600 = 1.30 g/s (mật độ hỗn hợp 10\% ethanol \cite{crc2020})
    \item $C_p = 4039$ J/(kg·K) (hỗn hợp 10\% ethanol, tính từ quy tắc trộn \cite{perry2019})
    \item $\Delta T = 92.5 - 70 = 22.5$ °C
\end{itemize}

Kết quả: $Q_{feed} = 118$ W

\textbf{Nhiệt hóa hơi ($Q_{vaporize}$):}

Từ cân bằng năng lượng tại reboiler:
\begin{equation}
    Q_{vaporize} = Q_{reboiler} - Q_{feed} = 5400 - 118 = 5282 \text{ W}
\end{equation}

Đây là công suất nhiệt thực tế dùng để hóa hơi hỗn hợp tại đáy tháp.

\subsubsection{Công suất ngưng tụ tối ưu}

Để vận hành tối ưu, công suất ngưng tụ cần bằng với nhiệt hóa hơi:
\begin{equation}
    Q_{condenser,optimal} = Q_{vaporize} = 5282 \text{ W}
\end{equation}

Nếu $Q_{condenser} < Q_{vaporize}$: hơi không ngưng tụ hết, áp suất tháp tăng.

Nếu $Q_{condenser} > Q_{vaporize}$: dịch ngưng bị làm lạnh dưới nhiệt độ sôi (subcooling), lãng phí nước làm mát.

\subsubsection{Tính toán lưu lượng nước làm mát}

Công suất ngưng tụ được tính từ cân bằng năng lượng phía nước làm mát:
\begin{equation}
    Q_{condenser} = \dot{m}_{water} \cdot C_p \cdot (T_{out} - T_{in})
\end{equation}

Trong đó:
\begin{itemize}
    \item $\dot{m}_{water}$: lưu lượng khối lượng nước làm mát (kg/s)
    \item $C_p = 4186$ J/(kg·K): nhiệt dung riêng của nước
    \item $T_{in} = 30^\circ$C: nhiệt độ nước vào (nhiệt độ nước máy)
    \item $T_{out}$: nhiệt độ nước ra
\end{itemize}

\textbf{Phân tích chế độ baseline (van mở 100\%):}

Với lưu lượng tối đa $F = 7.2$ L/min = 0.12 kg/s, nếu condenser chỉ cần hấp thụ $Q_{vaporize} = 5282$ W:
\begin{equation}
    T_{out} = T_{in} + \frac{Q_{vaporize}}{\dot{m}_{water} \cdot C_p} = 30 + \frac{5282}{0.12 \times 4186} = 40.5^\circ\text{C}
\end{equation}

Nhiệt độ nước ra thấp (40.5$^\circ$C) so với nhiệt độ ngưng tụ (78$^\circ$C) cho thấy lưu lượng nước đang dư thừa đáng kể. Phần công suất làm mát dư thừa sẽ làm lạnh dịch ngưng xuống dưới nhiệt độ sôi (subcooling).

\textbf{Tính toán lưu lượng tối ưu:}

Để vận hành tối ưu với $Q_{condenser} = Q_{vaporize} = 5282$ W, lưu lượng nước cần thiết phụ thuộc vào nhiệt độ nước ra:
\begin{equation}
    \dot{m}_{optimal} = \frac{Q_{vaporize}}{C_p \cdot (T_{out} - T_{in})}
\end{equation}

Trong thiết kế thiết bị trao đổi nhiệt, chênh lệch nhiệt độ tiếp cận (approach temperature) thường được chọn khoảng 10$^\circ$C để đảm bảo động lực truyền nhiệt đủ lớn. Với $T_{condensation} = 78^\circ$C:
\begin{equation}
    T_{out} = T_{condensation} - \Delta T_{approach} = 78 - 10 = 68^\circ\text{C}
\end{equation}

Lưu lượng nước làm mát tối ưu:
\begin{equation}
    \dot{m}_{optimal} = \frac{5282}{4186 \times (68 - 30)} = 0.033 \text{ kg/s} = 2.0 \text{ L/min}
\end{equation}

Tương ứng với độ mở van 28\% (so với 7.2 L/min ở 100\%), tiết kiệm 72\% lưu lượng nước làm mát.

\textbf{Lưu ý:} Giá trị thực tế cần được xác nhận qua thí nghiệm trong LVTN bằng cách đo nhiệt độ nước vào/ra ở các mức độ mở van khác nhau.

\subsubsection{Điều khiển công suất reboiler bằng PWM}

Reboiler hiện tại chạy 100\% công suất (ước tính 6000 W -- cần đo xác nhận trong LVTN), nhưng thực tế có thể không cần nhiều như vậy. Việc giảm công suất reboiler sẽ:
\begin{itemize}
    \item Giảm lượng hơi tạo ra trong tháp
    \item Giảm công suất ngưng tụ cần thiết
    \item Tiết kiệm cả điện năng reboiler và nước làm mát
\end{itemize}

\textbf{Phương pháp xác định duty cycle tối ưu:}

Duty cycle tối ưu được xác định bằng thực nghiệm theo quy trình sau:

\begin{enumerate}
    \item \textbf{Bắt đầu ở 100\%:} Hệ thống hoạt động ổn định, sản phẩm đạt chất lượng yêu cầu (90\% vol).
    
    \item \textbf{Giảm dần duty cycle:} Giảm từng bước 10\% (90\% $\rightarrow$ 80\% $\rightarrow$ 70\%...), chờ hệ thống ổn định sau mỗi bước (khoảng 10--15 phút).
    
    \item \textbf{Giám sát các thông số:}
    \begin{itemize}
        \item Nhiệt độ đỉnh tháp (tăng khi nồng độ sản phẩm giảm)
        \item Áp suất đỉnh tháp (cần ổn định)
    \end{itemize}
    
    \item \textbf{Xác định điểm giới hạn:} Duty cycle tại đó nhiệt độ đỉnh tháp tăng vượt ngưỡng cho phép (tương ứng nồng độ sản phẩm giảm dưới 90\% vol) hoặc vận hành trở nên không ổn định.
    
    \item \textbf{Thêm biên độ an toàn:} Tăng duty cycle lên 10\% so với điểm giới hạn để đảm bảo vận hành ổn định.
\end{enumerate}

Giá trị duty cycle tối ưu phụ thuộc vào:
\begin{itemize}
    \item Năng suất sản phẩm mong muốn
    \item Nồng độ nhập liệu thực tế
    \item Hiệu suất cách nhiệt của tháp
    \item Nhiệt độ nước làm mát
\end{itemize}

\textbf{Kết hợp điều khiển van làm mát và PWM reboiler:}

Khi giảm công suất reboiler bằng PWM, lượng hơi tạo ra giảm, dẫn đến công suất ngưng tụ cần thiết giảm tương ứng. Bộ điều khiển áp suất PC-01 sẽ tự động giảm độ mở van làm mát để cân bằng.

% ------------------------------------------------------------
\subsection{Kết quả tính toán trạng thái xác lập}
% ------------------------------------------------------------

Dựa trên các phương trình cân bằng năng lượng và truyền nhiệt, tính toán trạng thái xác lập cho hai chế độ vận hành: baseline (van mở 100\%) và vận hành tối ưu với điều khiển.

\subsubsection{Chế độ baseline (van mở 100\%)}

Trong chế độ vận hành baseline, van nước làm mát được mở hoàn toàn (100\%) -- đây là cách vận hành truyền thống không có điều khiển tự động.

Kết quả tính toán trạng thái xác lập:
\begin{itemize}
    \item Áp suất đỉnh tháp: 1.0 bar (giá trị đặt)
    \item Van nước làm mát: 100\%
    \item Lưu lượng nước làm mát: 7.2 L/min
    \item Công suất ngưng tụ: 9113 W (vượt quá nhu cầu thực tế 5282 W)
    \item Nhiệt độ đỉnh: 78.3°C
\end{itemize}

\subsubsection{Chế độ vận hành tối ưu với điều khiển}

Với bộ điều khiển áp suất, van nước làm mát được điều chỉnh tự động để cân bằng công suất ngưng tụ với nhiệt hóa hơi.

Kết quả tính toán trạng thái xác lập:
\begin{itemize}
    \item Áp suất đỉnh tháp: 1.0 bar (duy trì ổn định)
    \item Van nước làm mát: 28\% (tiết kiệm 72\%)
    \item Lưu lượng nước làm mát: 2.0 L/min
    \item Công suất ngưng tụ: 5282 W (cân bằng với $Q_{vaporize}$)
    \item Nhiệt độ đỉnh: 78.3°C
\end{itemize}

% ------------------------------------------------------------
\subsection{So sánh hiệu quả vận hành}
% ------------------------------------------------------------

Bảng \ref{tab:sim_productivity_comparison} so sánh kết quả tính toán giữa hai phương án vận hành.

\begin{table}[H]
\centering
\begin{tabular}{|l|c|c|c|}
\hline
\textbf{Thông số} & \textbf{Baseline} & \textbf{Điều khiển van} & \textbf{Đơn vị} \\
\hline
Công suất reboiler & 6000 & 6000 & W \\
\hline
Độ mở van & 100 & 28 & \% \\
\hline
Công suất ngưng tụ & 9113 & 5282 & W \\
\hline
Lưu lượng nước & 7.2 & 2.0 & L/min \\
\hline
\end{tabular}
\caption{So sánh kết quả tính toán giữa các chế độ vận hành}
\label{tab:sim_productivity_comparison}
\end{table}

\textbf{Kết quả chính:}
\begin{itemize}
    \item \textbf{Điều khiển van (so với baseline):} Giảm 72\% lưu lượng nước làm mát trong khi duy trì áp suất đỉnh tháp ổn định.
\end{itemize}

% ------------------------------------------------------------
\subsection{Thảo luận}
% ------------------------------------------------------------

\subsubsection{Phân tích kết quả tính toán}

Kết quả tính toán cho thấy:
\begin{enumerate}
    \item \textbf{Cân bằng năng lượng}: Bộ điều khiển tự động điều chỉnh độ mở van để công suất ngưng tụ (5282 W) cân bằng với nhiệt hóa hơi ($Q_{vaporize} = Q_{reboiler} - Q_{feed} = 5400 - 118 = 5282$ W). Điều này đảm bảo quá trình ngưng tụ hiệu quả mà không lãng phí nước làm mát.
    
    \item \textbf{Tiết kiệm nước làm mát}: Van làm mát giảm từ 100\% xuống 28\%, tiết kiệm 72\% lưu lượng nước làm mát. Đây là kết quả của việc sử dụng đúng lượng nước cần thiết thay vì mở van tối đa.
    
    \item \textbf{Áp suất ổn định}: Bộ điều khiển duy trì áp suất đỉnh tháp ổn định tại giá trị đặt 1.0 bar bằng cách điều chỉnh độ mở van nước làm mát.
\end{enumerate}

\subsubsection{Giải thích cơ chế hoạt động}

\textbf{Nguyên lý cân bằng năng lượng:} Công suất ngưng tụ được xác định bởi cân bằng năng lượng phía nước làm mát:
\begin{equation}
    Q_{condenser} = \dot{m}_{water} \cdot C_p \cdot (T_{out} - T_{in})
\end{equation}

Khi van mở 100\%:
\begin{itemize}
    \item Lưu lượng nước: 7.2 L/min
    \item Nước chỉ cần hấp thụ $Q_{vaporize} = 5282$ W
    \item Nhiệt độ nước ra: $T_{out} = 30 + \frac{5282}{0.12 \times 4186} = 40.5^\circ$C
    \item Chênh lệch với nhiệt độ ngưng tụ (78°C) rất lớn -- lãng phí 72\% lưu lượng nước.
\end{itemize}

Với điều khiển (van 28\%):
\begin{itemize}
    \item Lưu lượng nước: 2.0 L/min
    \item Nhiệt độ nước ra: $T_{out} = 68^\circ$C (chênh lệch 10°C với nhiệt độ ngưng tụ)
    \item Cân bằng năng lượng tối ưu -- tiết kiệm 72\% nước làm mát.
\end{itemize}

\subsubsection{Phương pháp kiểm chứng thực nghiệm}

Để kiểm chứng hiệu quả của chiến lược điều khiển van làm mát, sử dụng độ chênh nhiệt độ giữa đỉnh tháp và dịch ngưng:
\begin{equation}
    \Delta T_{subcool} = T_{đỉnh} - T_{dịch ngưng}
\end{equation}

Trong đó:
\begin{itemize}
    \item $T_{đỉnh}$: Nhiệt độ hơi tại đỉnh tháp (đo bằng cảm biến nhiệt độ đỉnh) $\approx$ 78.4°C
    \item $T_{dịch ngưng}$: Nhiệt độ dịch ngưng sau condenser (đo bằng cảm biến nhiệt độ dịch ngưng)
\end{itemize}

Ý nghĩa của $\Delta T_{subcool}$:
\begin{itemize}
    \item $\Delta T_{subcool} \approx 0$°C: Dịch ngưng ở nhiệt độ sôi, không bị làm lạnh quá mức -- vận hành tối ưu.
    \item $\Delta T_{subcool} > 0$°C: Dịch ngưng bị làm lạnh dưới nhiệt độ sôi (subcooling) -- lãng phí nước làm mát.
\end{itemize}

Kết quả dự kiến:
\begin{table}[H]
\centering
\begin{tabular}{|l|c|c|c|}
\hline
\textbf{Chế độ} & \textbf{Độ mở van} & \textbf{$\Delta T_{subcool}$ dự kiến} & \textbf{Đánh giá} \\
\hline
Baseline & 100\% & $>$ 10°C & Lãng phí \\
\hline
Tối ưu & 28\% & $\approx$ 0°C & Hiệu quả \\
\hline
\end{tabular}
\caption{Dự kiến độ subcooling theo chế độ vận hành}
\label{tab:subcool_prediction}
\end{table}

Phương pháp này cho phép kiểm chứng trực tiếp hiệu quả tiết kiệm nước làm mát thông qua phép đo nhiệt độ đơn giản.

\subsubsection{Hạn chế của phương pháp tính toán}

\begin{itemize}
    \item Mô hình đơn giản hóa, chưa xét đầy đủ các yếu tố phi tuyến.
    \item Chưa tính đến tổn thất nhiệt ra môi trường qua thành thiết bị và đường ống.
    \item Chưa xem xét ảnh hưởng của nhiễu loạn (thay đổi nồng độ nhập liệu, nhiệt độ nước làm mát).
    \item Cần kiểm chứng thực nghiệm để xác nhận kết quả tính toán.
\end{itemize}
