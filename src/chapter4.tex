% ============================================================
% Chương 4: Tính toán và kết quả
% ============================================================

\newpage
\section{TÍNH TOÁN VÀ KẾT QUẢ}

% ------------------------------------------------------------
\subsection{Thiết lập tính toán}
% ------------------------------------------------------------

\subsubsection{Công cụ tính toán}

Tính toán được thực hiện bằng ngôn ngữ Python với các thư viện:
\begin{itemize}
    \item \texttt{numpy}: Tính toán số học
    \item \texttt{scipy}: Giải phương trình, tối ưu hóa
    \item \texttt{matplotlib}: Vẽ đồ thị
\end{itemize}

\subsubsection{Thông số hệ thống}

Các thông số hệ thống được sử dụng trong tính toán:

\begin{table}[H]
\centering
\begin{tabular}{|l|c|c|}
\hline
\textbf{Thông số} & \textbf{Giá trị} & \textbf{Đơn vị} \\
\hline
Công suất reboiler tối đa & 6000 & W \\
\hline
Lưu lượng nước làm mát tối đa & 7.2 & L/min \\
\hline
Diện tích trao đổi nhiệt condenser & 0.28 & m$^2$ \\
\hline
Áp suất vận hành & 1.0 & bar \\
\hline
Nồng độ nhập liệu & 10 & \% vol \\
\hline
Lưu lượng nhập liệu & 4.8 & L/h \\
\hline
\end{tabular}
\caption{Thông số hệ thống cho tính toán}
\label{tab:simulation_params}
\end{table}

% ------------------------------------------------------------
\subsection{Tính toán cân bằng năng lượng}
% ------------------------------------------------------------

Tính toán cân bằng năng lượng để xác định công suất ngưng tụ tối ưu và lưu lượng nước làm mát tương ứng.

\subsubsection{Phương pháp McCabe-Thiele}

Từ phương pháp McCabe-Thiele [4] (xem Chương 2), với các thông số đầu vào:
\begin{itemize}
    \item Nhập liệu: 4.8 L/h, 10\% vol ethanol, 70$^\circ$C (subcooled)
    \item Sản phẩm đỉnh: 90\% vol ethanol
    \item Sản phẩm đáy: 1\% vol ethanol (giả thiết)
    \item Số mâm thực tế: 6
    \item Hiệu suất mâm giả định: 80\% (giá trị điển hình cho mâm xuyên lỗ)
    \item Số mâm lý thuyết tương đương: $6 \times 0.80 = 4.8$
\end{itemize}

Kết quả tính toán:
\begin{itemize}
    \item Hệ số q (feed quality): $q = 1.042$ (nhập liệu subcooled)
    \item Tỷ số hồi lưu tối thiểu: $R_{min} = 1.68$
    \item Tỷ số hồi lưu vận hành: $R = 10.4$ ($6.2 \times R_{min}$, xem mục 2.2)
\end{itemize}

Từ cân bằng vật chất:
\begin{align}
    D &= F \cdot \frac{x_F - x_W}{x_D - x_W} = 0.197 \text{ L/h} \\
    W &= F - D = 4.603 \text{ L/h} \\
    L &= R \cdot D = 2.05 \text{ L/h} \\
    V &= (R+1) \cdot D = 2.25 \text{ L/h}
\end{align}

\subsubsection{Cân bằng năng lượng tại reboiler}

Công suất reboiler bao gồm hai thành phần:
\begin{equation}
    Q_{reboiler} = Q_{feed} + Q_{vaporize}
\end{equation}

\textbf{Công suất reboiler hiệu dụng:}

Công suất điện đầu vào không hoàn toàn chuyển thành nhiệt hữu ích do sụt áp trên dây dẫn, tiếp điểm và tổn hao nhiệt ra môi trường. Ước tính tổn hao: 10\%.
\begin{equation}
    Q_{reboiler} = Q_{electrical} \times (1 - 0.10) = 6000 \times 0.90 = 5400 \text{ W}
\end{equation}

\textbf{Nhiệt gia nhiệt nhập liệu ($Q_{feed}$):}

Nhiệt cần thiết để nâng nhiệt độ nhập liệu từ 70$^\circ$C đến điểm sôi (92.5$^\circ$C):
\begin{equation}
    Q_{feed} = \dot{m}_F \cdot C_p \cdot \Delta T
\end{equation}
với:
\begin{itemize}
    \item $\dot{m}_F = 4.8$ L/h $\times$ 977 kg/m$^3$ / 3600 = 1.30 g/s
    \item $C_p = 4039$ J/(kg·K) (hỗn hợp 10\% ethanol)
    \item $\Delta T = 92.5 - 70 = 22.5$ °C
\end{itemize}

Kết quả: $Q_{feed} = 118$ W

\textbf{Nhiệt hóa hơi ($Q_{vaporize}$):}

Từ cân bằng năng lượng tại reboiler:
\begin{equation}
    Q_{vaporize} = Q_{reboiler} - Q_{feed} = 5400 - 118 = 5282 \text{ W}
\end{equation}

Đây là công suất nhiệt thực tế dùng để hóa hơi hỗn hợp tại đáy tháp.

\subsubsection{Công suất ngưng tụ tối ưu}

Để vận hành tối ưu, công suất ngưng tụ cần bằng với nhiệt hóa hơi:
\begin{equation}
    Q_{condenser,optimal} = Q_{vaporize} = 5282 \text{ W}
\end{equation}

Nếu $Q_{condenser} < Q_{vaporize}$: hơi không ngưng tụ hết, áp suất tháp tăng.

Nếu $Q_{condenser} > Q_{vaporize}$: dịch ngưng bị làm lạnh dưới nhiệt độ sôi (subcooling), lãng phí nước làm mát.

\subsubsection{Tính toán lưu lượng nước làm mát}

Từ phương trình truyền nhiệt [4]:
\begin{equation}
    Q = U \cdot A \cdot \Delta T_{LMTD}
\end{equation}

Với condenser coil (A = 0.28 m$^2$), hệ số truyền nhiệt U phụ thuộc vào lưu lượng:
\begin{equation}
    U = U_{ref} \cdot \left(\frac{F}{F_{ref}}\right)^{0.8}
\end{equation}

Trong đó $U_{ref}$ là hệ số truyền nhiệt tại lưu lượng tham chiếu $F_{ref} = 7.2$ L/min (van mở 100\%). Số mũ 0.8 xuất phát từ tương quan Dittus-Boelter cho dòng chảy rối trong ống ($Nu \propto Re^{0.8}$) [5].

\textbf{Lựa chọn giá trị $U_{ref}$:} Theo Engineering ToolBox (2003), hệ số truyền nhiệt tổng cho hệ thống ngưng tụ hơi hữu cơ với nước làm mát nằm trong khoảng 300--1200 W/(m²·K). Giá trị thấp (300--500) áp dụng cho thiết bị bám bẩn, cáu cặn; giá trị cao (1000--1200) cho thiết bị sạch. Hệ thống ethanol--nước là hệ sạch (không gây cáu cặn, không ăn mòn với SS304), do đó có thể sử dụng giá trị cao. Tuy nhiên, để đảm bảo tính bảo thủ trong tính toán, chọn $U_{ref} = 800$ W/(m²·K) -- giá trị trung bình của khoảng điển hình.

\textbf{Kết quả tính toán:}

\begin{table}[H]
\centering
\begin{tabular}{|l|c|c|c|}
\hline
\textbf{Chế độ} & \textbf{$Q_{condenser}$} & \textbf{Valve} & \textbf{Lưu lượng} \\
\hline
Vận hành tối ưu & 5282 W & 53\% & 3.8 L/min \\
\hline
Baseline (100\% valve) & 9113 W & 100\% & 7.2 L/min \\
\hline
\end{tabular}
\caption{Công suất ngưng tụ và lưu lượng nước làm mát theo các chế độ}
\label{tab:condenser_modes}
\end{table}

\textbf{Nhận xét:} Với công suất reboiler hiệu dụng 5400 W, lưu lượng nước làm mát tối ưu là khoảng 53\% (3.8 L/min), tiết kiệm 47\% so với mở van 100\%. Chế độ baseline (van mở 100\%) cung cấp công suất ngưng tụ 9113 W, vượt quá nhu cầu thực tế gần gấp đôi.

% ------------------------------------------------------------
\subsection{Kết quả tính toán trạng thái xác lập}
% ------------------------------------------------------------

Dựa trên các phương trình cân bằng năng lượng và truyền nhiệt, tính toán trạng thái xác lập cho hai chế độ vận hành: baseline (van mở 100\%) và vận hành tối ưu với điều khiển.

\subsubsection{Chế độ baseline (van mở 100\%)}

Trong chế độ vận hành baseline, van nước làm mát được mở hoàn toàn (100\%) -- đây là cách vận hành truyền thống không có điều khiển tự động.

Kết quả tính toán trạng thái xác lập:
\begin{itemize}
    \item Áp suất đỉnh tháp: 1.0 bar (setpoint)
    \item Van nước làm mát: 100\%
    \item Lưu lượng nước làm mát: 7.2 L/min
    \item Công suất ngưng tụ: 9113 W (vượt quá nhu cầu thực tế 5282 W)
    \item Nhiệt độ đỉnh: 78.3°C
\end{itemize}

\subsubsection{Chế độ vận hành tối ưu với điều khiển}

Với bộ điều khiển áp suất, van nước làm mát được điều chỉnh tự động để cân bằng công suất ngưng tụ với nhiệt hóa hơi.

Kết quả tính toán trạng thái xác lập:
\begin{itemize}
    \item Áp suất đỉnh tháp: 1.0 bar (duy trì ổn định)
    \item Van nước làm mát: 53\% (tiết kiệm 47\%)
    \item Lưu lượng nước làm mát: 3.8 L/min
    \item Công suất ngưng tụ: 5282 W (cân bằng với $Q_{vaporize}$)
    \item Nhiệt độ đỉnh: 78.3°C
\end{itemize}

% ------------------------------------------------------------
\subsection{So sánh hiệu quả vận hành}
% ------------------------------------------------------------

Bảng \ref{tab:sim_productivity_comparison} so sánh kết quả tính toán giữa hai phương án vận hành.

\begin{table}[H]
\centering
\begin{tabular}{|l|c|c|c|}
\hline
\textbf{Thông số} & \textbf{Baseline} & \textbf{Có điều khiển} & \textbf{Đơn vị} \\
\hline
Áp suất xác lập & 1.00 & 1.00 & bar \\
\hline
Độ mở van xác lập & 100 & 53 & \% \\
\hline
Công suất ngưng tụ & 9113 & 5282 & W \\
\hline
Lưu lượng nước làm mát & 7.2 & 3.8 & L/min \\
\hline
Nhiệt độ đỉnh tháp & 78.3 & 78.3 & °C \\
\hline
\end{tabular}
\caption{So sánh kết quả tính toán giữa chế độ baseline và có điều khiển}
\label{tab:sim_productivity_comparison}
\end{table}

\textbf{Các chỉ số cải thiện chính:}
\begin{itemize}
    \item Giảm độ mở van: từ 100\% xuống 53\% (giảm 47\%)
    \item Giảm lưu lượng nước làm mát: từ 7.2 L/min xuống 3.8 L/min (tiết kiệm 47\%)
    \item Công suất ngưng tụ cân bằng với nhiệt hóa hơi: $Q_{condenser} = Q_{vaporize} = 5282$ W
\end{itemize}

% ------------------------------------------------------------
\subsection{Thảo luận}
% ------------------------------------------------------------

\subsubsection{Phân tích kết quả tính toán}

Kết quả tính toán cho thấy:
\begin{enumerate}
    \item \textbf{Cân bằng năng lượng}: Bộ điều khiển tự động điều chỉnh độ mở van để công suất ngưng tụ (5282 W) cân bằng với nhiệt hóa hơi ($Q_{vaporize} = Q_{reboiler} - Q_{feed} = 5400 - 118 = 5282$ W). Điều này đảm bảo quá trình ngưng tụ hiệu quả mà không lãng phí nước làm mát.
    
    \item \textbf{Tiết kiệm nước làm mát}: Van làm mát giảm từ 100\% xuống 53\%, tiết kiệm 47\% lưu lượng nước làm mát. Đây là kết quả của việc sử dụng đúng lượng nước cần thiết thay vì mở van tối đa.
    
    \item \textbf{Áp suất ổn định}: Bộ điều khiển duy trì áp suất đỉnh tháp ổn định tại setpoint 1.0 bar bằng cách điều chỉnh độ mở van nước làm mát.
\end{enumerate}

\subsubsection{Giải thích cơ chế hoạt động}

\textbf{Nguyên lý vật lý quan trọng:} Hệ số truyền nhiệt $U$ phụ thuộc vào lưu lượng nước làm mát:
\begin{equation}
    Q = U \cdot A \cdot \Delta T_{LMTD}
\end{equation}
trong đó $U$ phụ thuộc vào số Reynolds ($Re$) và số Nusselt ($Nu$):
\begin{itemize}
    \item Lưu lượng cao $\rightarrow$ $Re$ cao $\rightarrow$ $Nu$ cao $\rightarrow$ $U$ cao
    \item Với dòng chảy rối: $U \propto (\text{lưu lượng})^{0.8}$
\end{itemize}

Khi van mở 100\%:
\begin{itemize}
    \item Lưu lượng nước: 7.2 L/min $\rightarrow$ $U = 800$ W/(m²·K)
    \item Công suất ngưng tụ: $Q_c = 9113$ W $>$ $Q_{vaporize} = 5282$ W
    \item Dư thừa ~3800 W công suất ngưng tụ -- lãng phí 47\% lưu lượng nước.
\end{itemize}

Với điều khiển (van 53\%):
\begin{itemize}
    \item Lưu lượng nước: 3.8 L/min $\rightarrow$ $U = 489$ W/(m²·K)
    \item Công suất ngưng tụ: $Q_c = 5282$ W $=$ $Q_{vaporize}$
    \item Cân bằng năng lượng tối ưu -- tiết kiệm 47\% nước làm mát.
\end{itemize}

\subsubsection{Tính toán năng suất từ lưu lượng hồi lưu}

Lưu lượng sản phẩm đỉnh được tính gián tiếp qua lưu lượng hồi lưu (đo được qua cảm biến lưu lượng):
\begin{equation}
    D = \frac{L}{R}
\end{equation}

Trong đó:
\begin{itemize}
    \item $D$: Lưu lượng sản phẩm đỉnh (L/h)
    \item $L$: Lưu lượng hồi lưu (L/h) -- đo trực tiếp
    \item $R$: Tỷ số hồi lưu -- xác định từ thiết kế
\end{itemize}

\subsubsection{Phương pháp kiểm chứng thực nghiệm}

Để kiểm chứng hiệu quả của chiến lược điều khiển, sử dụng độ chênh nhiệt độ giữa đỉnh tháp và dịch ngưng:
\begin{equation}
    \Delta T_{subcool} = T_{đỉnh} - T_{dịch ngưng}
\end{equation}

Trong đó:
\begin{itemize}
    \item $T_{đỉnh}$: Nhiệt độ hơi tại đỉnh tháp (đo bằng cảm biến nhiệt độ đỉnh) $\approx$ 78.4°C
    \item $T_{dịch ngưng}$: Nhiệt độ dịch ngưng sau condenser (đo bằng cảm biến nhiệt độ dịch ngưng)
\end{itemize}

Ý nghĩa của $\Delta T_{subcool}$:
\begin{itemize}
    \item $\Delta T_{subcool} \approx 0$°C: Dịch ngưng ở nhiệt độ sôi, không bị làm lạnh quá mức -- vận hành tối ưu.
    \item $\Delta T_{subcool} > 0$°C: Dịch ngưng bị làm lạnh dưới nhiệt độ sôi (subcooling) -- lãng phí nước làm mát.
\end{itemize}

Kết quả dự kiến:
\begin{table}[H]
\centering
\begin{tabular}{|l|c|c|c|}
\hline
\textbf{Chế độ} & \textbf{Độ mở van} & \textbf{$\Delta T_{subcool}$ dự kiến} & \textbf{Đánh giá} \\
\hline
Baseline & 100\% & $>$ 10°C & Lãng phí \\
\hline
Tối ưu & 53\% & $\approx$ 0°C & Hiệu quả \\
\hline
\end{tabular}
\caption{Dự kiến độ subcooling theo chế độ vận hành}
\label{tab:subcool_prediction}
\end{table}

Phương pháp này cho phép kiểm chứng trực tiếp hiệu quả tiết kiệm năng lượng làm mát thông qua phép đo nhiệt độ đơn giản.

\subsubsection{Hạn chế của phương pháp tính toán}

\begin{itemize}
    \item Mô hình đơn giản hóa, chưa xét đầy đủ các yếu tố phi tuyến.
    \item Chưa xem xét ảnh hưởng của nhiễu loạn (thay đổi nồng độ nhập liệu, nhiệt độ nước làm mát).
    \item Cần kiểm chứng thực nghiệm để xác nhận kết quả tính toán.
\end{itemize}
