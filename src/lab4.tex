% ============================================================
% Lab 4: Factory I/O
% ============================================================

\setcounter{section}{4}
\setcounter{subsection}{0}
\section*{BÀI THÍ NGHIỆM 4: FACTORY I/O}
\addcontentsline{toc}{section}{Bài thí nghiệm 4: Factory I/O}

% ------------------------------------------------------------
\subsection{Mục tiêu}
% ------------------------------------------------------------

\begin{itemize}
    \item Làm quen với phần mềm mô phỏng Factory I/O.
    \item Kết nối PLC S7-1200 với Factory I/O qua giao thức S7.
    \item Lập trình điều khiển và giám sát hệ thống sản xuất ảo.
\end{itemize}

% ------------------------------------------------------------
\subsection{Cơ sở lý thuyết}
% ------------------------------------------------------------

\subsubsection{Giới thiệu Factory I/O}

Factory I/O là phần mềm mô phỏng 3D các hệ thống tự động hóa công nghiệp. Phần mềm cho phép:
\begin{itemize}
    \item Xây dựng các kịch bản sản xuất ảo với các thiết bị công nghiệp.
    \item Kết nối với PLC thực hoặc mô phỏng để điều khiển.
    \item Kiểm tra và gỡ lỗi chương trình điều khiển trong môi trường an toàn.
\end{itemize}

\subsubsection{Các thành phần trong Factory I/O}

\begin{itemize}
    \item \textbf{Sensors (Cảm biến):} Proximity sensor, photoelectric sensor, vision sensor...
    \item \textbf{Actuators (Cơ cấu chấp hành):} Conveyor, pusher, robot arm, elevator...
    \item \textbf{Indicators:} Stack light, digital display, alarm...
    \item \textbf{Stations:} Pick \& place, sorting, palletizing...
\end{itemize}

\subsubsection{Kết nối với PLC S7-1200}

Factory I/O hỗ trợ kết nối với PLC Siemens qua giao thức S7 (TCP/IP). Các bước kết nối:
\begin{enumerate}
    \item Cấu hình địa chỉ IP của PLC trong TIA Portal.
    \item Thiết lập Driver trong Factory I/O là ``Siemens S7-1200/1500''.
    \item Map các biến I/O giữa PLC và Factory I/O.
\end{enumerate}

% ------------------------------------------------------------
\subsection{Thiết bị thí nghiệm}
% ------------------------------------------------------------

\begin{itemize}
    \item Máy tính cài đặt Factory I/O (phiên bản Education hoặc Ultimate)
    \item PLC Siemens S7-1200 CPU 1214C
    \item Cáp Ethernet kết nối PLC với máy tính
    \item Phần mềm TIA Portal V16
\end{itemize}

% ------------------------------------------------------------
\subsection{Quy trình thực hiện}
% ------------------------------------------------------------

\subsubsection{Cấu hình kết nối}

\begin{enumerate}
    \item Trong TIA Portal, cấu hình địa chỉ IP cho PLC (ví dụ: 192.168.0.1).
    \item Mở Factory I/O, vào \texttt{File > Driver Configuration}.
    \item Chọn Driver: \texttt{Siemens S7-1200/1500}.
    \item Nhập địa chỉ IP của PLC và cấu hình vùng nhớ I/O.
    \item Kiểm tra kết nối bằng nút \texttt{Connect}.
\end{enumerate}

\subsubsection{Xây dựng kịch bản mô phỏng}

\begin{enumerate}
    \item Chọn một scene có sẵn hoặc tạo scene mới.
    \item Thêm các thiết bị: băng tải, cảm biến, pusher...
    \item Cấu hình các tag I/O cho từng thiết bị.
    \item Map các tag với địa chỉ PLC tương ứng.
\end{enumerate}

\subsubsection{Lập trình PLC}

\begin{enumerate}
    \item Tạo chương trình điều khiển trong TIA Portal.
    \item Sử dụng các khối logic: AND, OR, Timer, Counter...
    \item Download chương trình xuống PLC.
    \item Chạy mô phỏng và quan sát hoạt động của hệ thống.
\end{enumerate}

\subsubsection{Kịch bản thí nghiệm}

% TODO: Mô tả kịch bản cụ thể (sorting, pick & place, conveyor control...)

% ------------------------------------------------------------
\subsection{Kết quả thí nghiệm}
% ------------------------------------------------------------

\subsubsection{Cấu hình hệ thống}

\begin{table}[H]
\centering
\begin{tabular}{|l|l|l|}
\hline
\textbf{Thiết bị} & \textbf{Tag Factory I/O} & \textbf{Địa chỉ PLC} \\
\hline
Conveyor 1 & & \\
\hline
Sensor 1 & & \\
\hline
Pusher 1 & & \\
\hline
Stack Light & & \\
\hline
\end{tabular}
\caption{Bảng cấu hình I/O}
\end{table}

\subsubsection{Chương trình điều khiển}

% TODO: Chèn hình chụp chương trình Ladder hoặc mô tả logic điều khiển

\subsubsection{Kết quả vận hành}

% TODO: Mô tả kết quả quan sát được khi chạy mô phỏng

% ------------------------------------------------------------
\subsection{Bàn luận}
% ------------------------------------------------------------

\begin{itemize}
    \item Đánh giá ưu điểm của việc sử dụng mô phỏng Factory I/O trong học tập.
    \item So sánh với thực hành trên thiết bị thực.
    \item Nhận xét về khả năng áp dụng vào các hệ thống công nghiệp thực tế.
\end{itemize}

% ------------------------------------------------------------
\subsection{Kết luận}
% ------------------------------------------------------------

% TODO: Điền kết luận sau khi thực hiện thí nghiệm

% ------------------------------------------------------------
\subsection{Câu hỏi kiểm tra}
% ------------------------------------------------------------

\textbf{Câu 1: Factory I/O là gì? Ứng dụng trong đào tạo tự động hóa?}

% TODO: Trả lời

\textbf{Câu 2: Mô tả các bước kết nối PLC S7-1200 với Factory I/O?}

% TODO: Trả lời

\textbf{Câu 3: So sánh ưu nhược điểm của mô phỏng so với thực hành trên thiết bị thực?}

% TODO: Trả lời

% ------------------------------------------------------------
\subsection{Tài liệu tham khảo}
% ------------------------------------------------------------

\begin{enumerate}[label={[\arabic*]}]
    \item B. N. Pha, \textit{Thiết bị Đo lường và Điều khiển}. Trường Đại học Bách khoa, ĐHQG-HCM.
    \item Real Games, \textit{Factory I/O Manual}. https://factoryio.com/docs/
    \item Siemens, \textit{S7-1200 Programmable Controller System Manual}. Siemens AG, 2020.
\end{enumerate}

\newpage
