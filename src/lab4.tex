% ============================================================
% Lab 4: Khảo sát ảnh hưởng các thông số bộ điều khiển PID
% ============================================================

\setcounter{section}{4}
\section*{BÀI THÍ NGHIỆM 4: KHẢO SÁT ẢNH HƯỞNG CÁC THÔNG SỐ BỘ ĐIỀU KHIỂN PID}
\addcontentsline{toc}{section}{Bài thí nghiệm 4: Khảo sát ảnh hưởng các thông số bộ điều khiển PID}

% ------------------------------------------------------------
\subsection{Tóm tắt}
% ------------------------------------------------------------

Bài thí nghiệm này nhằm khảo sát ảnh hưởng của các thông số $K_P$, $K_I$, $K_D$ của bộ điều khiển PID đến chất lượng điều khiển. Thông qua việc thay đổi từng thông số một cách độc lập và ghi nhận đáp ứng của hệ thống, sinh viên hiểu rõ hơn về vai trò của từng thành phần trong bộ điều khiển PID. Ngoài ra, bài thí nghiệm còn giới thiệu phương pháp Ziegler-Nichols để thiết kế bộ điều khiển PID dựa trên đáp ứng quá độ của hệ hở.

% ------------------------------------------------------------
\subsection{Cơ sở lý thuyết}
% ------------------------------------------------------------

\subsubsection{Ảnh hưởng các thông số bộ điều khiển}

Các thông số $K_P$, $K_I$, $K_D$ của bộ điều khiển PID có ảnh hưởng đặc trưng đến chất lượng điều khiển của hệ thống. Bảng sau tóm tắt ảnh hưởng của việc tăng từng thông số:

\begin{table}[H]
\centering
\caption{Ảnh hưởng của các thông số PID đến chất lượng điều khiển}
\begin{tabular}{|c|c|c|c|c|}
\hline
\textbf{Tăng} & \textbf{Thời gian đáp ứng} & \textbf{Độ quá điều chỉnh} & \textbf{Sai lệch tĩnh} & \textbf{Ổn định} \\
\hline
$K_P$ & Giảm & Tăng & Giảm & Xấu đi \\
\hline
$K_I$ & Giảm & Tăng & Triệt tiêu & Xấu đi \\
\hline
$K_D$ & Ít thay đổi & Giảm & Không đổi & Tốt hơn \\
\hline
\end{tabular}
\end{table}

\subsubsection{Thiết kế bộ điều khiển theo Ziegler-Nichols 1 (ZN1)}

Phương pháp Ziegler-Nichols 1 dựa trên đáp ứng quá độ của hệ hở đối với tín hiệu bước. Từ đáp ứng, xác định các thông số:

\begin{itemize}
    \item $K$: Độ lợi tĩnh của quá trình
    \item $L$: Thời gian trễ (delay time)
    \item $T$: Hằng số thời gian (time constant)
\end{itemize}

Các thông số bộ điều khiển PID được tính theo công thức:

\begin{table}[H]
\centering
\caption{Công thức tính thông số PID theo Ziegler-Nichols 1}
\begin{tabular}{|c|c|c|c|}
\hline
\textbf{Bộ điều khiển} & $K_P$ & $T_I$ & $T_D$ \\
\hline
P & $\frac{T}{KL}$ & -- & -- \\
\hline
PI & $\frac{0.9T}{KL}$ & $\frac{L}{0.3}$ & -- \\
\hline
PID & $\frac{1.2T}{KL}$ & $2L$ & $0.5L$ \\
\hline
\end{tabular}
\end{table}

% ------------------------------------------------------------
\subsection{Mô tả thiết bị thí nghiệm}
% ------------------------------------------------------------

Với bài thí nghiệm này, sinh viên thực hiện thí nghiệm tối thiểu trên 01 trong số 05 hệ thống thiết bị điều khiển các đại lượng công nghệ cơ bản trong phòng thí nghiệm:

\begin{enumerate}
    \item Hệ thống điều chỉnh tự động mức chất lỏng
    \item Hệ thống điều chỉnh tự động nhiệt độ
    \item Hệ thống điều chỉnh tự động lưu lượng
    \item Hệ thống điều chỉnh tự động áp suất
    \item Hệ thống điều chỉnh tự động nồng độ
\end{enumerate}

Trên giao diện điều khiển, sinh viên sử dụng các chức năng:
\begin{itemize}
    \item Đồ thị hiển thị giá trị đại lượng cần điều khiển theo thời gian thực
    \item Giá trị cài đặt của đại lượng cần điều khiển
    \item Chế độ Tune để khảo sát các thông số bộ điều khiển PID
    \item Ô nhập các thông số bộ điều khiển PID ($K_P$, $K_I$, $K_D$)
\end{itemize}

% ------------------------------------------------------------
\subsection{Quy trình thực hiện}
% ------------------------------------------------------------

\subsubsection{Chuẩn bị thí nghiệm}
\begin{enumerate}
    \item Khởi động hệ thống thiết bị và phần mềm điều khiển.
    \item Kiểm tra các kết nối và trạng thái hoạt động của thiết bị.
    \item Nhập giá trị biến cần điều khiển mong muốn $y_{sp}$.
    \item Chọn chế độ Tune trên giao diện điều khiển.
\end{enumerate}

\subsubsection{Thí nghiệm 1: Khảo sát ảnh hưởng các thông số bộ điều khiển PID}

\textbf{a) Khảo sát ảnh hưởng của tham số $K_P$:}
\begin{enumerate}
    \item Giữ $K_I$ và $K_D$ cố định.
    \item Thay đổi $K_P$ theo các giá trị: 1, 2, 5, 20.
    \item Ghi nhận thời gian xác lập, sai lệch xác lập, độ quá điều chỉnh.
\end{enumerate}

\textbf{b) Khảo sát ảnh hưởng của tham số $K_I$:}
\begin{enumerate}
    \item Giữ $K_P$ và $K_D$ cố định.
    \item Thay đổi $K_I$ theo các giá trị: 0.5, 1, 1.5, 2.
    \item Ghi nhận các chỉ tiêu chất lượng.
\end{enumerate}

\textbf{c) Khảo sát ảnh hưởng của tham số $K_D$:}
\begin{enumerate}
    \item Giữ $K_P$ và $K_I$ cố định.
    \item Thay đổi $K_D$ theo các giá trị: 0.5, 1, 2, 5.
    \item Ghi nhận các chỉ tiêu chất lượng.
\end{enumerate}

\subsubsection{Thí nghiệm 2: Thiết kế bộ điều khiển PID theo Ziegler-Nichols 1}

\begin{enumerate}
    \item Đưa hệ thống về chế độ điều khiển tay (Manual).
    \item Đặt tín hiệu bước vào hệ thống và ghi nhận đáp ứng quá độ.
    \item Xác định các thông số $K$, $L$, $T$ từ đáp ứng.
    \item Tính toán thông số PID theo công thức ZN1.
    \item Cài đặt và đánh giá chất lượng điều khiển.
\end{enumerate}

% ------------------------------------------------------------
\subsection{Kết quả và bàn luận}
% ------------------------------------------------------------

\subsubsection{Kết quả thí nghiệm 1: Khảo sát ảnh hưởng thông số PID}

\textbf{a) Khảo sát ảnh hưởng của $K_P$:}

\begin{table}[H]
\centering
\caption{Kết quả khảo sát ảnh hưởng của $K_P$}
\begin{tabular}{|c|c|c|c|c|}
\hline
$K_P$ & 1 & 2 & 5 & 20 \\
\hline
Thời gian xác lập (s) & & & & \\
\hline
Sai lệch xác lập & & & & \\
\hline
Độ quá điều chỉnh (\%) & & & & \\
\hline
\end{tabular}
\end{table}

\textbf{b) Khảo sát ảnh hưởng của $K_I$:}

\begin{table}[H]
\centering
\caption{Kết quả khảo sát ảnh hưởng của $K_I$}
\begin{tabular}{|c|c|c|c|c|}
\hline
$K_I$ & 0.5 & 1 & 1.5 & 2 \\
\hline
Thời gian xác lập (s) & & & & \\
\hline
Sai lệch xác lập & & & & \\
\hline
Độ quá điều chỉnh (\%) & & & & \\
\hline
\end{tabular}
\end{table}

\textbf{c) Khảo sát ảnh hưởng của $K_D$:}

\begin{table}[H]
\centering
\caption{Kết quả khảo sát ảnh hưởng của $K_D$}
\begin{tabular}{|c|c|c|c|c|}
\hline
$K_D$ & 0.5 & 1 & 2 & 5 \\
\hline
Thời gian xác lập (s) & & & & \\
\hline
Sai lệch xác lập & & & & \\
\hline
Độ quá điều chỉnh (\%) & & & & \\
\hline
\end{tabular}
\end{table}

\textbf{d) Bộ thông số tối ưu:}

\begin{table}[H]
\centering
\caption{Kết quả với bộ thông số tối ưu}
\begin{tabular}{|l|c|}
\hline
\textbf{Thông số} & \textbf{Giá trị} \\
\hline
$K_P$ = , $K_I$ = , $K_D$ = & \\
\hline
Thời gian xác lập (s) & \\
\hline
Sai lệch xác lập & \\
\hline
Độ quá điều chỉnh (\%) & \\
\hline
\end{tabular}
\end{table}

\subsubsection{Kết quả thí nghiệm 2: Thiết kế theo Ziegler-Nichols 1}

Từ đáp ứng quá độ của hệ hở, xác định được:
\begin{itemize}
    \item Độ lợi tĩnh $K$ = 
    \item Thời gian trễ $L$ = 
    \item Hằng số thời gian $T$ = 
\end{itemize}

Thông số PID tính theo ZN1:
\begin{itemize}
    \item $K_P$ = $\frac{1.2T}{KL}$ = 
    \item $T_I$ = $2L$ = 
    \item $T_D$ = $0.5L$ = 
\end{itemize}

\begin{table}[H]
\centering
\caption{Kết quả với bộ thông số ZN1}
\begin{tabular}{|l|c|}
\hline
\textbf{Thông số} & \textbf{Giá trị} \\
\hline
$K_P$ = , $K_I$ = , $K_D$ = & \\
\hline
Thời gian xác lập (s) & \\
\hline
Sai lệch xác lập & \\
\hline
Độ quá điều chỉnh (\%) & \\
\hline
\end{tabular}
\end{table}

\subsubsection{Bàn luận}

\begin{itemize}
    \item Khi tăng $K_P$: Hệ thống đáp ứng nhanh hơn nhưng có thể xuất hiện dao động nếu $K_P$ quá lớn.
    \item Khi tăng $K_I$: Sai lệch tĩnh được triệt tiêu nhưng có thể gây dao động và hiện tượng integral windup.
    \item Khi tăng $K_D$: Độ quá điều chỉnh giảm nhưng hệ thống nhạy hơn với nhiễu.
    \item Phương pháp ZN1 cho kết quả ban đầu hợp lý, tuy nhiên cần tinh chỉnh thêm trong thực tế.
\end{itemize}

% ------------------------------------------------------------
\subsection{Kết luận và khuyến nghị}
% ------------------------------------------------------------

\textbf{Kết luận:}
\begin{itemize}
    \item Đã khảo sát được ảnh hưởng của từng thông số $K_P$, $K_I$, $K_D$ đến chất lượng điều khiển.
    \item Xác định được quy luật điều khiển và chỉnh định các thông số PID cho phù hợp.
    \item Phương pháp Ziegler-Nichols 1 là phương pháp thực nghiệm hiệu quả để xác định các thông số PID ban đầu.
\end{itemize}

\textbf{Khuyến nghị:}
\begin{itemize}
    \item Trong thực tế, cần tinh chỉnh thêm dựa trên yêu cầu cụ thể của quá trình.
    \item Nên kết hợp với các phương pháp khác như Cohen-Coon hoặc auto-tuning để có kết quả tốt hơn.
\end{itemize}

% ------------------------------------------------------------
\subsection{Trả lời câu hỏi kiểm tra}
% ------------------------------------------------------------

\textbf{Câu 1: Vẽ sơ đồ khối tổng quát về các biến quá trình, sơ đồ khối và lưu đồ điều khiển?}

% Sơ đồ khối hệ thống điều khiển PID
\begin{figure}[H]
\centering
\begin{tikzpicture}[node distance=1.5cm]
    % Nodes
    \node (sp) {$Y_{SP}$};
    \node[sum, right=0.8cm of sp] (sum) {};
    \node[block, right=1cm of sum, minimum width=4em] (pid) {PID};
    \node[block, right=1.2cm of pid, minimum width=4em] (actuator) {Van/Bơm};
    \node[block, right=1.2cm of actuator, minimum width=4em] (process) {Quá trình};
    \node[right=1.2cm of process] (output) {$Y$};
    
    % Disturbance
    \node[above=0.8cm of process] (dist) {Nhiễu $Z$};
    
    % Sensor
    \node[block, below=1cm of actuator, minimum width=4em] (sensor) {Cảm biến};
    
    % Arrows
    \draw[arrow] (sp) -- (sum);
    \draw[arrow] (sum) -- node[labelnode, above] {$e$} (pid);
    \draw[arrow] (pid) -- node[labelnode, above] {$p$} (actuator);
    \draw[arrow] (actuator) -- node[labelnode, above] {$m$} (process);
    \draw[arrow] (process) -- (output);
    \draw[arrow] (dist) -- (process);
    
    % Feedback
    \coordinate (fb) at ($(process.east)!0.6!(output.west)$);
    \draw[line] (fb) |- (sensor);
    \draw[arrow] (sensor) -| (sum);
    
    % Sum signs - positioned in quadrants between cross lines
    \node[font=\tiny] at ($(sum.north west)+(0.12,-0.12)$) {$+$};
    \node[font=\tiny] at ($(sum.south west)+(0.12,0.12)$) {$-$};
\end{tikzpicture}
\caption{Sơ đồ khối hệ thống điều khiển PID}
\label{fig:pid_system}
\end{figure}

\textbf{Trong đó:}
\begin{itemize}
    \item $Y_{SP}$ - Giá trị đặt (Setpoint)
    \item $e$ - Sai lệch điều khiển ($e = Y_{SP} - Y$)
    \item $p$ - Tín hiệu điều khiển từ bộ PID
    \item $m$ - Biến điều khiển (Manipulated Variable)
    \item $Y$ - Biến được điều khiển (Controlled Variable)
    \item $Z$ - Biến nhiễu (Disturbance)
\end{itemize}

\textbf{Câu 2: Mô tả các thành phần của hệ thống điều khiển trong bài thí nghiệm này?}

\begin{itemize}
    \item Bộ điều khiển: PLC/DCS với thuật toán PID
    \item Thiết bị đo: Cảm biến mức/nhiệt độ/lưu lượng/áp suất
    \item Thiết bị chấp hành: Van điều khiển, bơm
    \item Giao diện: HMI hiển thị và cài đặt thông số
\end{itemize}

\textbf{Câu 3: Nhận xét ảnh hưởng các thông số bộ điều khiển đến chất lượng điều khiển?}

\begin{itemize}
    \item $K_P$ ảnh hưởng chủ yếu đến tốc độ đáp ứng và sai lệch tĩnh.
    \item $K_I$ quyết định khả năng triệt tiêu sai lệch tĩnh.
    \item $K_D$ giúp giảm độ quá điều chỉnh và tăng tốc độ đáp ứng.
    \item Cần cân bằng giữa các thông số để đạt chất lượng điều khiển tối ưu.
\end{itemize}

% ------------------------------------------------------------
\subsection{Tài liệu tham khảo}
% ------------------------------------------------------------

\begin{enumerate}
    \item Điều khiển Quá trình Công nghệ Hóa học - Cơ sở điều khiển Quá trình – Quyển 1.
    \item Điều khiển Quá trình Công nghệ Hóa học – Hướng dẫn thí nghiệm, Thực hành cơ sở Điều khiển – Quyển 2.
    \item Ziegler, J.G. and Nichols, N.B., "Optimum Settings for Automatic Controllers", Trans. ASME, 64, 759-768, 1942.
\end{enumerate}
