% ============================================================
% Lab 4: Factory I/O
% ============================================================

\setcounter{section}{4}
\setcounter{subsection}{0}
\section*{BÀI THÍ NGHIỆM 4: FACTORY I/O}
\addcontentsline{toc}{section}{Bài thí nghiệm 4: Factory I/O}

% ------------------------------------------------------------
\subsection{Mục tiêu}
% ------------------------------------------------------------

\begin{itemize}
    \item Làm quen với phần mềm mô phỏng Factory I/O.
    \item Kết nối PLCSIM với Factory I/O qua giao thức S7.
    \item Thực hiện điều khiển mức nước (Level Control) với bộ điều khiển PID.
    \item Chỉnh định thông số PID và đánh giá chất lượng điều khiển.
\end{itemize}

% ------------------------------------------------------------
\subsection{Cơ sở lý thuyết}
% ------------------------------------------------------------

\subsubsection{Giới thiệu Factory I/O}

Factory I/O là phần mềm mô phỏng 3D các hệ thống tự động hóa công nghiệp. Phần mềm cho phép:
\begin{itemize}
    \item Xây dựng các kịch bản sản xuất ảo với các thiết bị công nghiệp.
    \item Kết nối với PLC thực hoặc mô phỏng để điều khiển.
    \item Kiểm tra và gỡ lỗi chương trình điều khiển trong môi trường an toàn.
\end{itemize}

\subsubsection{Các thành phần trong Factory I/O}

\begin{itemize}
    \item \textbf{Sensors (Cảm biến):} Proximity sensor, photoelectric sensor, vision sensor...
    \item \textbf{Actuators (Cơ cấu chấp hành):} Conveyor, pusher, robot arm, elevator...
    \item \textbf{Indicators:} Stack light, digital display, alarm...
    \item \textbf{Stations:} Pick \& place, sorting, palletizing...
\end{itemize}

\subsubsection{Kết nối với PLCSIM}

Factory I/O hỗ trợ kết nối với PLCSIM (PLC mô phỏng) qua giao thức S7 (TCP/IP). Các bước kết nối:
\begin{enumerate}
    \item Tạo project trong TIA Portal với PLC S7-1200/1500.
    \item Khởi chạy PLCSIM và load chương trình.
    \item Thiết lập Driver trong Factory I/O là ``Siemens S7-PLCSIM''.
    \item Map các biến I/O giữa PLCSIM và Factory I/O.
\end{enumerate}

% ------------------------------------------------------------
\subsection{Thiết bị thí nghiệm}
% ------------------------------------------------------------

\begin{itemize}
    \item Máy tính cài đặt Factory I/O (phiên bản Education hoặc Ultimate)
    \item Phần mềm TIA Portal V16 với PLCSIM
\end{itemize}

% ------------------------------------------------------------
\subsection{Quy trình thực hiện}
% ------------------------------------------------------------

\subsubsection{Cấu hình kết nối}

\begin{enumerate}
    \item Trong TIA Portal, tạo project với PLC S7-1200/1500.
    \item Khởi chạy PLCSIM từ TIA Portal.
    \item Mở Factory I/O, vào \texttt{File > Driver Configuration}.
    \item Chọn Driver: \texttt{Siemens S7-PLCSIM}.
    \item Cấu hình vùng nhớ I/O và kiểm tra kết nối bằng nút \texttt{Connect}.
\end{enumerate}

\subsubsection{Xây dựng kịch bản mô phỏng}

\begin{enumerate}
    \item Chọn một scene có sẵn hoặc tạo scene mới.
    \item Thêm các thiết bị: băng tải, cảm biến, pusher...
    \item Cấu hình các tag I/O cho từng thiết bị.
    \item Map các tag với địa chỉ PLC tương ứng.
\end{enumerate}

\subsubsection{Lập trình PLC}

\begin{enumerate}
    \item Tạo chương trình điều khiển trong TIA Portal.
    \item Sử dụng các khối logic: AND, OR, Timer, Counter...
    \item Download chương trình xuống PLCSIM.
    \item Chạy mô phỏng và quan sát hoạt động của hệ thống.
\end{enumerate}

\subsubsection{Kịch bản thí nghiệm: Điều khiển mức bồn chứa qua van cấp}

Sử dụng scene \texttt{Level Control} trong Factory I/O để thực hiện điều khiển mức nước trong bồn chứa thông qua van cấp nước:

\begin{itemize}
    \item \textbf{Mục tiêu:} Duy trì mức nước ở giá trị đặt (setpoint = 100cm)
    \item \textbf{Biến điều khiển (MV):} Độ mở van cấp nước (0-100\%)
    \item \textbf{Biến được điều khiển (PV):} Mức nước trong bồn (0-100cm)
    \item \textbf{Nhiễu tải:} Van xả cố định ở mức 4.1, tạo dòng chảy ra liên tục
    \item \textbf{Thuật toán:} PID\_Compact trong TIA Portal
\end{itemize}

Nguyên lý hoạt động: Bộ điều khiển PID so sánh mức nước đo được (PV) với giá trị đặt (SV), tính toán sai lệch và điều chỉnh độ mở van cấp nước để bù đắp lượng nước chảy ra qua van xả, duy trì mức nước ổn định.

Quy trình thí nghiệm:
\begin{enumerate}
    \item Chỉnh định P-only: tăng $K_P$ từ 4 đến 7.4 để tìm $K_u$
    \item Xác định chu kỳ dao động $T_u$ từ đáp ứng P-only
    \item Tính toán thông số PID theo Ziegler-Nichols
    \item Thử nghiệm PI và tinh chỉnh $T_i$
    \item Thử nghiệm PID đầy đủ với step response
\end{enumerate}

% ------------------------------------------------------------
\subsection{Kết quả thí nghiệm}
% ------------------------------------------------------------

\subsubsection{Giao diện HMI và Factory I/O}

\begin{figure}[H]
\centering
\includegraphics[width=0.9\textwidth]{../assets/HMI_n_FactoryIO_TN4.jpg}
\caption{Giao diện HMI (phía trước) và mô phỏng Factory I/O (phía sau) - Hệ thống điều khiển mức bồn chứa qua van cấp nước}
\label{fig:hmi_tn4}
\end{figure}

Hệ thống điều khiển mức bồn chứa trong Factory I/O bao gồm:
\begin{itemize}
    \item Bồn chứa nước với cảm biến mức analog (0-100cm)
    \item Van cấp nước (Fill Valve) điều khiển bằng tín hiệu analog 0-100\%
    \item Van xả (Discharge Valve) cố định ở mức 4.1, đóng vai trò nhiễu tải
    \item Bộ điều khiển PID điều chỉnh độ mở van cấp để duy trì mức đặt
\end{itemize}

\subsubsection{Cấu hình hệ thống}

\begin{table}[H]
\centering
\begin{tabular}{|l|l|}
\hline
\textbf{Thiết bị} & \textbf{Mô tả} \\
\hline
Cảm biến mức (Level Sensor) & Đo mức nước 0-100cm (Analog Input) \\
\hline
Van cấp nước (Fill Valve) & Điều khiển lưu lượng cấp 0-100\% (Analog Output) \\
\hline
Van xả (Discharge Valve) & Cố định ở mức 4.1, đóng vai trò nhiễu tải \\
\hline
\end{tabular}
\caption{Cấu hình thiết bị hệ thống điều khiển mức bồn chứa}
\end{table}

\subsubsection{Quá trình chỉnh định PID}

Thí nghiệm sử dụng phương pháp Ziegler-Nichols để chỉnh định thông số PID. Quá trình gồm các bước:

\textbf{Bước 1: Xác định $K_u$ (Ultimate Gain)}

Tắt thành phần I và D ($T_i = 0$, $T_d = 0$), tăng dần $K_P$ cho đến khi hệ thống bắt đầu dao động tới hạn:

\begin{table}[H]
\centering
\begin{tabular}{|c|c|c|c|}
\hline
\textbf{$K_P$} & \textbf{Mức trung bình (cm)} & \textbf{Sai lệch tĩnh (cm)} & \textbf{Nhận xét} \\
\hline
4.0 & 92.02 & 7.98 & Ổn định, sai lệch lớn \\
\hline
5.0 & 93.75 & 6.25 & Ổn định, sai lệch giảm \\
\hline
6.0 & 94.89 & 5.11 & Ổn định, sai lệch giảm \\
\hline
7.0 & 95.40 & 4.60 & Bắt đầu dao động nhẹ \\
\hline
7.2 & 95.50 & 4.50 & Dao động rõ hơn \\
\hline
7.4 & 95.61 & 4.39 & Dao động tới hạn ($K_u$) \\
\hline
\end{tabular}
\caption{Kết quả khảo sát $K_P$ để tìm $K_u$}
\end{table}

Từ kết quả, xác định: $K_u = 7.4$, chu kỳ dao động $T_u \approx 10$s.

\textbf{Bước 2: Tính thông số PID theo Ziegler-Nichols}

\begin{align}
K_P &= 0.6 \times K_u = 0.6 \times 7.4 = 4.44 \\
T_i &= \frac{T_u}{2} = \frac{10}{2} = 5 \text{ s (điều chỉnh lên 7.5s)} \\
T_d &= \frac{T_u}{8} = \frac{10}{8} = 1.25 \text{ s (điều chỉnh lên 1.875s)}
\end{align}

\textbf{Bộ thông số PID sử dụng:} $K_P = 4.44$, $T_i = 7.5$s, $T_d = 1.875$s

\subsubsection{Đáp ứng điều khiển mức}

\textbf{Giai đoạn chỉnh định (Tuning):}

\begin{figure}[H]
\centering
\begin{tikzpicture}
\begin{axis}[
    width=0.95\textwidth,
    height=7cm,
    xlabel={Thời gian (s)},
    ylabel={Mức nước (cm)},
    grid=both,
    grid style={line width=.1pt, draw=gray!30},
    major grid style={line width=.2pt,draw=gray!50},
    xmin=0, xmax=460,
    ymin=88, ymax=102,
    legend pos=south east,
    restrict x to domain=0:460,
]
\addplot[blue, thick] table[x=time_s, y=PV, col sep=comma] {../data/TN4_processed.csv};
\addlegendentry{PV (Mức đo được)}
\addplot[red, dashed, thick] table[x=time_s, y=SV, col sep=comma] {../data/TN4_processed.csv};
\addlegendentry{SV (Giá trị đặt)}

% Đánh dấu các giai đoạn Kp
\draw[gray, dashed] (axis cs:37,88) -- (axis cs:37,102);
\node[font=\scriptsize] at (axis cs:18,101) {$K_P=4$};

\draw[gray, dashed] (axis cs:62,88) -- (axis cs:62,102);
\node[font=\scriptsize] at (axis cs:50,101) {$5$};

\draw[gray, dashed] (axis cs:92,88) -- (axis cs:92,102);
\node[font=\scriptsize] at (axis cs:77,101) {$6$};

\draw[gray, dashed] (axis cs:163,88) -- (axis cs:163,102);
\node[font=\scriptsize] at (axis cs:128,101) {$7$};

\draw[gray, dashed] (axis cs:259,88) -- (axis cs:259,102);
\node[font=\scriptsize] at (axis cs:211,101) {$7.2$};

\node[font=\scriptsize] at (axis cs:360,101) {$7.4$ ($K_u$)};

\end{axis}
\end{tikzpicture}
\caption{Giai đoạn chỉnh định: Khảo sát $K_P$ để xác định $K_u$}
\label{fig:level_tuning_tn4}
\end{figure}

\textbf{Giai đoạn kiểm tra (Test):}

\begin{figure}[H]
\centering
\begin{tikzpicture}
\begin{axis}[
    width=0.95\textwidth,
    height=7cm,
    xlabel={Thời gian (s)},
    ylabel={Mức nước (cm)},
    grid=both,
    grid style={line width=.1pt, draw=gray!30},
    major grid style={line width=.2pt,draw=gray!50},
    xmin=700, xmax=850,
    ymin=0, ymax=115,
    legend pos=south east,
    restrict x to domain=700:850,
]
\addplot[blue, thick] table[x=time_s, y=PV, col sep=comma] {../data/TN4_processed.csv};
\addlegendentry{PV (Mức đo được)}
\addplot[red, dashed, thick] table[x=time_s, y=SV, col sep=comma] {../data/TN4_processed.csv};
\addlegendentry{SV (Giá trị đặt)}

% Đánh dấu thông số PID
\node[font=\small, fill=white] at (axis cs:775,50) {$K_P=4.44$, $T_i=7.5$s, $T_d=1.875$s};

% Đánh dấu overshoot
\draw[<->, gray] (axis cs:742,100) -- (axis cs:742,105.5);
\node[font=\scriptsize] at (axis cs:750,108) {Overshoot 5.52cm};

\end{axis}
\end{tikzpicture}
\caption{Giai đoạn kiểm tra: Đáp ứng bước của bộ điều khiển PID (SV: $0 \rightarrow 100$cm)}
\label{fig:level_test_tn4}
\end{figure}

\subsubsection{Phân tích đáp ứng}

\begin{itemize}
    \item \textbf{Giai đoạn chỉnh định (0-460s) - Khảo sát P-only:}
    \begin{itemize}
        \item Khảo sát $K_P$ từ 4 đến 7.4
        \item Xác định $K_u = 7.4$ (hệ thống bắt đầu dao động tới hạn)
        \item Sai lệch tĩnh với P-only: 4.39cm (do không có thành phần I)
    \end{itemize}
    
    \item \textbf{Giai đoạn kiểm tra (736-832s) - PID hoàn chỉnh:}
    \begin{itemize}
        \item Áp dụng $K_P = 4.44$, $T_i = 7.5$s, $T_d = 1.875$s
        \item Step từ mức 9.1cm lên 100cm
        \item Độ quá điều chỉnh: 5.52cm
        \item Sai lệch tĩnh: 0.04cm (gần như triệt tiêu)
    \end{itemize}
\end{itemize}

\subsubsection{Chỉ tiêu chất lượng điều khiển}

\begin{table}[H]
\centering
\begin{tabular}{|l|c|c|}
\hline
\textbf{Chỉ tiêu} & \textbf{P-only ($K_P=7.4$)} & \textbf{PID (Step response)} \\
\hline
Độ quá điều chỉnh & -- & 5.52cm (5.52\%) \\
\hline
Sai lệch tĩnh & 4.39cm & 0.04cm \\
\hline
Thời gian đạt setpoint & -- & $\approx$ 20s \\
\hline
Thời gian xác lập ($\pm 2$cm) & -- & $\approx$ 50s \\
\hline
\end{tabular}
\caption{So sánh chất lượng điều khiển P-only và PID}
\end{table}

% ------------------------------------------------------------
\subsection{Bàn luận}
% ------------------------------------------------------------

\subsubsection{So sánh điều khiển P-only và PID}

\begin{table}[H]
\centering
\begin{tabular}{|l|c|c|}
\hline
\textbf{Tiêu chí} & \textbf{P-only ($K_P=7.4$)} & \textbf{PID} \\
\hline
Sai lệch tĩnh & 4.39cm & 0.04cm \\
\hline
Độ quá điều chỉnh & 0 (không đạt setpoint) & 5.52cm \\
\hline
Đáp ứng động & Chậm, không đạt mục tiêu & Nhanh, đạt mục tiêu \\
\hline
Triệt tiêu nhiễu & Kém & Tốt \\
\hline
\end{tabular}
\caption{So sánh điều khiển P-only và PID cho hệ thống mức nước}
\end{table}

\subsubsection{Nhận xét về phương pháp Ziegler-Nichols}

\begin{itemize}
    \item \textbf{Ưu điểm:}
    \begin{itemize}
        \item Phương pháp đơn giản, dễ thực hiện trong môi trường mô phỏng
        \item Cho kết quả ban đầu tốt, có thể tinh chỉnh thêm
        \item Phù hợp cho các hệ thống có đáp ứng nhanh như điều khiển mức
    \end{itemize}
    
    \item \textbf{Điều chỉnh so với công thức gốc:}
    \begin{itemize}
        \item $T_i$ tăng từ 5s (theo công thức) lên 7.5s để giảm overshoot
        \item $T_d$ tăng từ 1.25s lên 1.875s để cải thiện độ ổn định
        \item Các điều chỉnh này phù hợp với đặc tính hệ thống mức nước trong Factory I/O
    \end{itemize}
\end{itemize}

\subsubsection{Đánh giá Factory I/O trong đào tạo}

\begin{itemize}
    \item \textbf{Ưu điểm:}
    \begin{itemize}
        \item Môi trường an toàn để thử nghiệm các thuật toán điều khiển
        \item Dễ dàng thay đổi thông số và quan sát kết quả ngay lập tức
        \item Tiết kiệm chi phí so với thiết bị thực
        \item Có thể mô phỏng các tình huống khó xảy ra trong thực tế
    \end{itemize}
    
    \item \textbf{Hạn chế:}
    \begin{itemize}
        \item Không phản ánh hoàn toàn các yếu tố nhiễu trong thực tế
        \item Độ chính xác phụ thuộc vào mô hình vật lý của phần mềm
    \end{itemize}
\end{itemize}

% ------------------------------------------------------------
\subsection{Kết luận}
% ------------------------------------------------------------

Qua thí nghiệm điều khiển mức nước với Factory I/O và PLCSIM, rút ra các kết luận sau:

\begin{enumerate}
    \item \textbf{Kết nối thành công:} Thiết lập kết nối giữa PLCSIM và Factory I/O qua giao thức S7, cho phép điều khiển thời gian thực hệ thống mô phỏng.
    
    \item \textbf{Chỉnh định PID theo Ziegler-Nichols:}
    \begin{itemize}
        \item Xác định $K_u = 7.4$ và $T_u \approx 10$s từ thí nghiệm P-only
        \item Tính toán: $K_P = 4.44$, $T_i = 7.5$s, $T_d = 1.875$s
        \item Bộ thông số này cho đáp ứng tốt với độ quá điều chỉnh 5.52cm
    \end{itemize}
    
    \item \textbf{Kết quả điều khiển:}
    \begin{itemize}
        \item Điều khiển P-only: sai lệch tĩnh 4.39cm, không đạt setpoint 100cm
        \item Điều khiển PID: sai lệch tĩnh 0.04cm, overshoot 5.52cm
        \item Thời gian xác lập khoảng 50s
    \end{itemize}
    
    \item \textbf{Ưu điểm Factory I/O:} Môi trường mô phỏng an toàn, tiện lợi cho việc học tập và thử nghiệm các thuật toán điều khiển.
\end{enumerate}

% ------------------------------------------------------------
\subsection{Câu hỏi kiểm tra}
% ------------------------------------------------------------

\textbf{Câu 1: Phương pháp Ziegler-Nichols II là gì? Tại sao trong bài này lại sử dụng phương pháp này để chỉnh định bộ điều khiển? Phương pháp này thích hợp cho những hệ có tính chất như thế nào?}

\textbf{Phương pháp Ziegler-Nichols II (Ultimate Gain Method):}

Phương pháp Ziegler-Nichols II (còn gọi là phương pháp dao động tới hạn) được đề xuất bởi Ziegler và Nichols năm 1942. Các bước thực hiện:
\begin{enumerate}
    \item Tắt thành phần I và D, chỉ dùng điều khiển P
    \item Tăng dần $K_P$ cho đến khi hệ thống bắt đầu dao động với biên độ không đổi
    \item Ghi nhận $K_u$ (ultimate gain) và $T_u$ (ultimate period - chu kỳ dao động)
    \item Tính thông số PID: $K_P = 0.6K_u$, $T_i = T_u/2$, $T_d = T_u/8$
\end{enumerate}

\textbf{Tại sao sử dụng phương pháp này trong bài thí nghiệm:}
\begin{itemize}
    \item Hệ thống mức nước trong Factory I/O có đáp ứng nhanh, dễ quan sát dao động
    \item Môi trường mô phỏng an toàn, có thể đưa hệ thống đến biên giới mất ổn định mà không gây hư hỏng
    \item Phương pháp đơn giản, trực quan, dễ thực hiện trong phòng thí nghiệm
    \item Cho kết quả ban đầu tốt, có thể tinh chỉnh thêm nếu cần
\end{itemize}

\textbf{Phương pháp này thích hợp cho các hệ có tính chất:}
\begin{itemize}
    \item Hệ thống có khả năng dao động khi tăng độ khuếch đại (có ít nhất 2 cực)
    \item Hệ thống ổn định ở độ khuếch đại thấp
    \item Có thể chịu được việc đưa đến biên giới mất ổn định (an toàn hoặc mô phỏng)
    \item Hệ thống tuyến tính hoặc gần tuyến tính quanh điểm làm việc
    \item Hệ thống có thời gian đáp ứng không quá chậm (để quan sát dao động)
\end{itemize}

\textbf{Lưu ý:} Phương pháp này \textbf{không phù hợp} cho các hệ thống:
\begin{itemize}
    \item Không thể đưa đến biên giới mất ổn định (nguy hiểm, đắt tiền)
    \item Hệ thống bậc 1 (không có dao động tới hạn)
    \item Hệ thống có trễ lớn (pure delay dominant)
\end{itemize}

\textbf{Câu 2: Cảm biến đo mức trong bài hoạt động theo nguyên lý nào? Trong khi vận hành, loại cảm biến này sẽ bị nhiễu bởi những yếu tố nào?}

\textbf{Nguyên lý hoạt động của cảm biến đo mức:}

Trong Factory I/O, cảm biến đo mức hoạt động theo nguyên lý \textbf{đo áp suất thủy tĩnh} (hydrostatic pressure):
\begin{itemize}
    \item Áp suất tại đáy bể tỉ lệ với chiều cao cột chất lỏng: $P = \rho g h$
    \item Trong đó: $\rho$ là khối lượng riêng chất lỏng, $g$ là gia tốc trọng trường, $h$ là chiều cao mức
    \item Cảm biến áp suất chuyển đổi áp suất thành tín hiệu điện (4-20mA hoặc 0-10V)
    \item PLC đọc tín hiệu analog và quy đổi ra giá trị mức (cm hoặc \%)
\end{itemize}

\textbf{Các yếu tố gây nhiễu cho cảm biến đo mức:}

\begin{enumerate}
    \item \textbf{Nhiễu do sóng trên bề mặt chất lỏng:}
    \begin{itemize}
        \item Dòng chảy vào/ra tạo sóng, gây dao động tín hiệu đo
        \item Khắc phục: Dùng ống giảm chấn (stilling well), lọc tín hiệu
    \end{itemize}
    
    \item \textbf{Thay đổi khối lượng riêng chất lỏng:}
    \begin{itemize}
        \item Nhiệt độ thay đổi làm $\rho$ thay đổi, ảnh hưởng đến phép đo
        \item Chất lỏng có bọt khí hoặc tạp chất
        \item Khắc phục: Bù nhiệt độ, hiệu chuẩn định kỳ
    \end{itemize}
    
    \item \textbf{Nhiễu điện từ:}
    \begin{itemize}
        \item Nhiễu từ động cơ, biến tần, thiết bị đóng cắt
        \item Khắc phục: Dùng cáp chống nhiễu, nối đất đúng cách
    \end{itemize}
    
    \item \textbf{Tắc nghẽn đường ống dẫn áp:}
    \begin{itemize}
        \item Cặn bẩn tích tụ trong đường ống nối cảm biến
        \item Khắc phục: Bảo trì định kỳ, dùng màng cách ly
    \end{itemize}
    
    \item \textbf{Thay đổi áp suất khí quyển:}
    \begin{itemize}
        \item Ảnh hưởng đến cảm biến đo áp suất tuyệt đối
        \item Khắc phục: Dùng cảm biến đo áp suất gauge (có thông khí)
    \end{itemize}
\end{enumerate}

% ------------------------------------------------------------
\subsection{Tài liệu tham khảo}
% ------------------------------------------------------------

\begin{enumerate}[label={[\arabic*]}]
    \item B. N. Pha, \textit{Thiết bị Đo lường và Điều khiển}. Trường Đại học Bách khoa, ĐHQG-HCM.
    \item Real Games, \textit{Factory I/O Manual}. https://factoryio.com/docs/
    \item Siemens, \textit{S7-1200 Programmable Controller System Manual}. Siemens AG, 2020.
\end{enumerate}

\newpage
