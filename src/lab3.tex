% ============================================================
% Lab 3: Điều khiển PID
% ============================================================

\setcounter{section}{3}
\setcounter{subsection}{0}
\section*{BÀI THÍ NGHIỆM 3: ĐIỀU KHIỂN PID}
\addcontentsline{toc}{section}{Bài thí nghiệm 3: Điều khiển PID}

% ------------------------------------------------------------
\subsection{Mục tiêu}
% ------------------------------------------------------------

\begin{itemize}
    \item Hiểu nguyên lý hoạt động của bộ điều khiển PID.
    \item Cấu hình và sử dụng khối PID trong TIA Portal.
    \item Chỉnh định thông số PID và đánh giá chất lượng điều khiển.
\end{itemize}

% ------------------------------------------------------------
\subsection{Cơ sở lý thuyết}
% ------------------------------------------------------------

\subsubsection{Bộ điều khiển PID}

Bộ điều khiển PID là bộ điều khiển đưa ra tác động điều khiển dựa trên ba thành phần: tỉ lệ (P), tích phân (I) và vi phân (D).

\begin{figure}[H]
\centering
\begin{tikzpicture}[node distance=1.5cm]
    % Input
    \node (input) {$e(t)$};
    
    % P, I, D blocks
    \node[block, right=1.5cm of input, yshift=1.2cm] (P) {$K_P$};
    \node[block, right=1.5cm of input] (I) {$K_I \int$};
    \node[block, right=1.5cm of input, yshift=-1.2cm] (D) {$K_D \frac{d}{dt}$};
    
    % Sum
    \node[sum, right=1.5cm of I] (sum) {};
    
    % Output
    \node[right=1cm of sum] (output) {$u(t)$};
    
    % Arrows from input
    \coordinate (split) at ($(input.east)+(0.5,0)$);
    \draw[line] (input) -- (split);
    \draw[arrow] (split) |- (P);
    \draw[arrow] (split) -- (I);
    \draw[arrow] (split) |- (D);
    
    % Arrows to sum
    \draw[arrow] (P) -| (sum);
    \draw[arrow] (I) -- (sum);
    \draw[arrow] (D) -| (sum);
    
    % Output arrow
    \draw[arrow] (sum) -- (output);
    
    % Sum signs
    \node[font=\scriptsize] at ($(sum.north)+(0,0.15)$) [right] {$+$};
    \node[font=\scriptsize] at ($(sum.west)+(-0.15,0)$) [below] {$+$};
    \node[font=\scriptsize] at ($(sum.south)+(0,-0.15)$) [right] {$+$};
\end{tikzpicture}
\caption{Sơ đồ khối bộ điều khiển PID}
\label{fig:pid_block_lab3}
\end{figure}

Công thức tổng quát:
\begin{equation}
u(t) = K_P \cdot e(t) + K_I \int_0^t e(\tau)d\tau + K_D \frac{de(t)}{dt}
\end{equation}

\subsubsection{Ảnh hưởng của các thông số}

\begin{table}[H]
\centering
\begin{tabular}{|c|c|c|c|c|}
\hline
\textbf{Tăng} & \textbf{Thời gian đáp ứng} & \textbf{Độ quá điều chỉnh} & \textbf{Sai lệch tĩnh} & \textbf{Ổn định} \\
\hline
$K_P$ & Giảm & Tăng & Giảm & Xấu đi \\
\hline
$K_I$ & Giảm & Tăng & Triệt tiêu & Xấu đi \\
\hline
$K_D$ & Ít thay đổi & Giảm & Không đổi & Tốt hơn \\
\hline
\end{tabular}
\caption{Ảnh hưởng của các thông số PID}
\end{table}

\subsubsection{Khối PID\_Compact trong TIA Portal}

TIA Portal cung cấp khối \texttt{PID\_Compact} để thực hiện điều khiển PID. Các tham số chính:
\begin{itemize}
    \item \texttt{Setpoint}: Giá trị đặt
    \item \texttt{Input}: Giá trị phản hồi (PV)
    \item \texttt{Output}: Tín hiệu điều khiển
    \item \texttt{Gain} ($K_P$): Hệ số khuếch đại
    \item \texttt{Ti}: Thời gian tích phân
    \item \texttt{Td}: Thời gian vi phân
\end{itemize}

% ------------------------------------------------------------
\subsection{Thiết bị thí nghiệm}
% ------------------------------------------------------------

\begin{itemize}
    \item PLC Siemens S7-1200 CPU 1214C
    \item Module analog input/output
    \item Cảm biến nhiệt độ Pt100
    \item Bộ gia nhiệt (Heater)
    \item Phần mềm TIA Portal V16
\end{itemize}

% ------------------------------------------------------------
\subsection{Quy trình thực hiện}
% ------------------------------------------------------------

\subsubsection{Cấu hình khối PID\_Compact}
\begin{enumerate}
    \item Thêm khối \texttt{PID\_Compact} vào chương trình.
    \item Cấu hình các tham số đầu vào/ra.
    \item Thiết lập giới hạn Output (0-100\%).
    \item Kết nối Setpoint và Input với biến tương ứng.
\end{enumerate}

\subsubsection{Chỉnh định thông số PID}

\textbf{Phương pháp 1: Auto-tuning}
\begin{enumerate}
    \item Sử dụng chức năng Auto-tune của PID\_Compact.
    \item Hệ thống tự động xác định $K_P$, $T_i$, $T_d$ tối ưu.
\end{enumerate}

\textbf{Phương pháp 2: Chỉnh định thủ công}
\begin{enumerate}
    \item Bắt đầu với $K_I = 0$, $K_D = 0$.
    \item Tăng $K_P$ cho đến khi hệ thống bắt đầu dao động.
    \item Thêm $K_I$ để triệt tiêu sai lệch tĩnh.
    \item Thêm $K_D$ để giảm độ quá điều chỉnh.
\end{enumerate}

\subsubsection{Khảo sát ảnh hưởng của thông số}
\begin{enumerate}
    \item Thay đổi $K_P$ và quan sát đáp ứng.
    \item Thay đổi $K_I$ và quan sát đáp ứng.
    \item Thay đổi $K_D$ và quan sát đáp ứng.
    \item Ghi nhận các chỉ tiêu chất lượng.
\end{enumerate}

% ------------------------------------------------------------
\subsection{Kết quả thí nghiệm}
% ------------------------------------------------------------

\subsubsection{Kết quả chỉnh định PID}

\begin{table}[H]
\centering
\begin{tabular}{|c|c|c|c|}
\hline
\textbf{Bộ thông số} & \textbf{$K_P$} & \textbf{$T_i$ (s)} & \textbf{$T_d$ (s)} \\
\hline
Auto-tune & & & \\
\hline
Thủ công & & & \\
\hline
\end{tabular}
\caption{Kết quả chỉnh định thông số PID}
\end{table}

\subsubsection{Kết quả khảo sát ảnh hưởng $K_P$}

\begin{table}[H]
\centering
\begin{tabular}{|c|c|c|c|}
\hline
\textbf{$K_P$} & \textbf{Thời gian xác lập (s)} & \textbf{Độ quá điều chỉnh (\%)} & \textbf{Sai lệch tĩnh} \\
\hline
& & & \\
\hline
& & & \\
\hline
& & & \\
\hline
\end{tabular}
\caption{Ảnh hưởng của $K_P$}
\end{table}

\subsubsection{Kết quả khảo sát ảnh hưởng $K_I$}

\begin{table}[H]
\centering
\begin{tabular}{|c|c|c|c|}
\hline
\textbf{$T_i$ (s)} & \textbf{Thời gian xác lập (s)} & \textbf{Độ quá điều chỉnh (\%)} & \textbf{Sai lệch tĩnh} \\
\hline
& & & \\
\hline
& & & \\
\hline
& & & \\
\hline
\end{tabular}
\caption{Ảnh hưởng của $T_i$}
\end{table}

\subsubsection{Kết quả khảo sát ảnh hưởng $K_D$}

\begin{table}[H]
\centering
\begin{tabular}{|c|c|c|c|}
\hline
\textbf{$T_d$ (s)} & \textbf{Thời gian xác lập (s)} & \textbf{Độ quá điều chỉnh (\%)} & \textbf{Sai lệch tĩnh} \\
\hline
& & & \\
\hline
& & & \\
\hline
& & & \\
\hline
\end{tabular}
\caption{Ảnh hưởng của $T_d$}
\end{table}

% ------------------------------------------------------------
\subsection{Bàn luận}
% ------------------------------------------------------------

\begin{itemize}
    \item Nhận xét về kết quả Auto-tune so với chỉnh định thủ công.
    \item Phân tích ảnh hưởng của từng thông số đến chất lượng điều khiển.
    \item So sánh với kết quả điều khiển ON/OFF (Bài TN 2).
\end{itemize}

% ------------------------------------------------------------
\subsection{Kết luận}
% ------------------------------------------------------------

% TODO: Điền kết luận sau khi thực hiện thí nghiệm

% ------------------------------------------------------------
\subsection{Câu hỏi kiểm tra}
% ------------------------------------------------------------

\textbf{Câu 1: Trình bày các bước chỉnh định PID theo phương pháp Ziegler-Nichols?}

% TODO: Trả lời

\textbf{Câu 2: Giải thích ý nghĩa của thời gian tích phân $T_i$ và thời gian vi phân $T_d$?}

% TODO: Trả lời

\textbf{Câu 3: Khi nào nên sử dụng bộ điều khiển P, PI, hay PID?}

% TODO: Trả lời

% ------------------------------------------------------------
\subsection{Tài liệu tham khảo}
% ------------------------------------------------------------

\begin{enumerate}[label={[\arabic*]}]
    \item B. N. Pha, \textit{Thiết bị Đo lường và Điều khiển}. Trường Đại học Bách khoa, ĐHQG-HCM.
    \item Siemens, \textit{S7-1200 Programmable Controller System Manual}. Siemens AG, 2020.
    \item Siemens, \textit{PID Control with SIMATIC S7-1200}. Siemens AG, 2019.
\end{enumerate}

\newpage
