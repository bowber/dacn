% ============================================================
% Lab 3: Điều khiển PID
% ============================================================

\setcounter{section}{3}
\setcounter{subsection}{0}
\section*{BÀI THÍ NGHIỆM 3: ĐIỀU KHIỂN PID}
\addcontentsline{toc}{section}{Bài thí nghiệm 3: Điều khiển PID}

% ------------------------------------------------------------
\subsection{Mục tiêu}
% ------------------------------------------------------------

\begin{itemize}
    \item Hiểu nguyên lý hoạt động của bộ điều khiển PID.
    \item Cấu hình và sử dụng khối PID trong TIA Portal.
    \item Chỉnh định thông số PID và đánh giá chất lượng điều khiển.
\end{itemize}

% ------------------------------------------------------------
\subsection{Cơ sở lý thuyết}
% ------------------------------------------------------------

\subsubsection{Bộ điều khiển PID}

Bộ điều khiển PID là bộ điều khiển đưa ra tác động điều khiển dựa trên ba thành phần: tỉ lệ (P), tích phân (I) và vi phân (D).

\begin{figure}[H]
\centering
\begin{tikzpicture}[node distance=1.5cm]
    % Input
    \node (input) {$e(t)$};
    
    % P, I, D blocks
    \node[block, right=1.5cm of input, yshift=1.2cm] (P) {$K_P$};
    \node[block, right=1.5cm of input] (I) {$K_I \int$};
    \node[block, right=1.5cm of input, yshift=-1.2cm] (D) {$K_D \frac{d}{dt}$};
    
    % Sum
    \node[sum, right=1.5cm of I] (sum) {};
    
    % Output
    \node[right=1cm of sum] (output) {$u(t)$};
    
    % Arrows from input
    \coordinate (split) at ($(input.east)+(0.5,0)$);
    \draw[line] (input) -- (split);
    \draw[arrow] (split) |- (P);
    \draw[arrow] (split) -- (I);
    \draw[arrow] (split) |- (D);
    
    % Arrows to sum
    \draw[arrow] (P) -| (sum);
    \draw[arrow] (I) -- (sum);
    \draw[arrow] (D) -| (sum);
    
    % Output arrow
    \draw[arrow] (sum) -- (output);
    
    % Sum signs
    \node[font=\scriptsize] at ($(sum.north)+(0,0.15)$) [right] {$+$};
    \node[font=\scriptsize] at ($(sum.west)+(-0.15,0)$) [below] {$+$};
    \node[font=\scriptsize] at ($(sum.south)+(0,-0.15)$) [right] {$+$};
\end{tikzpicture}
\caption{Sơ đồ khối bộ điều khiển PID}
\label{fig:pid_block_lab3}
\end{figure}

Công thức tổng quát:
\begin{equation}
u(t) = K_P \cdot e(t) + K_I \int_0^t e(\tau)d\tau + K_D \frac{de(t)}{dt}
\end{equation}

\subsubsection{Ảnh hưởng của các thông số}

\begin{table}[H]
\centering
\begin{tabular}{|c|c|c|c|c|}
\hline
\textbf{Tăng} & \textbf{Thời gian đáp ứng} & \textbf{Độ quá điều chỉnh} & \textbf{Sai lệch tĩnh} & \textbf{Ổn định} \\
\hline
$K_P$ & Giảm & Tăng & Giảm & Xấu đi \\
\hline
$K_I$ & Giảm & Tăng & Triệt tiêu & Xấu đi \\
\hline
$K_D$ & Ít thay đổi & Giảm & Không đổi & Tốt hơn \\
\hline
\end{tabular}
\caption{Ảnh hưởng của các thông số PID}
\end{table}

\subsubsection{Khối PID\_Compact trong TIA Portal}

TIA Portal cung cấp khối \texttt{PID\_Compact} để thực hiện điều khiển PID. Các tham số chính:
\begin{itemize}
    \item \texttt{Setpoint}: Giá trị đặt
    \item \texttt{Input}: Giá trị phản hồi (PV)
    \item \texttt{Output}: Tín hiệu điều khiển
    \item \texttt{Gain} ($K_P$): Hệ số khuếch đại
    \item \texttt{Ti}: Thời gian tích phân
    \item \texttt{Td}: Thời gian vi phân
\end{itemize}

% ------------------------------------------------------------
\subsection{Thiết bị thí nghiệm}
% ------------------------------------------------------------

\begin{itemize}
    \item PLC Siemens S7-1200 CPU 1214C
    \item Module analog input/output
    \item Cảm biến nhiệt độ Pt100
    \item Bộ gia nhiệt (Heater)
    \item Phần mềm TIA Portal V16
\end{itemize}

% ------------------------------------------------------------
\subsection{Quy trình thực hiện}
% ------------------------------------------------------------

\subsubsection{Cấu hình khối PID\_Compact}
\begin{enumerate}
    \item Thêm khối \texttt{PID\_Compact} vào chương trình.
    \item Cấu hình các tham số đầu vào/ra.
    \item Thiết lập giới hạn Output (0-100\%).
    \item Kết nối Setpoint và Input với biến tương ứng.
\end{enumerate}

\subsubsection{Chỉnh định thông số PID}

\textbf{Phương pháp 1: Auto-tuning}
\begin{enumerate}
    \item Sử dụng chức năng Auto-tune của PID\_Compact.
    \item Hệ thống tự động xác định $K_P$, $T_i$, $T_d$ tối ưu.
\end{enumerate}

\textbf{Phương pháp 2: Chỉnh định thủ công}
\begin{enumerate}
    \item Bắt đầu với $K_I = 0$, $K_D = 0$.
    \item Tăng $K_P$ cho đến khi hệ thống bắt đầu dao động.
    \item Thêm $K_I$ để triệt tiêu sai lệch tĩnh.
    \item Thêm $K_D$ để giảm độ quá điều chỉnh.
\end{enumerate}

\subsubsection{Khảo sát ảnh hưởng của thông số}
\begin{enumerate}
    \item Thay đổi $K_P$ và quan sát đáp ứng.
    \item Thay đổi $K_I$ và quan sát đáp ứng.
    \item Thay đổi $K_D$ và quan sát đáp ứng.
    \item Ghi nhận các chỉ tiêu chất lượng.
\end{enumerate}

% ------------------------------------------------------------
\subsection{Kết quả thí nghiệm}
% ------------------------------------------------------------

\subsubsection{Đáp ứng điều khiển PID}

Thí nghiệm điều khiển PID với SV = $40^\circ$C, khảo sát ảnh hưởng của các thông số $K_P$, $T_i$, $T_d$.

\begin{figure}[H]
\centering
\begin{tikzpicture}
\begin{axis}[
    width=0.9\textwidth,
    height=8cm,
    xlabel={Thời gian (s)},
    ylabel={Nhiệt độ ($^\circ$C)},
    grid=both,
    grid style={line width=.1pt, draw=gray!30},
    major grid style={line width=.2pt,draw=gray!50},
    xmin=0, xmax=1600,
    ymin=24, ymax=50,
    legend pos=north east,
]
\addplot[blue, thick] table[x=time_s, y=PV, col sep=comma] {../tmp/TN3_processed.csv};
\addlegendentry{PV (Nhiệt độ đo được)}
\addplot[red, dashed, thick] table[x=time_s, y=SV, col sep=comma] {../tmp/TN3_processed.csv};
\addlegendentry{SV = $40^\circ$C}

% Đánh dấu các điểm thay đổi thông số PID
\draw[gray, dashed] (axis cs:591,24) -- (axis cs:591,50);
\draw[gray, dashed] (axis cs:767,24) -- (axis cs:767,50);
\draw[gray, dashed] (axis cs:1002,24) -- (axis cs:1002,50);

\node[font=\tiny, rotate=90] at (axis cs:300,47) {kp=0.2, ti=1, td=5};
\node[font=\tiny, rotate=90] at (axis cs:680,47) {kp=1, ti=1, td=1};
\node[font=\tiny, rotate=90] at (axis cs:880,47) {kp=0.2, td=0.1};
\node[font=\tiny, rotate=90] at (axis cs:1300,47) {kp=0.4, ti=0.1};
\end{axis}
\end{tikzpicture}
\caption{Đáp ứng điều khiển PID với các bộ thông số khác nhau}
\label{fig:pid_response_tn3}
\end{figure}

Phân tích đáp ứng:
\begin{itemize}
    \item \textbf{Giai đoạn 1 (0-591s):} $K_P=0.2$, $T_i=1$, $T_d=5$ - Hệ thống đáp ứng chậm, vượt quá setpoint rồi dao động
    \item \textbf{Giai đoạn 2 (596-762s):} $K_P=1$, $T_i=1$, $T_d=1$ - Tăng $K_P$ làm hệ thống đáp ứng mạnh, dao động lớn
    \item \textbf{Giai đoạn 3 (767-897s):} $K_P=0.2$, $T_i=1$, $T_d=1$ - Giảm $K_P$, hệ thống ổn định hơn
    \item \textbf{Giai đoạn 4 (sau 1000s):} $K_P=0.4$, $T_i=0.1$ - Điều chỉnh tinh, sai lệch giảm đáng kể
\end{itemize}

\subsubsection{Kết quả chỉnh định PID}

\begin{table}[H]
\centering
\begin{tabular}{|c|c|c|c|}
\hline
\textbf{Bộ thông số} & \textbf{$K_P$} & \textbf{$T_i$ (s)} & \textbf{$T_d$ (s)} \\
\hline
Bộ 1 (ban đầu) & 0.2 & 1 & 5 \\
\hline
Bộ 2 (tăng $K_P$) & 1 & 1 & 1 \\
\hline
Bộ 3 (tối ưu) & 0.4 & 0.1 & 0.2 \\
\hline
\end{tabular}
\caption{Kết quả chỉnh định thông số PID}
\end{table}

\subsubsection{Kết quả khảo sát ảnh hưởng $K_P$}

\begin{table}[H]
\centering
\begin{tabular}{|c|c|c|c|}
\hline
\textbf{$K_P$} & \textbf{Thời gian xác lập (s)} & \textbf{Độ quá điều chỉnh (\%)} & \textbf{Sai lệch tĩnh} \\
\hline
0.2 & $>$ 300 & 0 & $\approx 10^\circ$C \\
\hline
0.4 & $\approx$ 200 & 5-10 & $\approx 2^\circ$C \\
\hline
1.0 & $\approx$ 100 & 10-15 & $< 1^\circ$C \\
\hline
\end{tabular}
\caption{Ảnh hưởng của $K_P$}
\end{table}

\subsubsection{Kết quả khảo sát ảnh hưởng $K_I$}

\begin{table}[H]
\centering
\begin{tabular}{|c|c|c|c|}
\hline
\textbf{$T_i$ (s)} & \textbf{Thời gian xác lập (s)} & \textbf{Độ quá điều chỉnh (\%)} & \textbf{Sai lệch tĩnh} \\
\hline
1 & $\approx$ 200 & 5 & $\approx 3^\circ$C \\
\hline
0.5 & $\approx$ 150 & 8 & $\approx 1^\circ$C \\
\hline
0.1 & $\approx$ 120 & 10 & $< 0.5^\circ$C \\
\hline
\end{tabular}
\caption{Ảnh hưởng của $T_i$}
\end{table}

\subsubsection{Kết quả khảo sát ảnh hưởng $K_D$}

\begin{table}[H]
\centering
\begin{tabular}{|c|c|c|c|}
\hline
\textbf{$T_d$ (s)} & \textbf{Thời gian xác lập (s)} & \textbf{Độ quá điều chỉnh (\%)} & \textbf{Sai lệch tĩnh} \\
\hline
0.05 & $\approx$ 150 & 12 & $\approx 2^\circ$C \\
\hline
0.2 & $\approx$ 130 & 6 & $\approx 1^\circ$C \\
\hline
1 & $\approx$ 120 & 3 & $\approx 1^\circ$C \\
\hline
\end{tabular}
\caption{Ảnh hưởng của $T_d$}
\end{table}

% ------------------------------------------------------------
\subsection{Bàn luận}
% ------------------------------------------------------------

\subsubsection{Nhận dạng hệ thống (System Identification)}

Từ đáp ứng bước của hệ thống trong giai đoạn đầu (TN3), xác định được các thông số của mô hình FOPDT (First Order Plus Dead Time):

\begin{itemize}
    \item Thời gian trễ (Dead Time): $L = 90$s
    \item Hằng số thời gian (Time Constant): $T = 75$s
    \item Tỷ số $T/L = 0.83$ (hệ thống có trễ đáng kể)
\end{itemize}

\subsubsection{Tính toán thông số PID theo các phương pháp lý thuyết}

Dựa trên các thông số nhận dạng được, áp dụng các phương pháp chỉnh định PID:

\begin{table}[H]
\centering
\begin{tabular}{|l|c|c|c|}
\hline
\textbf{Phương pháp} & \textbf{$K_P$} & \textbf{$T_i$ (s)} & \textbf{$T_d$ (s)} \\
\hline
Ziegler-Nichols (vòng hở) & 1.00 & 180 & 45 \\
\hline
Cohen-Coon & 1.36 & 156 & 27 \\
\hline
CHR (0\% overshoot) & 0.50 & 75 & 45 \\
\hline
CHR (20\% overshoot) & 0.79 & 105 & 42 \\
\hline
Lambda Tuning & 0.21 & 75 & 0 \\
\hline
IMC & 0.46 & 75 & 28 \\
\hline
\end{tabular}
\caption{Thông số PID theo các phương pháp lý thuyết}
\label{tab:pid_methods}
\end{table}

\textbf{Công thức tính:}
\begin{itemize}
    \item \textbf{Ziegler-Nichols:} $K_P = 1.2 \cdot \frac{T}{L}$, $T_i = 2L$, $T_d = 0.5L$
    \item \textbf{CHR (0\% overshoot):} $K_P = 0.6 \cdot \frac{T}{L}$, $T_i = T$, $T_d = 0.5L$
    \item \textbf{Lambda:} $K_P = \frac{T}{\lambda + L}$ với $\lambda = \max(T, 3L)$
\end{itemize}

\subsubsection{So sánh kết quả thực nghiệm với lý thuyết}

Phân tích các bộ thông số đã thử nghiệm theo sai số trung bình tuyệt đối (MAE):

\begin{table}[H]
\centering
\begin{tabular}{|c|c|c|c|c|c|}
\hline
\textbf{$K_P$} & \textbf{$T_i$} & \textbf{$T_d$} & \textbf{MAE ($^\circ$C)} & \textbf{Max Error} & \textbf{Dao động} \\
\hline
0.40 & 0.1 & 0.15 & 1.49 & 2.60 & 2.35 \\
\hline
0.40 & 0.1 & 0.20 & 1.86 & 2.97 & 5.66 \\
\hline
0.40 & 0.1 & 10.0 & 2.08 & 3.70 & 6.85 \\
\hline
0.20 & 1.0 & 0.1 & 2.84 & 4.82 & 5.44 \\
\hline
1.00 & 1.0 & 1.0 & 6.82 & 11.61 & 15.18 \\
\hline
\end{tabular}
\caption{Hiệu suất các bộ thông số PID (xếp theo MAE)}
\label{tab:pid_performance}
\end{table}

\textbf{Nhận xét quan trọng:}
\begin{itemize}
    \item Giá trị $K_P$ thực nghiệm (0.2--1.0) phù hợp với dự đoán lý thuyết
    \item Giá trị $T_i$ thực nghiệm (0.1--1.0s) nhỏ hơn nhiều so với lý thuyết (75--180s), có thể do TIA Portal sử dụng đơn vị hoặc công thức khác
    \item Bộ thông số tốt nhất: $K_P = 0.4$, $T_i = 0.1$, $T_d = 0.15$ với MAE = $1.49^\circ$C
\end{itemize}

\subsubsection{Ảnh hưởng của từng thông số}

\begin{itemize}
    \item \textbf{Tăng $K_P$:} Giảm thời gian đáp ứng, giảm sai lệch tĩnh, nhưng tăng độ quá điều chỉnh và có thể gây mất ổn định (thấy rõ khi $K_P=1.0$)
    \item \textbf{Giảm $T_i$:} Tăng tác động tích phân, triệt tiêu sai lệch tĩnh nhanh hơn. $T_i=0.1$ cho kết quả tốt nhất
    \item \textbf{Thay đổi $T_d$:} $T_d$ vừa phải (0.15--0.2) giúp giảm dao động và cải thiện độ ổn định
\end{itemize}

% ------------------------------------------------------------
\subsection{Kết luận}
% ------------------------------------------------------------

Qua thí nghiệm điều khiển PID hệ thống gia nhiệt, rút ra các kết luận sau:

\begin{enumerate}
    \item \textbf{Nhận dạng hệ thống:} Hệ thống gia nhiệt có thời gian trễ $L = 90$s và hằng số thời gian $T = 75$s, thuộc loại hệ thống có trễ đáng kể ($T/L < 1$).
    
    \item \textbf{Phương pháp chỉnh định:} Các phương pháp lý thuyết (Ziegler-Nichols, CHR, IMC, Lambda) cho giá trị $K_P$ trong khoảng 0.2--1.4, phù hợp với kết quả thực nghiệm.
    
    \item \textbf{Bộ thông số tối ưu từ thực nghiệm:} 
    \begin{center}
    $K_P = 0.4$, \quad $T_i = 0.1$s, \quad $T_d = 0.15$s
    \end{center}
    Cho sai số trung bình $1.49^\circ$C và dao động $2.35^\circ$C.
    
    \item \textbf{Khuyến nghị:}
    \begin{itemize}
        \item Với hệ thống có trễ lớn như thế này, nên bắt đầu với thông số bảo thủ (Lambda hoặc IMC)
        \item $K_P$ nên trong khoảng 0.3--0.5 để cân bằng giữa tốc độ đáp ứng và độ ổn định
        \item $T_i$ nhỏ (0.1) giúp triệt tiêu sai lệch tĩnh hiệu quả
        \item $T_d$ vừa phải (0.1--0.2) giúp giảm quá điều chỉnh
    \end{itemize}
\end{enumerate}

% ------------------------------------------------------------
\subsection{Câu hỏi kiểm tra}
% ------------------------------------------------------------

\textbf{Câu 1: Trình bày các bước chỉnh định PID theo phương pháp Ziegler-Nichols?}

Phương pháp Ziegler-Nichols (phương pháp dao động tới hạn):
\begin{enumerate}
    \item Đặt $T_i = \infty$ (tắt thành phần I), $T_d = 0$ (tắt thành phần D)
    \item Tăng dần $K_P$ từ 0 cho đến khi hệ thống bắt đầu dao động điều hòa với biên độ không đổi
    \item Ghi nhận giá trị $K_P$ tới hạn ($K_u$) và chu kỳ dao động ($T_u$)
    \item Tính toán thông số PID theo bảng Ziegler-Nichols:
    \begin{itemize}
        \item Bộ P: $K_P = 0.5 K_u$
        \item Bộ PI: $K_P = 0.45 K_u$, $T_i = T_u/1.2$
        \item Bộ PID: $K_P = 0.6 K_u$, $T_i = T_u/2$, $T_d = T_u/8$
    \end{itemize}
\end{enumerate}

\textbf{Câu 2: Giải thích ý nghĩa của thời gian tích phân $T_i$ và thời gian vi phân $T_d$?}

\begin{itemize}
    \item \textbf{Thời gian tích phân $T_i$:} Là thời gian cần thiết để thành phần tích phân tạo ra tác động điều khiển bằng với thành phần tỉ lệ khi sai lệch không đổi. $T_i$ nhỏ làm tác động tích phân mạnh hơn, giúp triệt tiêu sai lệch tĩnh nhanh hơn nhưng có thể gây mất ổn định.
    
    \item \textbf{Thời gian vi phân $T_d$:} Là thời gian dự đoán (prediction time) của thành phần vi phân, thể hiện khả năng phản ứng với tốc độ thay đổi của sai lệch. $T_d$ lớn làm tác động vi phân mạnh hơn, giúp giảm độ quá điều chỉnh và cải thiện độ ổn định.
\end{itemize}

\textbf{Câu 3: Khi nào nên sử dụng bộ điều khiển P, PI, hay PID?}

\begin{itemize}
    \item \textbf{Bộ P:} Sử dụng khi chấp nhận được sai lệch tĩnh và cần đáp ứng đơn giản, nhanh. Ví dụ: điều khiển mức nước trong bể chứa lớn.
    
    \item \textbf{Bộ PI:} Sử dụng khi cần triệt tiêu sai lệch tĩnh và hệ thống không quá nhạy với nhiễu. Ví dụ: điều khiển lưu lượng, điều khiển áp suất.
    
    \item \textbf{Bộ PID:} Sử dụng cho các hệ thống cần độ chính xác cao, cần triệt tiêu sai lệch tĩnh và giảm độ quá điều chỉnh. Ví dụ: điều khiển nhiệt độ, điều khiển vị trí servo, các quy trình công nghiệp yêu cầu cao.
\end{itemize}

% ------------------------------------------------------------
\subsection{Tài liệu tham khảo}
% ------------------------------------------------------------

\begin{enumerate}[label={[\arabic*]}]
    \item B. N. Pha, \textit{Thiết bị Đo lường và Điều khiển}. Trường Đại học Bách khoa, ĐHQG-HCM.
    \item Siemens, \textit{S7-1200 Programmable Controller System Manual}. Siemens AG, 2020.
    \item Siemens, \textit{PID Control with SIMATIC S7-1200}. Siemens AG, 2019.
\end{enumerate}

\newpage
