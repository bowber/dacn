% ============================================================
% Lab 3: Khảo sát các bộ điều khiển ON-OFF và PID
% ============================================================

\setcounter{section}{3}
\setcounter{subsection}{0}
\section*{BÀI THÍ NGHIỆM 3: KHẢO SÁT CÁC BỘ ĐIỀU KHIỂN ON-OFF VÀ PID}
\addcontentsline{toc}{section}{Bài thí nghiệm 3: Khảo sát các bộ điều khiển ON-OFF và PID}

% ------------------------------------------------------------
\subsection{Tóm tắt}
% ------------------------------------------------------------

Bài thí nghiệm này nhằm khảo sát và so sánh các bộ điều khiển ON-OFF và PID trên hệ thống điều khiển các đại lượng cơ bản trong phòng thí nghiệm. Thông qua việc thực hiện các thí nghiệm với các bộ điều khiển khác nhau, sinh viên có thể đánh giá chất lượng điều khiển dựa trên các chỉ tiêu như thời gian xác lập, độ quá điều chỉnh và sai lệch tĩnh. Kết quả cho thấy bộ điều khiển PID cho chất lượng điều khiển tốt nhất với thời gian xác lập ngắn và sai lệch tĩnh nhỏ.

% ------------------------------------------------------------
\subsection{Cơ sở lý thuyết}
% ------------------------------------------------------------

\subsubsection{Bộ điều khiển ON-OFF}

Bộ điều khiển ON-OFF là bộ điều khiển chỉ cho tín hiệu ra ở hai chế độ ($u_{max}$ và $u_{min}$) hoặc hai trạng thái (bật và tắt) tùy thuộc vào sai lệch điều khiển $e$ mà nó nhận được.

% Sơ đồ khối điều khiển ON-OFF
\begin{figure}[H]
\centering
\begin{tikzpicture}[node distance=1.8cm]
    % Nodes
    \node (sp) {$Y_{SP}$};
    \node[sum, right=1cm of sp] (sum) {};
    \node[block, right=1.5cm of sum, minimum width=5em] (controller) {ON-OFF};
    \node[block, right=1.5cm of controller] (process) {Quá trình};
    \node[right=1.5cm of process] (output) {$Y$};
    
    % Sensor
    \node[block, below=1.2cm of process] (sensor) {Cảm biến};
    
    % Arrows
    \draw[arrow] (sp) -- (sum);
    \draw[arrow] (sum) -- node[labelnode, above] {$e$} (controller);
    \draw[arrow] (controller) -- node[labelnode, above] {$u$} (process);
    \draw[arrow] (process) -- (output);
    
    % Feedback
    \coordinate (fb) at ($(process.east)!0.6!(output.west)$);
    \draw[line] (fb) |- (sensor);
    \draw[arrow] (sensor) -| (sum);
    
    % Sum signs - positioned outside circle, below arrow lines
    \node[font=\scriptsize] at ($(sp.east)!0.5!(sum.west)$) [below] {$+$};
    \node[font=\scriptsize] at ($(sum.south)+(0,-0.5)$) [left] {$-$};
\end{tikzpicture}
\caption{Sơ đồ khối bộ điều khiển ON-OFF}
\label{fig:onoff_block}
\end{figure}

\begin{figure}[H]
\centering
\begin{tikzpicture}
    % ON-OFF characteristic
    \draw[->] (-1.5,0) -- (2,0) node[right] {$e$};
    \draw[->] (0,-1.5) -- (0,2) node[above] {$u$};
    \draw[thick] (-1.2,-0.8) -- (0,-0.8) -- (0,1) -- (1.5,1);
    \node[font=\small] at (2,1) {$u_{max}$};
    \node[font=\small] at (2,-0.8) {$u_{min}$};
\end{tikzpicture}
\caption{Đặc tính bộ điều khiển ON-OFF}
\label{fig:onoff_characteristic}
\end{figure}

Phương trình mô tả:
\begin{equation}
u = \begin{cases}
u_{max} & \text{nếu } e > 0 \\
u_{min} & \text{nếu } e < 0
\end{cases}
\end{equation}

Loại điều khiển này không thực sự giữ biến được điều khiển chính xác tại điểm đặt, mà biến được điều khiển luôn dao động quanh điểm đặt với một biên độ nhất định.

\subsubsection{Bộ điều khiển liên tục PID}

Bộ điều khiển PID là bộ điều khiển đưa ra tác động điều khiển dựa trên các tác động thành phần bao gồm P, I và D:

% Sơ đồ khối bộ điều khiển PID
\begin{figure}[H]
\centering
\begin{tikzpicture}[node distance=1.5cm]
    % Input
    \node (input) {$e(t)$};
    
    % P, I, D blocks
    \node[block, right=1.5cm of input, yshift=1.2cm] (P) {$K_P$};
    \node[block, right=1.5cm of input] (I) {$K_I \int$};
    \node[block, right=1.5cm of input, yshift=-1.2cm] (D) {$K_D \frac{d}{dt}$};
    
    % Sum
    \node[sum, right=1.5cm of I] (sum) {};
    
    % Output
    \node[right=1cm of sum] (output) {$u(t)$};
    
    % Arrows from input
    \coordinate (split) at ($(input.east)+(0.5,0)$);
    \draw[line] (input) -- (split);
    \draw[arrow] (split) |- (P);
    \draw[arrow] (split) -- (I);
    \draw[arrow] (split) |- (D);
    
    % Arrows to sum
    \draw[arrow] (P) -| (sum);
    \draw[arrow] (I) -- (sum);
    \draw[arrow] (D) -| (sum);
    
    % Output arrow
    \draw[arrow] (sum) -- (output);
    
    % Sum signs - positioned below arrow lines
    \node[font=\scriptsize] at ($(sum.north)+(0,0.15)$) [right] {$+$};
    \node[font=\scriptsize] at ($(sum.west)+(-0.15,0)$) [below] {$+$};
    \node[font=\scriptsize] at ($(sum.south)+(0,-0.15)$) [right] {$+$};
\end{tikzpicture}
\caption{Sơ đồ khối bộ điều khiển PID}
\label{fig:pid_block}
\end{figure}

\begin{itemize}
    \item \textbf{Thành phần P (Proportional):} tác động tỉ lệ với độ lớn sai lệch điều khiển $e(t)$
    \item \textbf{Thành phần I (Integral):} tác động tỉ lệ với tích phân sai lệch điều khiển $\int e(t)dt$
    \item \textbf{Thành phần D (Derivative):} tác động tỉ lệ với vi phân sai lệch điều khiển $\frac{de(t)}{dt}$
\end{itemize}

Công thức tổng quát của bộ điều khiển PID:
\begin{equation}
u(t) = K_P \cdot e(t) + K_I \int_0^t e(\tau)d\tau + K_D \frac{de(t)}{dt}
\end{equation}

\subsubsection{Ảnh hưởng của các thành phần P, I, D}

\begin{table}[H]
\centering
\begin{tabular}{|c|c|c|c|c|}
\hline
\textbf{Thành phần} & \textbf{Thời gian đáp ứng} & \textbf{Độ quá điều chỉnh} & \textbf{Sai lệch tĩnh} & \textbf{Ổn định} \\
\hline
Tăng $K_P$ & Giảm & Tăng & Giảm & Xấu đi \\
\hline
Tăng $K_I$ & Giảm & Tăng & Triệt tiêu & Xấu đi \\
\hline
Tăng $K_D$ & Ít thay đổi & Giảm & Không ảnh hưởng & Tốt hơn \\
\hline
\end{tabular}
\caption{Ảnh hưởng của các thành phần P, I, D đến chất lượng điều khiển}
\end{table}

\subsubsection{Các chỉ tiêu chất lượng điều khiển}

\begin{itemize}
    \item \textbf{Thời gian xác lập ($t_s$):} Thời gian để đáp ứng đạt đến và duy trì trong phạm vi ±5\% của giá trị xác lập.
    \item \textbf{Độ quá điều chỉnh ($\sigma$\%):} Mức vượt quá giá trị xác lập so với giá trị xác lập, tính bằng phần trăm.
    \item \textbf{Sai lệch tĩnh ($e_{ss}$):} Sai lệch giữa giá trị đặt và giá trị thực khi hệ thống đạt trạng thái xác lập.
\end{itemize}

% ------------------------------------------------------------
\subsection{Mô tả thiết bị thí nghiệm}
% ------------------------------------------------------------

Cấu tạo mỗi hệ thống điều chỉnh bao gồm quá trình kỹ thuật và máy tính cài đặt phần mềm điều khiển và giao diện giám sát Halalab. Giao diện giám sát điều khiển bao gồm các phần sau:

\begin{itemize}
    \item Đồ thị hiển thị giá trị đại lượng cần điều khiển theo thời gian thực.
    \item Đồ thị hiển thị các thông số trạng thái hệ thống.
    \item Giá trị cài đặt của đại lượng cần điều khiển.
    \item Khảo sát các bộ điều khiển tự động.
    \item Lựa chọn bộ điều khiển ON-OFF hoặc PID để khảo sát.
\end{itemize}

% ------------------------------------------------------------
\subsection{Quy trình thực hiện}
% ------------------------------------------------------------

\subsubsection{Chuẩn bị thí nghiệm}
\begin{enumerate}
    \item Kiểm tra nguồn điện và các kết nối của hệ thống.
    \item Kiểm tra mức nước trong bể chứa nguồn.
    \item Khởi động phần mềm điều khiển trên máy tính.
    \item Đặt giá trị mức chất lỏng mong muốn $y_{sp}$ = 9 cm.
\end{enumerate}

\subsubsection{Thí nghiệm 1: Khảo sát bộ điều khiển ON/OFF}

\begin{enumerate}
    \item Nhập giá trị biến cần điều khiển mong muốn $y_{sp}$
    \item Chọn bộ điều khiển ON/OFF trên giao diện điều khiển
    \item Thực hiện khảo sát bằng cách nhấn vào nút Auto
    \item Quan sát đáp ứng của hệ thống trên đồ thị
    \item Ghi nhận chỉ tiêu chất lượng: thời gian xác lập, chu kì dao động, biên độ dao động
\end{enumerate}

\subsubsection{Thí nghiệm 2: Khảo sát các bộ điều khiển P, PI và PID}

\begin{enumerate}
    \item Chọn lần lượt bộ điều khiển P, PI và PID trên giao diện
    \item Sử dụng các thông số mặc định của bộ điều khiển
    \item Quan sát đáp ứng của hệ thống
    \item Ghi nhận chỉ tiêu chất lượng cho từng bộ điều khiển
    \item So sánh kết quả giữa các bộ điều khiển
\end{enumerate}

% ------------------------------------------------------------
\subsection{Kết quả và bàn luận}
% ------------------------------------------------------------

\subsubsection{Kết quả thí nghiệm 1: Bộ điều khiển ON-OFF}

\begin{figure}[H]
\centering
\begin{tikzpicture}
\begin{axis}[
    width=14cm,
    height=6cm,
    xlabel={Thời gian (giây)},
    ylabel={Giá trị},
    grid=both,
    grid style={gray!30},
    ymin=-50,
    ymax=550,
    xmin=0,
    xmax=466,
    legend pos=south east,
    legend style={font=\small},
]
% Setpoint line
\addplot[red, dashed, thick] coordinates {(0,450) (466,450)};
\addlegendentry{SP = 450}

% Process variable - Exp 1 (ON-OFF)
\addplot[blue, thick, each nth point=3] table[x=time, y=pv, col sep=comma, 
    restrict x to domain=0:466] {../data/3-4.csv};
\addlegendentry{PV (ON-OFF)}
\end{axis}
\end{tikzpicture}
\caption{Đáp ứng bộ điều khiển ON-OFF}
\label{fig:exp_onoff}
\end{figure}

\begin{table}[H]
\centering
\begin{tabular}{|l|c|}
\hline
\textbf{Thông số} & \textbf{Giá trị} \\
\hline
Thời gian xác lập & 104s \\
\hline
Chu kì dao động & 60s \\
\hline
Độ quá điều chỉnh & 12.9\% \\
\hline
Biên độ dao động & 386 - 508 \\
\hline
\end{tabular}
\caption{Kết quả bộ điều khiển ON-OFF}
\end{table}

\textbf{Nhận xét:} Bộ điều khiển ON/OFF cho giá trị cuối của biến điều khiển không ổn định mà dao động quanh giá trị đặt SP = 450 với biên độ dao động từ 386 đến 508. Điều này phù hợp với lý thuyết về điều khiển ON-OFF.

\subsubsection{Kết quả thí nghiệm 2: Bộ điều khiển P, PI, PID}

% P Controller - Exp 3
\begin{figure}[H]
\centering
\begin{tikzpicture}
\begin{axis}[
    width=14cm,
    height=6cm,
    xlabel={Thời gian (giây)},
    ylabel={Giá trị},
    grid=both,
    grid style={gray!30},
    ymin=-50,
    ymax=550,
    xmin=607,
    xmax=1200,
    legend pos=south east,
    legend style={font=\small},
]
% Setpoint line
\addplot[red, dashed, thick] coordinates {(607,450) (1200,450)};
\addlegendentry{SP = 450}

% Process variable - Exp 3 (P)
\addplot[blue, thick, each nth point=3] table[x=time, y=pv, col sep=comma, 
    restrict x to domain=607:1200] {../data/3-4.csv};
\addlegendentry{PV (P)}
\end{axis}
\end{tikzpicture}
\caption{Đáp ứng bộ điều khiển P}
\label{fig:exp_p}
\end{figure}

% PI Controller - Exp 4
\begin{figure}[H]
\centering
\begin{tikzpicture}
\begin{axis}[
    width=14cm,
    height=6cm,
    xlabel={Thời gian (giây)},
    ylabel={Giá trị},
    grid=both,
    grid style={gray!30},
    ymin=-150,
    ymax=550,
    xmin=1200,
    xmax=1920,
    legend pos=south east,
    legend style={font=\small},
]
% Setpoint line
\addplot[red, dashed, thick] coordinates {(1200,450) (1920,450)};
\addlegendentry{SP = 450}

% Process variable - Exp 4 (PI)
\addplot[blue, thick, each nth point=3] table[x=time, y=pv, col sep=comma, 
    restrict x to domain=1200:1920] {../data/3-4.csv};
\addlegendentry{PV (PI)}
\end{axis}
\end{tikzpicture}
\caption{Đáp ứng bộ điều khiển PI}
\label{fig:exp_pi}
\end{figure}

% PID Controller - Exp 5
\begin{figure}[H]
\centering
\begin{tikzpicture}
\begin{axis}[
    width=14cm,
    height=6cm,
    xlabel={Thời gian (giây)},
    ylabel={Giá trị},
    grid=both,
    grid style={gray!30},
    ymin=-50,
    ymax=550,
    xmin=1920,
    xmax=2300,
    legend pos=south east,
    legend style={font=\small},
]
% Setpoint line
\addplot[red, dashed, thick] coordinates {(1920,450) (2300,450)};
\addlegendentry{SP = 450}

% Process variable - Exp 5 (PID)
\addplot[blue, thick, each nth point=3] table[x=time, y=pv, col sep=comma, 
    restrict x to domain=1920:2300] {../data/3-4.csv};
\addlegendentry{PV (PID)}
\end{axis}
\end{tikzpicture}
\caption{Đáp ứng bộ điều khiển PID}
\label{fig:exp_pid}
\end{figure}

\begin{table}[H]
\centering
\begin{tabular}{|c|c|c|c|c|}
\hline
\textbf{Bộ điều khiển} & \textbf{Thời gian xác lập} & \textbf{Chu kì} & \textbf{Độ quá điều chỉnh} & \textbf{Sai lệch tĩnh} \\
\hline
P & 117s & -- & 0\% & Có ($e_{ss} \approx 19$) \\
\hline
PI & 272s & -- & 3.6\% & Không \\
\hline
PID & 158s & -- & 0.2\% & Không ($e_{ss} \approx 3$) \\
\hline
\end{tabular}
\caption{So sánh các bộ điều khiển P, PI và PID}
\end{table}

\subsubsection{Bàn luận}

\begin{itemize}
    \item \textbf{Bộ điều khiển P:} Có thời gian xác lập nhanh nhất (117s), nhưng tồn tại sai lệch tĩnh ($e_{ss} \approx 19$) do thiếu thành phần tích phân.
    \item \textbf{Bộ điều khiển PI:} Triệt tiêu được sai lệch tĩnh nhờ thành phần tích phân, nhưng thời gian xác lập dài nhất (272s) và có độ quá điều chỉnh 3.6\%.
    \item \textbf{Bộ điều khiển PID:} Cho kết quả cân bằng với thời gian xác lập trung bình (158s), độ quá điều chỉnh rất nhỏ (0.2\%), và sai lệch tĩnh gần như triệt tiêu ($e_{ss} \approx 3$).
\end{itemize}

% ------------------------------------------------------------
\subsection{Kết luận và khuyến nghị}
% ------------------------------------------------------------

\textbf{Kết luận:}
\begin{itemize}
    \item Bộ điều khiển ON-OFF đơn giản nhưng gây dao động liên tục quanh điểm đặt.
    \item Bộ điều khiển P phản ứng nhanh nhưng có sai lệch tĩnh.
    \item Bộ điều khiển PI triệt tiêu được sai lệch tĩnh nhưng đáp ứng chậm.
    \item Bộ điều khiển PID cho chất lượng điều khiển tốt nhất với thời gian xác lập ngắn và không có sai lệch tĩnh.
\end{itemize}

\textbf{Khuyến nghị:} Trong thực tế, nên sử dụng bộ điều khiển PID cho các ứng dụng yêu cầu chất lượng điều khiển cao. Việc chỉnh định thông số PID cần được thực hiện cẩn thận để đạt được đáp ứng tối ưu.

% ------------------------------------------------------------
\subsection{Trả lời câu hỏi kiểm tra}
% ------------------------------------------------------------

\textbf{Câu 1: So sánh ưu nhược điểm của bộ điều khiển ON-OFF và PID?}

\begin{itemize}
    \item ON-OFF: Ưu điểm là đơn giản, dễ triển khai. Nhược điểm là gây dao động liên tục, không phù hợp với quá trình yêu cầu độ chính xác cao.
    \item PID: Ưu điểm là cho đáp ứng mượt mà, triệt tiêu sai lệch tĩnh, có thể tối ưu hóa đáp ứng. Nhược điểm là phức tạp hơn, cần chỉnh định thông số.
\end{itemize}

\textbf{Câu 2: Giải thích vai trò của từng thành phần P, I, D?}

\begin{itemize}
    \item P: Tạo tác động tỉ lệ với sai lệch, giúp phản ứng nhanh nhưng không triệt tiêu được sai lệch tĩnh.
    \item I: Tích phân sai lệch theo thời gian, triệt tiêu sai lệch tĩnh nhưng có thể gây dao động.
    \item D: Vi phân sai lệch, dự đoán xu hướng và giảm độ quá điều chỉnh.
\end{itemize}

\textbf{Câu 3: Tại sao bộ điều khiển ON-OFF gây dao động?}

Do bộ điều khiển chỉ có hai trạng thái (bật/tắt), khi biến điều khiển vượt qua điểm đặt, bộ điều khiển chuyển trạng thái nhưng do quán tính của quá trình, biến điều khiển tiếp tục thay đổi trước khi đảo chiều, gây ra dao động liên tục.

% ------------------------------------------------------------
\subsection{Tài liệu tham khảo}
% ------------------------------------------------------------

\begin{enumerate}
    \item Điều khiển Quá trình Công nghệ Hóa học - Cơ sở điều khiển Quá trình – Quyển 1.
    \item Điều khiển Quá trình Công nghệ Hóa học – Hướng dẫn thí nghiệm, Thực hành cơ sở Điều khiển – Quyển 2.
    \item Åström, K.J. and Hägglund, T., PID Controllers: Theory, Design, and Tuning, ISA, 1995.
\end{enumerate}

\newpage
