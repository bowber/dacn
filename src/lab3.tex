% ============================================================
% Lab 3: Điều khiển PID
% ============================================================

\setcounter{section}{3}
\setcounter{subsection}{0}
\section*{BÀI THÍ NGHIỆM 3: BỘ ĐIỀU KHIỂN PID}
\addcontentsline{toc}{section}{Bài thí nghiệm 3: Bộ điều khiển PID}

% ------------------------------------------------------------
\subsection{Mục tiêu}
% ------------------------------------------------------------

\begin{itemize}
    \item Hiểu nguyên lý hoạt động của bộ điều khiển PID.
    \item Cấu hình và sử dụng khối PID trong TIA Portal.
    \item Chỉnh định thông số PID và đánh giá chất lượng điều khiển.
\end{itemize}

% ------------------------------------------------------------
\subsection{Cơ sở lý thuyết}
% ------------------------------------------------------------

\subsubsection{Bộ điều khiển PID}

Bộ điều khiển PID là bộ điều khiển đưa ra tác động điều khiển dựa trên ba thành phần: tỉ lệ (P), tích phân (I) và vi phân (D).

\begin{figure}[H]
\centering
\begin{tikzpicture}[node distance=1.5cm]
    % Input
    \node (input) {$e(t)$};
    
    % P, I, D blocks
    \node[block, right=1.5cm of input, yshift=1.2cm] (P) {$K_P$};
    \node[block, right=1.5cm of input] (I) {$\frac{K_P}{T_I} \int$};
    \node[block, right=1.5cm of input, yshift=-1.2cm] (D) {$K_P \cdot T_D \frac{d}{dt}$};
    
    % Sum
    \node[sum, right=1.5cm of I] (sum) {};
    
    % Output
    \node[right=1cm of sum] (output) {$u(t)$};
    
    % Arrows from input
    \coordinate (split) at ($(input.east)+(0.5,0)$);
    \draw[line] (input) -- (split);
    \draw[arrow] (split) |- (P);
    \draw[arrow] (split) -- (I);
    \draw[arrow] (split) |- (D);
    
    % Arrows to sum
    \draw[arrow] (P) -| (sum);
    \draw[arrow] (I) -- (sum);
    \draw[arrow] (D) -| (sum);
    
    % Output arrow
    \draw[arrow] (sum) -- (output);
    
    % Sum signs
    \node[font=\scriptsize] at ($(sum.north)+(0,0.15)$) [right] {$+$};
    \node[font=\scriptsize] at ($(sum.west)+(-0.15,0)$) [below] {$+$};
    \node[font=\scriptsize] at ($(sum.south)+(0,-0.15)$) [right] {$+$};
\end{tikzpicture}
\caption{Sơ đồ khối bộ điều khiển PID}
\label{fig:pid_block_lab3}
\end{figure}

Công thức tổng quát:
\begin{equation}
u(t) = K_P \left( e(t) + \frac{1}{T_I} \int_0^t e(\tau)d\tau + T_D \frac{de(t)}{dt} \right)
\end{equation}

\subsubsection{Ảnh hưởng của các thông số}

\begin{table}[H]
\centering
\begin{tabular}{|c|c|c|c|c|}
\hline
\textbf{Thay đổi} & \textbf{Thời gian đáp ứng} & \textbf{Độ quá điều chỉnh} & \textbf{Sai lệch tĩnh} & \textbf{Ổn định} \\
\hline
Tăng $K_P$ & Giảm & Tăng & Giảm & Xấu đi \\
\hline
Giảm $T_I$ & Giảm & Tăng & Triệt tiêu & Xấu đi \\
\hline
Tăng $T_D$ & Ít thay đổi & Giảm & Không đổi & Tốt hơn \\
\hline
\end{tabular}
\caption{Ảnh hưởng của các thông số PID}
\end{table}

\subsubsection{Phương pháp chỉnh định Ziegler-Nichols và Relay}

Có hai phương pháp phổ biến để xác định thông số PID dựa trên dao động tới hạn:

\textbf{Phương pháp Ziegler-Nichols gốc (1942):}

Phương pháp này được đề xuất bởi John G. Ziegler và Nathaniel B. Nichols năm 1942. Các bước thực hiện:
\begin{enumerate}
    \item Đặt $T_i = \infty$ (tắt I), $T_d = 0$ (tắt D), chỉ dùng điều khiển P
    \item Tăng dần $K_P$ từ 0 cho đến khi hệ thống bắt đầu dao động điều hòa với biên độ không đổi
    \item Ghi nhận $K_u$ (ultimate gain) và $T_u$ (chu kỳ dao động)
    \item Tính thông số PID theo bảng Ziegler-Nichols
\end{enumerate}

\textbf{Hạn chế:} Phương pháp gốc yêu cầu đưa hệ thống đến biên giới mất ổn định, có thể gây nguy hiểm cho thiết bị và quá trình.

\textbf{Phương pháp Relay (Åström-Hägglund, 1984):}

Phương pháp này được đề xuất bởi Karl Johan Åström và Tore Hägglund năm 1984, khắc phục hạn chế của phương pháp gốc bằng cách sử dụng điều khiển relay (ON/OFF):
\begin{enumerate}
    \item Điều khiển hệ thống bằng relay quanh setpoint
    \item Đo chu kỳ dao động $T_u$ và biên độ dao động $a$ của biến quá trình
    \item Tính $K_u$ theo công thức:
    \begin{equation}
    K_u = \frac{4d}{\pi a}
    \end{equation}
    trong đó $d$ là biên độ relay (từ tâm đến đỉnh), $a$ là biên độ dao động ngõ ra
    \item Áp dụng bảng Ziegler-Nichols để tính $K_P$, $T_i$, $T_d$
\end{enumerate}

\textbf{Ưu điểm:} Không cần đưa hệ thống đến biên giới mất ổn định; dao động được kiểm soát bởi relay; an toàn hơn cho thiết bị.

\textbf{So sánh hai phương pháp:}

\begin{table}[H]
\centering
\begin{tabular}{|l|c|c|}
\hline
\textbf{Tiêu chí} & \textbf{Z-N gốc (1942)} & \textbf{Relay (1984)} \\
\hline
Người đề xuất & Ziegler \& Nichols & Åström \& Hägglund \\
\hline
Cách tạo dao động & Tăng $K_P$ đến mất ổn định & Điều khiển ON/OFF \\
\hline
Xác định $K_u$ & $K_u$ = giá trị $K_P$ tới hạn & $K_u = 4d/(\pi a)$ \\
\hline
Mức độ an toàn & Rủi ro cao & An toàn hơn \\
\hline
Ứng dụng thực tế & Hạn chế & Phổ biến (auto-tuning) \\
\hline
\end{tabular}
\caption{So sánh phương pháp Ziegler-Nichols gốc và Relay}
\end{table}

\textbf{Lưu ý:} Trong bài thí nghiệm này, chúng ta sử dụng phương pháp Relay (Åström-Hägglund).

\subsubsection{Khối PID\_Compact trong TIA Portal}

TIA Portal cung cấp khối \texttt{PID\_Compact} để thực hiện điều khiển PID. Các tham số chính:
\begin{itemize}
    \item \texttt{Setpoint}: Giá trị đặt
    \item \texttt{Input}: Giá trị phản hồi (PV)
    \item \texttt{Output}: Tín hiệu điều khiển
    \item \texttt{Gain} ($K_P$): Hệ số khuếch đại
    \item \texttt{Ti}: Thời gian tích phân
    \item \texttt{Td}: Thời gian vi phân
\end{itemize}

% ------------------------------------------------------------
\subsection{Thiết bị thí nghiệm}
% ------------------------------------------------------------

\begin{itemize}
    \item PLC Siemens S7-1200 CPU 1213C
    \item Module analog input/output
    \item Cảm biến nhiệt độ
    \item Bộ gia nhiệt (Heater)
    \item Phần mềm TIA Portal V16
\end{itemize}

% ------------------------------------------------------------
\subsection{Quy trình thực hiện}
% ------------------------------------------------------------

\subsubsection{Cấu hình khối PID\_Compact}
\begin{enumerate}
    \item Thêm khối \texttt{PID\_Compact} vào chương trình.
    \item Cấu hình các tham số đầu vào/ra.
    \item Thiết lập giới hạn Output (0-100\%).
    \item Kết nối Setpoint và Input với biến tương ứng.
\end{enumerate}

\subsubsection{Chỉnh định thông số PID theo phương pháp Relay (Åström-Hägglund)}

Sử dụng dữ liệu từ thí nghiệm điều khiển ON/OFF (Bài 2) để tính toán thông số PID theo phương pháp Relay:

\textbf{Bước 1: Xác định các thông số từ thí nghiệm ON/OFF}
\begin{itemize}
    \item Chu kỳ dao động tới hạn: $T_u = 250$s (từ Bài thí nghiệm 2)
    \item Biên độ dao động: $a = 6.5^\circ$C
    \item Biên độ relay: $d = 50\%$ (relay dao động giữa 0\% và 100\%, biên độ từ tâm = 50\%)
\end{itemize}

\textbf{Bước 2: Tính Ultimate Gain}
\begin{equation}
K_u = \frac{4d}{\pi a} = \frac{4 \times 50}{\pi \times 6.5} \approx 9.8
\end{equation}

\textbf{Bước 3: Tính thông số PID theo bảng Ziegler-Nichols}
\begin{align}
K_P &= 0.6 \times K_u = 0.6 \times 9.8 \approx 5.9 \\
T_i &= \frac{T_u}{2} = \frac{250}{2} = 125 \text{ s} \\
T_d &= \frac{T_u}{8} = \frac{250}{8} = 31.25 \approx 31 \text{ s}
\end{align}

\textbf{Bộ thông số PID được sử dụng:} $K_P = 5.9$, $T_i = 125$s, $T_d = 31$s

% ------------------------------------------------------------
\subsection{Kết quả thí nghiệm}
% ------------------------------------------------------------

\subsubsection{Giao diện HMI}

\begin{figure}[H]
\centering
\includegraphics[width=0.9\textwidth]{../assets/HMI_TN3.jpg}
\caption{Giao diện HMI thí nghiệm điều khiển PID}
\label{fig:hmi_tn3}
\end{figure}

\subsubsection{Đáp ứng điều khiển PID}

Thí nghiệm điều khiển PID với bộ thông số $K_P = 5.9$, $T_i = 125$s, $T_d = 31$s. Hệ thống được gia nhiệt đến SV = $40^\circ$C, sau đó thực hiện thay đổi bước (step change) lên SV = $50^\circ$C để đánh giá chất lượng điều khiển.

\begin{figure}[H]
\centering
\begin{tikzpicture}
\begin{axis}[
    width=0.9\textwidth,
    height=8cm,
    xlabel={Thời gian (s)},
    ylabel={Nhiệt độ ($^\circ$C)},
    grid=both,
    grid style={line width=.1pt, draw=gray!30},
    major grid style={line width=.2pt,draw=gray!50},
    xmin=0, xmax=1600,
    ymin=20, ymax=55,
    legend pos=south east,
]
\addplot[blue, thick] table[x=time_s, y=PV, col sep=comma] {../data/TN3_processed.csv};
\addlegendentry{PV (Nhiệt độ đo được)}
\addplot[red, dashed, thick] table[x=time_s, y=SV, col sep=comma] {../data/TN3_processed.csv};
\addlegendentry{SV (Giá trị đặt)}

% Đánh dấu thời điểm thay đổi setpoint
\draw[gray, dashed] (axis cs:1054,20) -- (axis cs:1054,55);
\node[font=\small] at (axis cs:1054,22) [right] {Step change};

% Đánh dấu thông số PID
\node[font=\small, fill=white] at (axis cs:500,52) {$K_P=5.9$, $T_i=125$s, $T_d=31$s};
\end{axis}
\end{tikzpicture}
\caption{Đáp ứng điều khiển PID với thay đổi bước setpoint ($40^\circ$C $\rightarrow$ $50^\circ$C)}
\label{fig:pid_response_tn3}
\end{figure}

Phân tích đáp ứng:
\begin{itemize}
    \item \textbf{Giai đoạn 1 (0-1049s):} Gia nhiệt từ nhiệt độ môi trường ($22.8^\circ$C) đến SV = $40^\circ$C
    \begin{itemize}
        \item Thời gian đạt setpoint: 180s
        \item Độ quá điều chỉnh: $2.09^\circ$C (12.2\%)
        \item Thời gian xác lập ($\pm 2\%$): 606s
        \item Sai lệch tĩnh: $\approx 0^\circ$C
    \end{itemize}
    \item \textbf{Giai đoạn 2 (1054-1556s):} Thay đổi bước từ SV = $40^\circ$C lên SV = $50^\circ$C
    \begin{itemize}
        \item Thời gian tăng (rise time): 466s
        \item Độ quá điều chỉnh: $0.37^\circ$C (3.7\%)
        \item Thời gian xác lập ($\pm 2\%$): 231s
        \item Sai lệch tĩnh: $0.14^\circ$C
    \end{itemize}
\end{itemize}

\subsubsection{Kết quả chỉnh định PID}

Bộ thông số PID được sử dụng trong thí nghiệm:

\begin{table}[H]
\centering
\begin{tabular}{|c|c|c|c|}
\hline
\textbf{Thông số} & \textbf{Giá trị} & \textbf{Đơn vị} & \textbf{Ghi chú} \\
\hline
$K_P$ & 5.9 & -- & Hệ số khuếch đại \\
\hline
$T_i$ & 125 & s & Thời gian tích phân \\
\hline
$T_d$ & 31 & s & Thời gian vi phân \\
\hline
\end{tabular}
\caption{Thông số PID sử dụng trong thí nghiệm}
\end{table}

\subsubsection{Chỉ tiêu chất lượng điều khiển}

\begin{table}[H]
\centering
\begin{tabular}{|l|c|c|}
\hline
\textbf{Chỉ tiêu} & \textbf{Giai đoạn 1} & \textbf{Giai đoạn 2} \\
 & (SV = $40^\circ$C) & (Step: $40 \rightarrow 50^\circ$C) \\
\hline
Thời gian đạt setpoint & 180s & 466s \\
\hline
Độ quá điều chỉnh & 12.2\% ($2.09^\circ$C) & 3.7\% ($0.37^\circ$C) \\
\hline
Thời gian xác lập ($\pm 2\%$) & 606s & 231s \\
\hline
Sai lệch tĩnh & $\approx 0^\circ$C & $0.14^\circ$C \\
\hline
\end{tabular}
\caption{Chỉ tiêu chất lượng điều khiển PID}
\end{table}

\subsubsection{Đánh giá hiệu suất tổng thể}

\begin{table}[H]
\centering
\begin{tabular}{|l|c|}
\hline
\textbf{Chỉ tiêu} & \textbf{Giá trị} \\
\hline
Sai số trung bình tuyệt đối (MAE) & $2.22^\circ$C \\
\hline
Sai số tối đa & $17.16^\circ$C \\
\hline
MAE trạng thái xác lập & $0.12^\circ$C \\
\hline
\end{tabular}
\caption{Đánh giá hiệu suất tổng thể}
\end{table}

% ------------------------------------------------------------
\subsection{Bàn luận}
% ------------------------------------------------------------

\subsubsection{Phân tích kết quả điều khiển}

Bộ thông số PID ($K_P = 5.9$, $T_i = 125$s, $T_d = 31$s) cho kết quả:

\begin{itemize}
    \item \textbf{Hệ số khuếch đại $K_P = 5.9$:} Đúng theo bảng Ziegler-Nichols ($0.6 \times K_u$). Độ quá điều chỉnh 12.2\% trong giai đoạn khởi động là chấp nhận được.
    
    \item \textbf{Thời gian tích phân $T_i = 125$s:} Đúng theo bảng Ziegler-Nichols ($T_u/2$), giúp triệt tiêu sai lệch tĩnh hiệu quả (sai lệch tĩnh $\approx 0^\circ$C).
    
    \item \textbf{Thời gian vi phân $T_d = 31$s:} Đúng theo bảng Ziegler-Nichols ($T_u/8$), giúp giảm độ quá điều chỉnh trong giai đoạn step change xuống còn 3.7\%.
\end{itemize}

\subsubsection{So sánh với điều khiển ON/OFF (Bài 2)}

\begin{table}[H]
\centering
\begin{tabular}{|l|c|c|}
\hline
\textbf{Tiêu chí} & \textbf{ON/OFF (Bài 2)} & \textbf{PID (Bài 3)} \\
\hline
Biên độ dao động & $\pm 6.5^\circ$C & $< 0.5^\circ$C (xác lập) \\
\hline
Độ quá điều chỉnh & 50\% ($5.5^\circ$C) & 12.2\% ($2.09^\circ$C) \\
\hline
Sai lệch tĩnh & Dao động liên tục & $\approx 0^\circ$C \\
\hline
Chu kỳ dao động & 250s & Không dao động \\
\hline
Chất lượng điều khiển & Thấp & Cao \\
\hline
\end{tabular}
\caption{So sánh điều khiển ON/OFF và PID}
\end{table}

\textbf{Nhận xét:}
\begin{itemize}
    \item Điều khiển PID cải thiện đáng kể chất lượng so với ON/OFF: giảm biên độ dao động từ $\pm 6.5^\circ$C xuống gần như bằng 0 trong trạng thái xác lập
    \item PID triệt tiêu hoàn toàn sai lệch tĩnh, trong khi ON/OFF luôn dao động quanh setpoint
    \item Thông số PID tính từ dữ liệu ON/OFF theo phương pháp Relay cho kết quả tốt, xác nhận tính hiệu quả của phương pháp
\end{itemize}

\subsubsection{Đánh giá chất lượng điều khiển}

\begin{itemize}
    \item \textbf{Độ quá điều chỉnh:} Giai đoạn khởi động có độ quá điều chỉnh 12.2\% (chấp nhận được cho hệ thống nhiệt). Giai đoạn step change có độ quá điều chỉnh thấp hơn (3.7\%), cho thấy bộ điều khiển hoạt động tốt hơn khi hệ thống đã ổn định.
    
    \item \textbf{Thời gian xác lập:} 606s cho giai đoạn khởi động và 231s cho step change. Thời gian này phù hợp với đặc tính của hệ thống nhiệt có quán tính lớn.
    
    \item \textbf{Sai lệch tĩnh:} Gần như bằng 0 trong cả hai giai đoạn ($<0.15^\circ$C), cho thấy thành phần tích phân hoạt động hiệu quả.
    
    \item \textbf{Độ ổn định:} Hệ thống ổn định, không có dao động kéo dài. Đáp ứng mượt mà và không có hiện tượng mất ổn định.
\end{itemize}

% ------------------------------------------------------------
\subsection{Kết luận}
% ------------------------------------------------------------

Qua thí nghiệm điều khiển PID hệ thống gia nhiệt, rút ra các kết luận sau:

\begin{enumerate}
    \item \textbf{Phương pháp chỉnh định:} Thông số PID được tính toán theo phương pháp Relay (Åström-Hägglund), sử dụng dữ liệu từ thí nghiệm ON/OFF (Bài 2):
    \begin{itemize}
        \item Chu kỳ dao động $T_u = 250$s, biên độ $a = 6.5^\circ$C, biên độ relay $d = 50\%$
        \item Ultimate gain $K_u = 4d/(\pi a) = 9.8$
        \item Áp dụng bảng Ziegler-Nichols: $K_P = 0.6 K_u = 5.9$, $T_i = T_u/2 = 125$s, $T_d = T_u/8 = 31$s
    \end{itemize}
    
    \item \textbf{Kết quả điều khiển giai đoạn khởi động (SV = $40^\circ$C):}
    \begin{itemize}
        \item Thời gian đạt setpoint: 180s
        \item Độ quá điều chỉnh: 12.2\% ($2.09^\circ$C)
        \item Thời gian xác lập: 606s
        \item Sai lệch tĩnh: $\approx 0^\circ$C
    \end{itemize}
    
    \item \textbf{Kết quả điều khiển khi thay đổi bước ($40 \rightarrow 50^\circ$C):}
    \begin{itemize}
        \item Thời gian tăng: 466s
        \item Độ quá điều chỉnh: 3.7\% ($0.37^\circ$C)
        \item Thời gian xác lập: 231s
        \item Sai lệch tĩnh: $0.14^\circ$C
    \end{itemize}
    
    \item \textbf{So sánh với ON/OFF:} Điều khiển PID cải thiện đáng kể so với ON/OFF:
    \begin{itemize}
        \item Giảm biên độ dao động từ $\pm 6.5^\circ$C xuống $< 0.5^\circ$C
        \item Triệt tiêu sai lệch tĩnh (từ dao động liên tục xuống $\approx 0^\circ$C)
        \item Hệ thống ổn định, không có chu kỳ dao động
    \end{itemize}
    
    \item \textbf{Kết luận:} Phương pháp Relay (Åström-Hägglund) cho phép tính toán thông số PID hiệu quả từ dữ liệu thí nghiệm ON/OFF, không cần sử dụng Auto-tuning.
\end{enumerate}

% ------------------------------------------------------------
\subsection{Câu hỏi kiểm tra}
% ------------------------------------------------------------

\textbf{Câu 1: Anh/chị hãy thử tạo một khối chức năng FB và chạy thử thuật toán PID mà không sử dụng Cyclic interrupt. Thuật toán đó có chạy được không? Anh/chị có rút ra nhận xét gì về nguyên lý của khối Cyclic interrupt trong thuật toán điều khiển PID?}

\textbf{Thử nghiệm PID không sử dụng Cyclic interrupt:}

Khi tạo khối FB chứa thuật toán PID và gọi trong OB1 (Main) thay vì Cyclic interrupt OB:
\begin{itemize}
    \item Thuật toán PID \textbf{có thể chạy được} nhưng \textbf{kết quả không chính xác}
    \item Thành phần tích phân (I) và vi phân (D) tính toán sai do chu kỳ lấy mẫu không cố định
    \item Đáp ứng điều khiển không ổn định, có thể dao động hoặc mất ổn định
\end{itemize}

\textbf{Nhận xét về vai trò của Cyclic interrupt trong điều khiển PID:}

\begin{enumerate}
    \item \textbf{Đảm bảo chu kỳ lấy mẫu cố định:}
    \begin{itemize}
        \item Cyclic interrupt (OB30-OB38) được gọi với chu kỳ cố định (ví dụ: 100ms)
        \item Thành phần I cần tích phân theo thời gian: $\int e(t)dt \approx \sum e_k \cdot T_s$
        \item Thành phần D cần vi phân theo thời gian: $\frac{de}{dt} \approx \frac{e_k - e_{k-1}}{T_s}$
        \item Nếu $T_s$ không cố định, tính toán I và D sẽ sai
    \end{itemize}
    
    \item \textbf{Độ ưu tiên cao:}
    \begin{itemize}
        \item Cyclic interrupt có độ ưu tiên cao hơn OB1
        \item Đảm bảo thuật toán PID luôn được thực thi đúng thời điểm
        \item Không bị ảnh hưởng bởi thời gian thực thi của các tác vụ khác
    \end{itemize}
    
    \item \textbf{Tính xác định (Deterministic):}
    \begin{itemize}
        \item Hệ thống điều khiển yêu cầu thời gian đáp ứng xác định
        \item OB1 có thời gian thực thi thay đổi tùy thuộc vào chương trình
        \item Cyclic interrupt đảm bảo PID luôn được tính toán đúng chu kỳ
    \end{itemize}
\end{enumerate}

\textbf{Kết luận:}

Cyclic interrupt là \textbf{bắt buộc} cho điều khiển PID vì:
\begin{itemize}
    \item Công thức PID số (discrete PID) phụ thuộc vào chu kỳ lấy mẫu $T_s$
    \item Nếu $T_s$ không cố định, các thông số $K_P$, $T_i$, $T_d$ đã chỉnh định sẽ không còn phù hợp
    \item Khối PID\_Compact của Siemens yêu cầu phải được gọi trong Cyclic interrupt OB
\end{itemize}

% ------------------------------------------------------------
\subsection{Tài liệu tham khảo}
% ------------------------------------------------------------

\begin{enumerate}[label={[\arabic*]}]
    \item B. N. Pha, \textit{Điều khiển quá trình công nghệ hoá học - Quyển 1: Cơ sở điều khiển quá trình}. Khoa Kỹ thuật Hóa học, Trường Đại học Bách khoa, ĐHQG-HCM, 2021.
    \item K. J. Åström and T. Hägglund, ``Automatic Tuning of Simple Regulators with Specifications on Phase and Amplitude Margins,'' \textit{Automatica}, vol. 20, no. 5, pp. 645--651, 1984. DOI: 10.1016/0005-1098(84)90014-1
\end{enumerate}

\newpage
